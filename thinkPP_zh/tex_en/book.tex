% LaTeX source for textbook ``How to think like a (Python) Programmer''
% Copyright (c)  2007  Allen B. Downey.

% Permission is granted to copy, distribute and/or modify this
% document under the terms of the GNU Free Documentation License,
% Version 1.1  or any later version published by the Free Software
% Foundation; with no Invariant Sections, no Front-Cover Texts,
% and no Back-Cover Texts.

% This distribution includes a file named fdl.tex that contains the text
% of the GNU Free Documentation License.  If it is missing, you can obtain
% it from www.gnu.org or by writing to the Free Software Foundation,
% Inc., 59 Temple Place - Suite 330, Boston, MA 02111-1307, USA.
%
\documentclass[10pt]{book}
%\documentclass[cup6a]{cupbook}
\usepackage{url}
\usepackage{fancyhdr}
\usepackage{graphicx}
\usepackage{amsmath, amsthm, amssymb}
\usepackage{makeidx}
\usepackage{setspace}
\usepackage{hevea}

%\usepackage[draft,light]{draftcopy}

\newtheorem{ex}{Exercise}[chapter]

\newcommand{\thetitle}{How to Think Like a\\ (Python) Programmer}
\newcommand{\theversion}{0.9.15}

\makeindex

\begin{document}

\frontmatter


% LATEXONLY

\input{latexonly}

\begin{latexonly}

\renewcommand{\blankpage}{\thispagestyle{empty} \quad \newpage}
\blankpage
\blankpage

% TITLE PAGES FOR LATEX VERSION

%-half title--------------------------------------------------
\thispagestyle{empty}

\begin{flushright}
\vspace*{2.5in}

\begin{spacing}{3}
{\huge \thetitle}
\end{spacing}

\vspace{0.25in}

Version \theversion

\vfill

\end{flushright}

%--verso------------------------------------------------------

\blankpage
\blankpage
%\clearemptydoublepage
%\pagebreak
%\thispagestyle{empty}
%\vspace*{6in}

%--title page--------------------------------------------------
\pagebreak
\thispagestyle{empty}

\begin{flushright}
\vspace*{2.5in}

\begin{spacing}{3}
{\huge \thetitle}
\end{spacing}

\vspace{0.25in}

Version \theversion

\vspace{1in}


{\Large
Allen Downey\\
}


\vspace{0.5in}

{\Large Green Tea Press}

{\small Needham, Massachusetts}

%\includegraphics[width=1in]{figs/logo1.eps}
\vfill

\end{flushright}


%--copyright--------------------------------------------------
\pagebreak
\thispagestyle{empty}

{\small
Copyright \copyright ~2007 Allen Downey.


Printing history:

\begin{description}

\item[April 2002:] First edition of {\em How to Think Like
a Computer Scientist}.

\item[August 2007:] Major revision, changed title to
{\em How to Think Like a (Python) Programmer}.

\end{description}

\vspace{0.2in}

\begin{flushleft}
Green Tea Press       \\
9 Washburn Ave \\
Needham MA 02492
\end{flushleft}

Permission is granted to copy, distribute, and/or modify this document
under the terms of the GNU Free Documentation License, Version 1.1 or
any later version published by the Free Software Foundation; with no
Invariant Sections, no Front-Cover Texts, and with no Back-Cover Texts.

The GNU Free Documentation License is available from {\tt www.gnu.org}
or by writing to the Free Software Foundation, Inc., 59 Temple Place,
Suite 330, Boston, MA 02111-1307, USA.

The original form of this book is \LaTeX\ source code.  Compiling this
\LaTeX\ source has the effect of generating a device-independent
representation of a textbook, which can be converted to other formats
and printed.

The \LaTeX\ source for this book is available from
{\tt http://www.thinkpython.com}

\vspace{0.2in}

} % end small

\end{latexonly}


% HTMLONLY

\begin{htmlonly}

% TITLE PAGE FOR HTML VERSION

{\Huge \thetitle}

{\Large Allen B. Downey}

Version \theversion

\setcounter{chapter}{-1}

\end{htmlonly}

\chapter{Preface}

\section*{The strange history of this book}

In January 1999 I was preparing to teach an introductory programming
class in Java.  I had taught it three times and I was getting
frustrated.  The failure rate in the class was too high and even for
students who succeeded, the overall level of achievement was too low.

One of the problems I saw was the books.  I had tried three different
books (and read a dozen more), and they all had the same problems.
They were too big, with too much unnecessary detail about Java, and
not enough high-level guidance about how to program.  And they all
suffered from the trap door effect: they would start out very gradual
and easy, and then somewhere around Chapter 5, the bottom would
fall out.  The students would get too much new material, too fast,
and I would spend the rest of the semester picking up the pieces.

Two weeks before the first day of classes, I decided to write my
own book.  I wrote one 10-page chapter a day for 13 days.  I made
some revisions on Day 14 and then sent it out to be photocopied.

My goals were:

\begin{itemize}

\item Keep it short.  It is better for students to read 10 pages
than not read 50 pages.

\item Be careful with vocabulary.  I tried to minimize the jargon
and define each term at first use.

\item Build gradually. To avoid trap doors, I took the most difficult
topics and split them into a series of small steps. 

\item It's not about the language; it's about programming.  I included
the minumum useful subset of Java and left out the rest.

\end{itemize}

I needed a title, so on a whim I chose {\em How to Think Like
a Computer Scientist}.

My first version was rough, but it worked.  Students did the reading,
and they understood enough that I could spend class time on the hard
topics, the interesting topics and (most important) letting the
students practice.

As a user and advocate of free software, I believe in the idea
Benjamin Franklin expressed:

\begin{quote}
``As we enjoy great Advantages from the Inventions of others,
we should be glad of an Opportunity to serve others by any
Invention of ours, and this we should do freely and generously.''
\end{quote}

\index{Franklin, Benjamin}

So I released the book under the GNU Free Documenation License,
which allows users to copy, modify, and distribute the book.

What happened next is the cool part.  Jeff Elkner, a high school
teacher in Virginia, adopted my book and translated it into
Python.  He sent me a copy of his translation, and I had the
unusual experience of learning Python by reading my own book.

Jeff and I revised the book, incorporated a case study by
Chris Meyers, and released {\em How to Think Like
a Computer Scientist: Learning with Python}, also under
the GNU Free Documenation License.

At the same time, my wife and I started Green Tea Press, which
distributes several of my books electronically, and sells 
{\em How to Think} in hard copy.

I have been teaching with this book for more than five years
now, and I have done a lot more Python programming.  I still like
the structure of the book, but for some time I have felt the
need to make changes:

\begin{itemize}

\item Some of the examples in the first edition work better
than others.  In my classes I have discarded the less
effective ones and developed improvements.

\item There are only a few exercises in the first edition.
Now I have five years of quizzes, exams and homeworks to choose
from.

\item I have been programming in Python for a while now and have a
better appreciation of idiomatic Python.  The book is still about
programming, not Python, but now I think the book gets more
leverage from the language.

\end{itemize}

At the same time, Jeff has been working on his own second
edition, customized for his classes.  Rather than cram everything
into one book (which may be how other books got so big),
we decided to work on different versions.  They are both under
the Free Documentation License, so users can choose one or
combine material from both.

For my version, I am using the revised title 
{\em How to Think Like a (Python) Programmer}.  This is a more
modest goal than the original, but it might be more accurate.


Allen B. Downey \\
Needham MA\\

Allen Downey is a Professor of Computer Science at 
the Franklin W. Olin College of Engineering.



% \section*{For the teacher}

% Swampy and UML


% \section*{For the student}

% Try out examples.

% Do the in-chapter examples.

% Where to get the code.



\section*{Contributor List}

To paraphrase the philosophy of the Free Software Foundation, this
book is free like free speech, but not necessarily free like free
pizza.  It came about because of a collaboration that would not have
been possible without the GNU Free Documentation License.  So we
thank the Free Software Foundation for developing this license
and, of course, making it available to us.

We also thank the more than 100 sharp-eyed and
thoughtful readers who have sent us suggestions and corrections over
the past few years.  In the spirit of free software, we decided to
express our gratitude in the form of a contributor list.  Unfortunately,
this list is not complete, but we are doing our best to keep it
up to date.

If you have a chance to look through the list, you should
realize that each person here has spared you and all subsequent
readers from the confusion of a technical error or a
less-than-transparent explanation, just by sending us a note.

Impossible as it may seem after so many corrections, there may still
be errors in this book.  If you should stumble across one, please
check the online version of the book at {\tt http://thinkpython.com},
which is the most up-to-date version.  If the error has not been
corrected, please take a minute to send us email at {\tt
feedback@thinkpython.com}.  If we make a change due to your
suggestion, you will appear in the next version of the contributor
list (unless you ask to be omitted).  Thank you!

\small

\begin{itemize}

\item Lloyd Hugh Allen sent in a correction to Section 8.4.
%He can be reached at: {\tt lha2@columbia.edu}

\item Yvon Boulianne sent in a correction of a semantic error in
Chapter 5.
%She can be reached at: {\tt mystic@monuniverse.net}

\item Fred Bremmer submitted a correction in Section 2.1.
%He can be reached at:  {\tt Fred.Bremmer@ubc.cu}

\item Jonah Cohen wrote the Perl scripts to convert the
LaTeX source for this book into beautiful HTML.

%His Web page is {\tt jonah.ticalc.org}
%and his email is {\tt JonahCohen@aol.com}

\item Michael Conlon sent in a grammar correction in Chapter 2
and an improvement in style in Chapter 1, and he initiated discussion
on the technical aspects of interpreters.

%Michael can be reached at: {\tt michael.conlon@sru.edu}

\item Benoit Girard sent in a
correction to a humorous mistake in Section 5.6.

%Benoit can be reached at:
%{\tt benoit.girard@gouv.qc.ca}

\item Courtney Gleason and Katherine Smith wrote {\tt horsebet.py},
which was used as a case study in an earlier version of the book.  Their
program can now be found on the website.

%Courtney can be reached at: {\tt
%orion1558@aol.com}

\item Lee Harr submitted more corrections than we have room to list
here, and indeed he should be listed as one of the principal editors
of the text.

%He can be reached at: {\tt missive@linuxfreemail.com}

\item James Kaylin is a student using the text. He has submitted
numerous corrections.

%James can be reached by email at: {\tt Jamarf@aol.com}

\item David Kershaw fixed the broken {\tt catTwice} function in Section
3.10.

%He can be reached at: {\tt david\_kershaw@merck.com}

\item Eddie Lam has sent in numerous corrections to Chapters 
1, 2, and 3.
He also fixed the Makefile so that it creates an index the first time it is
run and helped us set up a versioning scheme.  

%Eddie can be reached at:
%{\tt nautilus@yoyo.cc.monash.edu.au}

\item Man-Yong Lee sent in a correction to the example code in
Section 2.4.  

%He can be reaced at: {\tt yong@linuxkorea.co.kr}

\item David Mayo pointed out that the word ``unconsciously"
in Chapter 1 needed
to be changed to ``subconsciously".

%David can be reached at:{\tt bdbear44@netscape.net}

\item Chris McAloon sent in several corrections to Sections 3.9 and
3.10.

%He can be reached at: {\tt cmcaloon@ou.edu}

\item Matthew J. Moelter has been a long-time contributor who sent
in numerous corrections and suggestions to the book.  

%He can be reached at:
%{\tt mmoelter@calpoly.edu}

\item Simon Dicon Montford reported a missing function definition and
several typos in Chapter 3.  He also found errors in the {\tt increment}
function in Chapter 13.

%He can be reached at: {\tt dicon@bigfoot.com}

\item John Ouzts corrected the definition of ``return value"
in Chapter 3.

%He can be reached at: {\tt jouzts@bigfoot.com}

\item Kevin Parks sent in valuable comments and suggestions as to how
to improve the distribution of the book.

%He can be reached at: {\tt cpsoct@lycos.com}

\item David Pool sent in a typo in the glossary of Chapter 1, as well
as kind words of encouragement.

%He can be reached at: {\tt pooldavid@hotmail.com}

\item Michael Schmitt sent in a correction to the chapter on files
and exceptions.

%He can be reached at: {\tt ipv6\_128@yahoo.com}

\item Robin Shaw pointed out an error in Section 13.1, where the
printTime function was used in an example without being defined.

%Robin can be reached at: {\tt randj@iowatelecom.net}

\item Paul Sleigh found an error in Chapter 7 and a bug in Jonah Cohen's
Perl script that generates HTML from LaTeX.

%He can be reached at: {\tt bat@atdot.dotat.org}

%\item Christopher Smith is a computer science teacher at the Blake
%School in Minnesota who teaches Python to his beginning students.

%He can be reached at: {\tt csmith@blakeschool.org or smiles@saysomething.com}

\item Craig T. Snydal is testing the text in a course at Drew
University.  He has contributed several valuable suggestions and corrections.

%and can be reached at: {\tt csnydal@drew.edu}

\item Ian Thomas and his students are using the text in a programming
course.  They are the first ones to test the chapters in the latter half
of the book, and they have made numerous corrections and suggestions.

%Ian can be reached at: {\tt ithomas@sd70.bc.ca}

\item Keith Verheyden sent in a correction in Chapter 3.

%He can be reached at: {\tt kverheyd@glam.ac.uk}

\item Peter Winstanley let us know about a longstanding error in
our Latin in Chapter 3.

%He can be reached at:{\tt Peter.Winstanley@scotland.gsi.gov.uk} 

\item Chris Wrobel made corrections to the code in the chapter on
file I/O and exceptions. 

%He can be reached at: {\tt ferz980@yahoo.com}

\item Moshe Zadka has made invaluable contributions to this project.
In addition to writing the first draft of the chapter on Dictionaries, he
provided continual guidance in the early stages of the book.

%He can be reached at: {\tt moshez@math.huji.ac.il}

\item Christoph Zwerschke sent several corrections and
pedagogic suggestions, and explained the difference between {\em gleich}
and {\em selbe}.

\item James Mayer sent us a whole slew of spelling and
typographical errors, including two in the contributor list.

% james.mayer@acm.org

\item Hayden McAfee caught a potentially confusing inconsistency
between two examples.
%hayden.mcafee@mindspring.com

\item Angel Arnal is part of an international team of translators
working on the Spanish version of the text.  He has also found several
errors in the English version.

\item Tauhidul Hoque and Lex Berezhny created the illustrations
in Chapter 1 and improved many of the other illustrations.

\item Dr. Michele Alzetta caught an error in Chapter 8 and sent
some interesting pedagogic comments and suggestions about Fibonacci
and Old Maid.
%mikalzet@libero.it

\item Andy Mitchell caught a typo in Chapter 1 and a broken example
in Chapter 2.
%phantom917@hotmail.com

\item Kalin Harvey suggested a clarification in Chapter 7 and
caught some typos.
%kalin@metamuscle.net

\item Christopher P. Smith caught several typos and is helping us
prepare to update the book for Python 2.2.
%csmith@blakeschool.org

\item David Hutchins caught a typo in the Foreword.
%jsdah2@uas.alaska.edu

\item Gregor Lingl is teaching Python at a high school in Vienna,
Austria.  He is working on a German translation of the book,
and he caught a couple of bad errors in Chapter 5.
%glingl@aon.at

%Sean McShane sent us a very nice note
%sean.mcshane@sheridanc.on.ca

\item Julie Peters caught a typo in the Preface.
%jkpeters@dmacc.cc.ia.us

\item Florin Oprina sent in an improvement in {\tt makeTime},
a correction in {\tt printTime}, and a nice typo.
%oprina@student.uit.no 

\item D.~J.~Webre suggested a clarification in Chapter 3.
%d_webre@yahoo.com

% \item 
% jkane@broadlink.com

\item Ken found a fistful of errors in Chapters 8, 9 and 11.
%ken@codeweavers.com

\item Ivo Wever caught a typo in Chapter 5 and suggested a clarification
in Chapter 3.
% I.J.W.Wever@student.tnw.tudelft.nl

% rbeumer@knijnenberg.nl

\item Curtis Yanko suggested a clarification in Chapter 2.
% YankoC@gspinc.com

\item Ben Logan sent in a number of typos and problems with translating
the book into HTML.
%ben@wblogan.net

%\item XXX suggested a clarification in Chapter 7, but prefers not
% to be included here.
%ejykfy@comcast.net

%\item Florian Thiel caught an inconsistency in Chapter 2.
%noroute@web.de

\item Jason Armstrong saw the missing word in Chapter 2.
%jarmstrong@technicacorp.com

\item Louis Cordier noticed a spot in Chapter 16 where the code
didn't match the text.
% lcordier@dsp.sun.ac.za

\item Brian Cain suggested several clarifications in Chapters 2 and 3.
% Brian.Cain@motorola.com

\item Rob Black sent in a passel of corrections, including some
changes for Python 2.2.
% Rob.Black@static2358.com

\item Jean-Philippe Rey at Ecole Centrale
Paris sent a number of patches, including some updates for Python 2.2
and other thoughtful improvements.
%<jean-philippe.rey@ecp.fr>

\item Jason Mader at George Washington University made a number
of useful suggestions and corrections.
%Jason Mader <jason@ncac.gwu.edu>

\item Jan Gundtofte-Bruun reminded us that ``a error'' is an error.
% Jan Gundtofte-Bruun <jan@g-b.dk>

\item Abel David and Alexis Dinno reminded us that the plural of
``matrix'' is ``matrices'', not ``matrixes''.  This error was in the
book for years, but two readers with the same initials reported it on
the same day.  Weird.
% Abel David <abel.david@gmail.com>, lexy-lou@doyenne.com

\item Charles Thayer encouraged us to get rid of the semi-colons
we had put at the ends of some statements and to clean up our
use of ``argument'' and ``parameter''.
% Charles Thayer <catintp@yahoo.com>

\item Roger Sperberg pointed out a twisted piece of logic in Chapter 3.
%<rsperberg@gmail.com>

\item Sam Bull pointed out a confusing paragraph in Chapter 2.
%Sam Bull <dreamsorcerer@gmail.com>

\item Andrew Cheung pointed out two instances of ``use before def.''
%cheunga@u.washington.edu

%Steven Johnson <swj_ms@yahoo.com>

\item C. Corey Capel spotted the missing word in the Third Theorem
of Debugging and a typo in Chapter 4.
%"C. Corey Capel" <corey_capel@yahoo.com>

\item Alessandra helped clear up some Turtle confusion.

% Sandra Amedick <almut@risclog.de>

\item Wim Champagne found a brain-o in a dictionary example.

% <Wim.Champagne@telindus.com>

\item Douglas Wright pointed out a problem with floor division in
{\tt arc}.

%<douglas277@gmail.com>

\item Jared Spindor found some jetsom at the end of a sentence.

%<jspindor@ruckerperformance.com>

\item Lin Peiheng sent a number of very helpful suggestions.

%linpeiheng@163.com

\item Ray Hagtvedt sent in two errors and a not-quite-error.

%<hagtvedt@yahoo.com>

\item Torsten H\"{u}bsch pointed out an inconsistency in Swampy.

%<huebsch@gmail.com>

\item Inga Petuhhov corrected an example in Chapter 14.

%<inga@tlu.ee>

\item Arne Babenhauserheide sent several helpful corrections.

%<arne_bab@web.de>

\item Mark E. Casida is is good at spotting repeated words.

% <mcasida@ujf-grenoble.fr> 

\item Scott Tyler filled in a that was missing.  And then sent in
a heap of corrections.

%<Scott.Tyler@ngc.com>

\item Gordon Shephard sent in several corrections, all in separate
emails.

%<gordon@shephard.org>

\item Andrew Turner {\tt spot}ted an error in Chapter 8.

% <andrew.turner@goodrich.com>

\item Adam Hobart fixed a problem with floor division in {\tt arc}.

% <ahobart@gmail.com>

\item Daryl Hammond and Sarah Zimmerman pointed out that served
up {\tt math.pi} too early.

% DHammond@aol.com

\item George Sass found a bug in a Debugging section.

% George Sass <George.Sass@students.olin.edu>

\item Brian Bingham suggested Exercise~\ref{exrotatepairs}.

% ENDCONTRIB

\end{itemize}

\normalsize

\clearemptydoublepage

% TABLE OF CONTENTS
\begin{latexonly}

\tableofcontents

\clearemptydoublepage

\end{latexonly}

% START THE BOOK
\mainmatter

\chapter{The way of the program}

The goal of this book is to teach you to think like a
computer scientist. This way of thinking combines some of the best features
of mathematics, engineering, and natural science.  Like mathematicians,
computer scientists use formal languages to denote ideas (specifically
computations).  Like engineers, they design things, assembling components
into systems and evaluating tradeoffs among alternatives.  Like scientists,
they observe the behavior of complex systems, form hypotheses, and test
predictions.

The single most important skill for a computer scientist is {\bf
problem solving}.  Problem solving means the ability to formulate
problems, think creatively about solutions, and express a solution clearly
and accurately.  As it turns out, the process of learning to program is an
excellent opportunity to practice problem-solving skills.  That's why
this chapter is called, ``The way of the program.''

On one level, you will be learning to program, a useful
skill by itself.  On another level, you will use programming as a means to
an end.  As we go along, that end will become clearer.

\section{The Python programming language}
\index{programming language}
\index{language!programming}

The programming language you will be learning is Python. Python is
an example of a {\bf high-level language}; other high-level languages
you might have heard of are C, C++, Perl, and Java.

As you might infer from the name ``high-level language,'' there are
also {\bf low-level languages}, sometimes referred to as ``machine
languages'' or ``assembly languages.''  Loosely speaking, computers
can only execute programs written in low-level languages.  So
programs written in a high-level language have to be processed before
they can run.  This extra processing takes some time, which is a small
disadvantage of high-level languages.

\index{portable}
\index{high-level language}
\index{low-level language}
\index{language!high-level}
\index{language!low-level}

But the advantages are enormous.  First, it is much easier to program
in a high-level language. Programs written in a high-level language
take less time to write, they are shorter and easier to read, and they
are more likely to be correct.  Second, high-level languages are {\bf
portable}, meaning that they can run on different kinds of computers
with few or no modifications.  Low-level programs can run on only one
kind of computer and have to be rewritten to run on another.

Due to these advantages, almost all programs are written in high-level
languages.  Low-level languages are used only for a few specialized
applications.

\index{compile}
\index{interpret}

Two kinds of programs process high-level languages
into low-level languages: {\bf interpreters} and {\bf compilers}.
An interpreter reads a high-level program and executes it, meaning that it
does what the program says.  It processes the program a little at a time,
alternately reading lines and performing computations.

\beforefig
\centerline{\includegraphics[height=0.77in]{figs/interpret.eps}}
\afterfig

A compiler reads the program and translates it completely before the
program starts running.  In this case, the high-level program is
called the {\bf source code}, and the translated program is called the
{\bf object code} or the {\bf executable}.  Once a program is
compiled, you can execute it repeatedly without further translation.

\beforefig
\centerline{\includegraphics[height=0.77in]{figs/compile.eps}}
\afterfig

Python is considered an interpreted language because Python
programs are executed by an interpreter.  There are two ways to
use the interpreter: interactive mode and script mode. In
interactive mode, you type Python programs and the interpreter
prints the result:

\beforeverb
\begin{verbatim}
Python 2.4.1 (#1, Apr 29 2005, 00:28:56)
Type "help", "copyright", "credits" or "license" for more information.
>>> print 1 + 1
2
\end{verbatim}
\afterverb
%
The first two lines in this example are displayed by the interpreter
when it starts up.  The third line starts with {\tt >>>}, which is the
{\bf prompt} the interpreter uses to indicate that it is ready.  If
you type {\tt print 1 + 1}, the interpreter replies {\tt 2}.

Alternatively, you can store code in a file and use the
interpreter to execute the contents of the file.  Such a file is
called a {\bf script}.  For example, you could use a text editor to
create a file named {\tt dinsdale.py} with the following contents:

\beforeverb
\begin{verbatim}
print 1 + 1
\end{verbatim}
\afterverb
%
By convention, Python scripts have names that end with {\tt .py}.

To execute the script, you have to tell the interpreter the name of
the file.  In a UNIX command window, you would type {\tt python
dinsdale.py}.  In other development environments, the details of
executing scripts are different.

Working in interactive mode is convenient for testing small pieces of
code because you can type and execute them immediately.  But for
anything more than a few lines, you should save your code
as a script so you can modify and execute it in the future.


\section{What is a program?}

A {\bf program} is a sequence of instructions that specifies how to
perform a computation.  The computation might be something
mathematical, such as solving a system of equations or finding the
roots of a polynomial, but it can also be a symbolic computation, such
as searching and replacing text in a document or (strangely enough)
compiling a program.

The details look different in
different languages, but a few basic instructions
appear in just about every language:

\begin{description}

\item[input:] Get data from the keyboard, a file, or some
other device.

\item[output:] Display data on the screen or send data to a
file or other device.

\item[math:] Perform basic mathematical operations like addition and
multiplication.

\item[conditional execution:] Check for certain conditions and
execute the appropriate sequence of statements.

\item[repetition:] Perform some action repeatedly, usually with
some variation.

\end{description}

Believe it or not, that's pretty much all there is to it.  Every
program you've ever used, no matter how complicated, is made up of
instructions that look pretty much like these.  So you can think of
programming as the process of breaking a large, complex task
into smaller and smaller subtasks until the subtasks are
simple enough to be performed with one of these basic instructions.

That may be a little vague, but we will come back to this topic
when we talk about {\bf algorithms}.

\section{What is debugging?}
\index{debugging}
\index{bug}

Programming is error-prone.  For whimsical reasons, programming errors
are called {\bf bugs} and the process of tracking them down is called
{\bf debugging}.

Three kinds of errors can occur in a program: syntax errors, runtime 
errors, and semantic errors. It is useful
to distinguish between them in order to track them down more quickly.

\subsection{Syntax errors}
\index{syntax error}
\index{error!syntax}

Python can only execute a program if the syntax is
correct; otherwise, the interpreter displays an error message.
{\bf Syntax} refers to the structure of a program and the rules about
that structure. \index{syntax} For example, in English, a sentence must
begin with a capital letter and end with a period.  this sentence contains
a {\bf syntax error}.  So does this one

For most readers, a few syntax errors are not a significant problem,
which is why we can read the poetry of e. e. cummings without spewing error
messages.  Python is not so forgiving.  If there is a single syntax error
anywhere in your program, Python will print an error message and quit,
and you will not be able to run your program. During the first few weeks
of your programming career, you will probably spend a lot of time tracking
down syntax errors.  As you gain experience, you will make fewer
errors and find them faster.

\subsection{Runtime errors}
\label{runtime}
\index{runtime error}
\index{error!runtime}
\index{exception}
\index{safe language}
\index{language!safe}

The second type of error is a runtime error, so called because the
error does not appear until after the program has started running.
These errors are also called {\bf exceptions} because they usually
indicate that something exceptional (and bad) has happened.

Runtime errors are rare in the simple programs you will see in the
first few chapters, so it might be a while before you encounter one.


\subsection{Semantic errors}
\index{semantics}
\index{semantic error}
\index{error!semantic}

The third type of error is the {\bf semantic error}.  If there is a
semantic error in your program, it will run successfully, in the sense
that the computer will not generate any error messages, but it will
not do the right thing.  It will do something else.  Specifically, it
will do what you told it to do.

The problem is that the program you wrote is not the program you
wanted to write.  The meaning of the program (its semantics) is wrong.
Identifying semantic errors can be tricky because it requires you to work
backward by looking at the output of the program and trying to figure
out what it is doing.

\subsection{Experimental debugging}

One of the most important skills you will acquire is debugging.
Although it can be frustrating, debugging is one of the most
intellectually rich, challenging, and interesting parts of
programming.

In some ways, debugging is like detective work.  You are confronted
with clues, and you have to infer the processes and events that led
to the results you see.

Debugging is also like an experimental science.  Once you have an idea
about what is going wrong, you modify your program and try again.  If
your hypothesis was correct, then you can predict the result of the
modification, and you take a step closer to a working program.  If
your hypothesis was wrong, you have to come up with a new one.  As
Sherlock Holmes pointed out, ``When you have eliminated the
impossible, whatever remains, however improbable, must be the truth.''
(A. Conan Doyle, {\em The Sign of Four})

\index{Holmes, Sherlock}
\index{Doyle, Arthur Conan}

For some people, programming and debugging are the same thing.  That
is, programming is the process of gradually debugging a program until
it does what you want.  The idea is that you should start with a
program that does {\em something} and make small modifications,
debugging them as you go, so that you always have a working program.

For example, Linux is an operating system that contains thousands of
lines of code, but it started out as a simple program Linus Torvalds
used to explore the Intel 80386 chip.  According to Larry Greenfield,
``One of Linus's earlier projects was a program that would switch
between printing AAAA and BBBB.  This later evolved to Linux.''
({\em The Linux Users' Guide} Beta Version 1)

\index{Linux}

Later chapters will make more suggestions about debugging and other
programming practices.

\section{Formal and natural languages}
\index{formal language}
\index{natural language}
\index{language!formal}
\index{language!natural}

{\bf Natural languages} are the languages people speak,
such as English, Spanish, and French.  They were not designed
by people (although people try to impose some order on them);
they evolved naturally.

{\bf Formal languages} are languages that are designed by people for
specific applications.  For example, the notation that mathematicians
use is a formal language that is particularly good at denoting
relationships among numbers and symbols.  Chemists use a formal
language to represent the chemical structure of molecules.  And
most importantly:

\begin{quote}
{\bf Programming languages are formal languages that have been
designed to express computations.}
\end{quote}

Formal languages tend to have strict rules about syntax.  For example,
$3 + 3 = 6$ is a syntactically correct mathematical statement, but 
$3 + = 3 \mbox{\$} 6$ is not.  $H_2O$ is a syntactically correct
chemical formula, but $_2Zz$ is not.

Syntax rules come in two flavors, pertaining to {\bf tokens} and
structure.  Tokens are the basic elements of the language, such as
words, numbers, and chemical elements.  One of the problems with $3 +
= 3 \mbox{\$} 6$ is that $\mbox{\$}$ is not a legal token in mathematics
(at least as far as I know).  Similarly, $_2Zz$ is not legal because
there is no element with the abbreviation $Zz$.

The second type of syntax error pertains to the structure of a
statement; that is, the way the tokens are arranged.  The statement $3
+ = 3 \mbox{\$} 6$ is illegal because even though $+$ and $=$ are
legal tokens, you can't have one right after the other.  Similarly,
in a chemical formula the subscript comes after the element name, not
before.

\begin{ex}
Write a well-structured English
sentence with invalid tokens in it.  Then write another sentence
with all valid tokens but with invalid structure.
\end{ex}

When you read a sentence in English or a statement in a formal
language, you have to figure out what the structure of the sentence is
(although in a natural language you do this subconsciously).  This
process is called {\bf parsing}.

\index{parse}

For example, when you hear the sentence, ``The penny dropped,'' you
understand that ``the penny'' is the subject and ``dropped'' is the
predicate.  Once you have parsed a sentence, you can figure out what it
means, or the semantics of the sentence.  Assuming that you know
what a penny is and what it means to drop, you will understand the
general implication of this sentence.

Although formal and natural languages have many features in
common---tokens, structure, syntax, and semantics---there are many
differences:

\index{ambiguity}
\index{redundancy}
\index{literalness}

\begin{description}

\item[ambiguity:] Natural languages are full of ambiguity, which
people deal with by using contextual clues and other information.
Formal languages are designed to be nearly or completely unambiguous,
which means that any statement has exactly one meaning,
regardless of context.

\item[redundancy:] In order to make up for ambiguity and reduce
misunderstandings, natural languages employ lots of
redundancy.  As a result, they are often verbose.  Formal languages
are less redundant and more concise.

\item[literalness:] Natural languages are full of idiom and metaphor.
If I say, ``The penny dropped,'' there is probably no penny and
nothing dropping\footnote{This idiom means that someone realized something
after a period of confusion.}.  Formal languages
mean exactly what they say.

\end{description}

People who grow up speaking a natural language---everyone---often have a
hard time adjusting to formal languages.  In some ways, the difference
between formal and natural language is like the difference between
poetry and prose, but more so:

\index{poetry}
\index{prose}

\begin{description}

\item[Poetry:] Words are used for their sounds as well as for
their meaning, and the whole poem together creates an effect or
emotional response.  Ambiguity is not only common but often
deliberate.

\item[Prose:] The literal meaning of words is more important,
and the structure contributes more meaning.  Prose is more amenable to
analysis than poetry but still often ambiguous.

\item[Programs:] The meaning of a computer program is unambiguous
and literal, and can be understood entirely by analysis of the
tokens and structure.

\end{description}

Here are some suggestions for reading programs (and other formal
languages).  First, remember that formal languages are much more dense
than natural languages, so it takes longer to read them.  Also, the
structure is very important, so it is usually not a good idea to read
from top to bottom, left to right.  Instead, learn to parse the
program in your head, identifying the tokens and interpreting the
structure.  Finally, the details matter.  Small errors in
spelling and punctuation, which you can get away
with in natural languages, can make a big difference in a formal
language.

\section{The first program}
\label{hello}
\index{hello world}

Traditionally, the first program you write in a new language
is called ``Hello, World!'' because all it does is display the
words, ``Hello, World!''  In Python, it looks like this:


\beforeverb
\begin{verbatim}
print 'Hello, World!'
\end{verbatim}
\afterverb
%
This is an example of a {\bf print statement}, which doesn't
actually print anything on paper.  It displays a value on the
screen.  In this case, the result is the words


\beforeverb
\begin{verbatim}
Hello, World!
\end{verbatim}
\afterverb
%
The quotation marks in the program mark the beginning and end
of the text to be displayed; they don't appear in the result.

\index{print statement}
\index{statement!print}

Some people judge the quality of a programming language by the
simplicity of the ``Hello, World!'' program.  By this standard, Python
does about as well as possible.


\section{Debugging}

It is a good idea to read this book in front of a computer so you can
try out the examples as you go.  You can run most of the examples in
interactive mode, but if you put the code into a script, it is easier
to try out variations.

Whenever you are experimenting with a new feature, you should try
to make mistakes.  For example, in the ``Hello, world!'' program,
what happens if you leave out one of the quotation marks?  What
if you leave out both?  What if you spell {\tt print} wrong?

This kind of experiment helps you remember what you read; it also helps
with debugging, because you get to know what the error messages mean.
And that brings us to the First Theorem of Debugging:

\begin{quote}
It is better to make mistakes now and on purpose than later
and accidentally.
\end{quote}

Learning to debug can be frustrating, but it is one of the most
important parts of thinking like a computer scientist.  At the
end of each chapter there is a debugging section, like this one,
with my thoughts (and theorems) of debugging.  I hope they help!


\section{Glossary}

\begin{description}

\item[problem solving:]  The process of formulating a problem, finding
a solution, and expressing the solution.

\item[high-level language:]  A programming language like Python that
is designed to be easy for humans to read and write.

\item[low-level language:]  A programming language that is designed
to be easy for a computer to execute; also called ``machine language'' or
``assembly language.''

\item[portability:]  A property of a program that can run on more
than one kind of computer.

\item[interpret:]  To execute a program in a high-level language
by translating it one line at a time.

\item[compile:]  To translate a program written in a high-level language
into a low-level language all at once, in preparation for later
execution.

\item[source code:]  A program in a high-level language before
being compiled.

\item[object code:]  The output of the compiler after it translates
the program.

\item[executable:]  Another name for object code that is ready
to be executed.

\item[prompt:] Characters displayed by the interpreter to indicate
that it is ready to take input from the user.

\item[script:] A program stored in a file (usually one that will be
interpreted).

\item[program:] A set of instructions that specifies a computation.

\item[algorithm:]  A general process for solving a category of
problems.

\item[bug:]  An error in a program.

\item[debugging:]  The process of finding and removing any of the
three kinds of programming errors.

\item[syntax:]  The structure of a program.

\item[syntax error:]  An error in a program that makes it impossible
to parse (and therefore impossible to interpret).

\item[exception:]  An error that is detected while the program is running.

\item[semantics:]  The meaning of a program.

\item[semantic error:]   An error in a program that makes it do something
other than what the programmer intended.

\item[natural language:]  Any one of the languages that people speak that
evolved naturally.

\item[formal language:]  Any one of the languages that people have designed
for specific purposes, such as representing mathematical ideas or
computer programs; all programming languages are formal languages.

\item[token:]  One of the basic elements of the syntactic structure of
a program, analogous to a word in a natural language.

\item[parse:]  To examine a program and analyze the syntactic structure.

\item[print statement:]  An instruction that causes the Python
interpreter to display a value on the screen.

\index{program}
\index{problem-solving}
\index{high-level language}
\index{low-level language}
\index{portability}
\index{interpret}
\index{compile}
\index{source code}
\index{object code}
\index{executable}
\index{algorithm}
\index{bug}
\index{debugging}
\index{syntax}
\index{semantics}
\index{syntax error}
\index{runtime error}
\index{exception}
\index{semantic error}
\index{formal language}
\index{natural language}
\index{parse}
\index{token}
\index{script}
\index{print statement}
\index{statement!print}

\end{description}


\section{Exercises}

\begin{ex}
Use a web browser to go to \url{http://python.org}.
This page contains a lot of information about Python, pointers
to Python-related pages, and it gives you the ability to search
the Python documentation.

For example, if you enter {\tt print} in the search window, the
first link that appears is the documentation of the {\tt print}
statement.  At this point, not all of it will make sense to you,
but it is good to know where it is!
\end{ex}

\begin{ex}
Start the Python interpreter and type {\tt help()} to
start the online help utility.  Alternatively, you can type
{\tt help('print')} to get information about a particular
topic, in this case the {\tt print} statement.  If this example
doesn't work, you may need to install additional Python documentation
or set an environment variable; unfortunately, the details depend
on your operating system and version of Python.
\end{ex}




\chapter{Variables, expressions and statements}

\section{Values and types}
\index{value}
\index{type}
\index{string}

A {\bf value} is one of the basic things a program works with,
like a letter or a
number.  The values we have seen so far
are {\tt 1}, {\tt 2}, and
{\tt 'Hello, World!'}.

These values belong to different {\bf types}:
{\tt 2} is an integer, and {\tt 'Hello, World!'} is a {\bf string},
so-called because it contains a ``string'' of letters.
You (and the interpreter) can identify
strings because they are enclosed in quotation marks.

The print statement also works for integers.

\beforeverb
\begin{verbatim}
>>> print 4
4
\end{verbatim}
\afterverb
%
If you are not sure what type a value has,
the interpreter can tell you.

\beforeverb
\begin{verbatim}
>>> type('Hello, World!')
<type 'str'>
>>> type(17)
<type 'int'>
\end{verbatim}
\afterverb
%
Not surprisingly, strings belong to the type {\tt str} and
integers belong to the type {\tt int}.  Less obviously, numbers
with a decimal point belong to a type called {\tt float},
because these numbers are represented in a
format called {\bf floating-point}.

\index{type}
\index{string}
\index{type!str}
\index{int}
\index{type!int}
\index{float}
\index{type!float}

\beforeverb
\begin{verbatim}
>>> type(3.2)
<type 'float'>
\end{verbatim}
\afterverb
%
What about values like {\tt '17'} and {\tt '3.2'}?
They look like numbers, but they are in quotation marks like
strings.

\beforeverb
\begin{verbatim}
>>> type('17')
<type 'str'>
>>> type('3.2')
<type 'str'>
\end{verbatim}
\afterverb
%
They're strings.

When you type a large integer, you might be tempted to use commas
between groups of three digits, as in {\tt 1,000,000}.  This is not a
legal integer in Python, but it is legal:

\beforeverb
\begin{verbatim}
>>> print 1,000,000
1 0 0
\end{verbatim}
\afterverb
%
Well, that's not what we expected at all!
Python interprets {\tt
1,000,000} as a comma-separated sequence of integers which it
prints with spaces between.

This is the first example we have seen of a
semantic error: the code runs without producing an error message, but
it doesn't do the ``right'' thing.


\section{Variables}
\index{variable}
\index{assignment}
\index{statement!assignment}

One of the most powerful features of a programming language is the
ability to manipulate {\bf variables}.  A variable is a name that
refers to a value.

The {\bf assignment statement} creates new variables and gives
them values:

\beforeverb
\begin{verbatim}
>>> message = 'And now for something completely different'
>>> n = 17
>>> pi = 3.1415926535897931
\end{verbatim}
\afterverb
%
This example makes three assignments.  The first assigns a string
to a new variable named {\tt message};
the second gives the integer {\tt 17} to {\tt n}; the third
assigns the (approximate) value of $\pi$ to {\tt pi}.

\index{state diagram}
\index{diagram!state}

A common way to represent variables on paper is to write the name with
an arrow pointing to the variable's value.  This kind of figure is
called a {\bf state diagram} because it shows what state each of the
variables is in (think of it as the variable's state of mind).
This diagram shows the result of the assignment statements:

\beforefig
\centerline{\includegraphics{figs/state2.eps}}
\afterfig

The print statement displays the value of a variable:

\beforeverb
\begin{verbatim}
>>> print n
17
>>> print pi
3.14159265359
\end{verbatim}
\afterverb
%
The type of a variable is the type of the value it refers to.

\beforeverb
\begin{verbatim}
>>> type(message)
<type 'str'>
>>> type(n)
<type 'int'>
>>> type(pi)
<type 'float'>
\end{verbatim}
\afterverb
%


\section{Variable names and keywords}
\index{keyword}

Programmers generally choose names for their variables that
are meaningful---they document what the variable is used for.

Variable names can be arbitrarily long.  They can contain
both letters and numbers, but they have to begin with a letter.
Although it is legal to use uppercase letters, by convention
we don't.  If you do, remember that case matters.  {\tt Bruce}
and {\tt bruce} are different variables.

The underscore character ({\tt \_}) can appear in a name.
It is often used in names with multiple words, such as
{\tt my\_name} or {\tt airspeed\_of\_unladen\_swallow}.

\index{underscore character}

If you give a variable an illegal name, you get a syntax error:


\beforeverb
\begin{verbatim}
>>> 76trombones = 'big parade'
SyntaxError: invalid syntax
>>> more@ = 1000000
SyntaxError: invalid syntax
>>> class = 'Advanced Theoretical Herpetology'
SyntaxError: invalid syntax
\end{verbatim}
\afterverb
%
{\tt 76trombones} is illegal because it does not begin with a letter.
{\tt more\@} is illegal because it contains an illegal character, {\tt
@}.  But what's wrong with {\tt class}?

It turns out that {\tt class} is one of Python's {\bf keywords}.  The
interpreter uses keywords to recognize the structure of the program,
and they cannot be used as variable names.

\index{keyword}

Python has 31 keywords:

\beforeverb
\begin{verbatim}
and       del       from      not       while    
as        elif      global    or        with     
assert    else      if        pass      yield    
break     except    import    print              
class     exec      in        raise              
continue  finally   is        return             
def       for       lambda    try
\end{verbatim}
\afterverb
%
You might want to keep this list handy.  If the interpreter complains
about one of your variable names and you don't know why, see if it
is on this list.


\section{Statements}

A statement is an instruction that the Python interpreter can
execute.  We have seen two kinds of statements: print
and assignment.

When you type a statement on the command line, Python
executes it and displays the result, if there is one.

A script usually contains a sequence of statements.  If there
is more than one statement, the results appear one at a time
as the statements execute.

For example, the script

\beforeverb
\begin{verbatim}
print 1
x = 2
print x
\end{verbatim}
\afterverb
%
produces the output

\beforeverb
\begin{verbatim}
1
2
\end{verbatim}
\afterverb
%
The assignment statement produces no output itself.



\section{Operators and operands}
\index{operator}
\index{operand}
\index{expression}

{\bf Operators} are special symbols that represent computations like
addition and multiplication.  The values the operator is applied to
are called {\bf operands}.

The following examples demonstrate the arithmetic operators:

\beforeverb
\begin{verbatim}
20+32   hour-1   hour*60+minute   minute/60   5**2   (5+9)*(15-7)
\end{verbatim}
\afterverb
%
The symbols {\tt +}, {\tt -}, and {\tt /}, and the use of parenthesis for
grouping, mean in Python what they mean in mathematics.  The asterisk
({\tt *}) is the symbol for multiplication, and {\tt **} is the symbol
for exponentiation.

When a variable name appears in the place of an operand, it
is replaced with its value before the operation is
performed.

Addition, subtraction, multiplication, and exponentiation all do what
you expect, but you might be surprised by division.  The following
operation has an unexpected result:

\beforeverb
\begin{verbatim}
>>> minute = 59
>>> minute/60
0
\end{verbatim}
\afterverb
%
The value of {\tt minute} is 59, and in conventional arithmetic 59
divided by 60 is 0.98333, not 0.  The reason for the discrepancy is
that Python is performing {\bf floor division}\footnote{This behavior
is likely to change in Python 3.0.}.

\index{floor division}
\index{floating-point division}
\index{division!floor}
\index{division!floating-point}

When both of the operands are integers, the result is also an
integer; floor division chops off the fraction
part, so in this example it rounds down to zero.

If either of the operands is a floating-point number, Python performs
floating-point division, and the result is a {\tt float}:

\beforeverb
\begin{verbatim}
>>> minute/60.0
0.98333333333333328
\end{verbatim}
\afterverb


\section{Expressions}

An {\bf expression} is a combination of values, variables, and operators.
If you type an expression on the command line, the interpreter
{\bf evaluates} it and displays the result:

\beforeverb
\begin{verbatim}
>>> 1 + 1
2
\end{verbatim}
\afterverb
%
Although expressions can contain values, variables, and operators,
not every expression contains all of these elements.
A value all by itself is considered an expression, and so is
a variable.

\beforeverb
\begin{verbatim}
>>> 17
17
>>> x
2
\end{verbatim}
\afterverb
%
In a script, an expression all by itself is a legal statement, but it
doesn't do anything.  The following script produces no output at all: 

\beforeverb
\begin{verbatim}
17
3.2
'Hello, World!'
1 + 1
\end{verbatim}
\afterverb
%
If you want the script to display the values of these expressions,
you have to use {\tt print} statements.



\section{Order of operations}
\index{order of operations}
\index{rules of precedence}

When more than one operator appears in an expression, the order of
evaluation depends on the {\bf rules of precedence}.  For
mathematical operators, Python follows the mathematical rules.
The acronym {\bf PEMDAS} is a useful way to
remember them:

\begin{itemize}

\item {\bf P}arentheses have the highest precedence and can be used 
to force an expression to evaluate in the order you want. Since
expressions in parentheses are evaluated first, {\tt 2 * (3-1)} is 4,
and {\tt (1+1)**(5-2)} is 8. You can also use parentheses to make an
expression easier to read, as in {\tt (minute * 100) / 60}, even
though it doesn't change the result.

\item {\bf E}xponentiation has the next highest precedence, so
{\tt 2**1+1} is 3 and not 4, and {\tt 3*1**3} is 3 and not 27.

\item {\bf M}ultiplication and {\bf D}ivision have the same precedence,
which is higher than {\bf A}ddition and {\bf S}ubtraction, which also
have the same precedence.  So {\tt 2*3-1} is 5, not 4, and
{\tt 6+4/2} is 8, not 5.

\item Operators with the same precedence are evaluated from left to 
right.  So in the expression {\tt degrees / 2 * pi}, the division
happens first and the result is multiplied by {\tt pi}.  If you meant
to divide by $2 \pi$, you should have used parentheses.

\end{itemize}


\section{String operations}
\index{string operation}

In general, you cannot perform mathematical operations on strings, even
if the strings look like numbers, so the following are illegal:

\beforeverb
\begin{verbatim}
'2'-'1'    'eggs'/'easy'    'third'*'a charm'
\end{verbatim}
\afterverb
%
The {\tt +} operator does work with strings, but it
might not do exactly what you expect: it performs
{\bf concatenation}, which means joining the strings by
linking them end-to-end.  For example:

\index{concatenation}

\beforeverb
\begin{verbatim}
first = 'throat'
second = 'warbler'
print first + second
\end{verbatim}
\afterverb
%
The output of this program is {\tt throatwarbler}.

The {\tt *} operator also works on strings; it performs repetition.
For example, {\tt 'Spam'*3} is {\tt 'SpamSpamSpam'}.  If one of the operands
is a string, the other has to be an integer.

On one hand, this use of {\tt +} and {\tt *} makes sense by
analogy with addition and multiplication.  Just as {\tt 4*3} is
equivalent to {\tt 4+4+4}, we expect {\tt 'Spam'*3} to be the same as
{\tt 'Spam'+'Spam'+'Spam'}, and it is.  On the other hand, there is a
significant way in which string concatenation and repetition are
different from integer addition and multiplication.
Can you think of a property that addition and multiplication have
that string concatenation and repetition do not?


\section{Comments}
\index{comment}

As programs get bigger and more complicated, they get more difficult
to read.  Formal languages are dense, and it is often difficult to
look at a piece of code and figure out what it is doing, or why.

For this reason, it is a good idea to add notes to your programs to explain
in natural language what the program is doing.  These notes are called
{\bf comments}, and they are marked with the {\tt \#} symbol:

\beforeverb
\begin{verbatim}
# compute the percentage of the hour that has elapsed
percentage = (minute * 100) / 60
\end{verbatim}
\afterverb
%
In this case, the comment appears on a line by itself.  You can also put
comments at the end of a line:

\beforeverb
\begin{verbatim}
percentage = (minute * 100) / 60     # percentage of an hour
\end{verbatim}
\afterverb
%
Everything from the {\tt \#} to the end of the line is ignored---it
has no effect on the program.

Comments are most useful when they document non-obvious features of
the code.  It is reasonable to assume that the reader can figure out
{\em what} the code does; it is much more useful to explain {\em why}.

This comment is redundant with the code and useless:

\beforeverb
\begin{verbatim}
v = 5     # assign 5 to v
\end{verbatim}
\afterverb
%

This comment contains useful information that is not in the code:

\beforeverb
\begin{verbatim}
v = 5     # velocity in meters/second. 
\end{verbatim}
\afterverb
%
Good variable names can reduce the need for comments, but
long names can make complex expressions hard to read, so there is
a tradeoff.

\section{Debugging}

At this point the syntax error you are most likely to make is
an illegal variable name, like {\tt class} and {\tt yield} (which
are keywords) or {\tt odd\verb+~+job} and {\tt US\$} which contain
illegal characters.

If you put a space in a variable name, Python thinks it is two
operands without an operator:

\beforeverb
\begin{verbatim}
>>> bad name = 5
SyntaxError: invalid syntax
\end{verbatim}
\afterverb
%
For syntax errors, the error messages don't help much.
The most common messages are {\tt SyntaxError: invalid syntax} and
{\tt SyntaxError: invalid token}, neither of which is very informative.

The run-time error you are most likely to make is a ``use before
def;'' that is, trying to use a variable before you have assigned
a value.  This can happen if you spell a variable name wrong:

\beforeverb
\begin{verbatim}
>>> principal = 327.68
>>> interest = principle * rate
NameError: name 'principle' is not defined
\end{verbatim}
\afterverb
%
Variables names are case sensitive, so {\tt Bob} is not the
same as {\tt bob}.

At this point the most likely cause of a semantic error is
the order of operations.  For example, to evaluate $\frac{1}{2 a}$,
you might be tempted to write

\beforeverb
\begin{verbatim}
>>> 1.0 / 2.0 * a
\end{verbatim}
\afterverb
%
But the division happens first, so you would get $a / 2$, which
is not the same thing!  Unfortunately, there is no way for Python
to know what you intended to write, so in this case you don't
get an error message; you just get the wrong answer.

And that brings us to the Second Theorem of Debugging:

\begin{quote}
The only thing worse than getting an error message is not
getting an error message.
\end{quote}



\section{Glossary}

\begin{description}

\item[value:]  One of the basic units of data, like a number or string, 
that a program manipulates.
\index{value}

\item[type:] A category of values.  The types we have seen so far
are integers (type {\tt int}), floating-point numbers (type {\tt
float}), and strings (type {\tt str}).
\index{type}

\item[integer:] A type that represents whole numbers.
\index{integer}

\item[floating-point:] A type that represents numbers with fractional
parts.
\index{floating-point}

\item[string:] A type that represents sequences of characters.
\index{string}

\item[variable:]  A name that refers to a value.
\index{variable}

\item[statement:]  A section of code that represents a command or action.  So
far, the statements we have seen are assignments and print statements.
\index{statement}

\item[assignment:]  A statement that assigns a value to a variable.
\index{assignment}

\item[state diagram:]  A graphical representation of a set of variables and the
values they refer to.
\index{state diagram}

\item[keyword:]  A reserved word that is used by the compiler to parse a
program; you cannot use keywords like {\tt if}, {\tt  def}, and {\tt while} as
variable names.
\index{keyword}

\item[operator:]  A special symbol that represents a simple computation like
addition, multiplication, or string concatenation.
\index{operator}

\item[operand:]  One of the values on which an operator operates.
\index{operand}

\item[floor division:] The operation that divides two numbers and chops off
the fraction part.
\index{floor division}

\item[expression:]  A combination of variables, operators, and values that
represents a single result value.
\index{expression}

\item[evaluate:]  To simplify an expression by performing the operations
in order to yield a single value.

\item[rules of precedence:]  The set of rules governing the order in which
expressions involving multiple operators and operands are evaluated.
\index{rules of precedence}
\index{precedence}

\item[concatenate:]  To join two operands end-to-end.
\index{concatenation}

\item[comment:]  Information in a program that is meant for other
programmers (or anyone reading the source code) and has no effect on the
execution of the program.
\index{comment}

\end{description}


\section{Exercises}

\begin{ex}
Assume that we execute the following assignment statements:

\begin{verbatim}
width = 17
height = 12.0
delimiter = '.'
\end{verbatim}

For each of the following expressions, write the value of the
expression and the type (of the value of the expression).

\begin{enumerate}

\item {\tt width/2}

\item {\tt height/3.0}

\item {\tt width/2.0}

\item {\tt 1 + 2 * 5}

\item {\tt delimiter * 5}

\end{enumerate}
\end{ex}

\begin{ex}
Practice using the Python interpreter as a calculator: 

\begin{itemize}

%\item What is the circumference of a circle with radius 5?
%What is the area?

\item If you ran 10 kilometers in
45 minutes 30 seconds, what was your average pace in minutes
per mile?  What was your average speed in miles per hour?
(Hint: there are 1.61 kilometers in a mile).

\end{itemize}
\end{ex}


\chapter{Functions}
\label{funcchap}

\section{Function calls}
\label{functionchap}
\index{function call}
\index{call!function}

In the context of programming, a {\bf function} is a named sequence of
statements that performs a computation.  When you define a function,
you specify the name and the sequence of statements.  Later, you can
``call'' the function by name.  
We have already seen one example of a {\bf function call}:

\beforeverb
\begin{verbatim}
>>> type('32')
<type 'str'>
\end{verbatim}
\afterverb
%
The name of the function is {\tt type}.  The expression in parentheses
is called the {\bf argument} of the function.  The result, for this
function, is the type of the argument, which is a string.

It is common to say that a function ``takes'' an argument and ``returns''
a result.  The result is called the {\bf return value}.

\index{argument}
\index{return value}

When you call a function in interactive mode, the interpreter displays
the return value, but in a script a function call, all by itself, doesn't
display anything.  To see the result, you have to print it:

\beforeverb
\begin{verbatim}
print type('32')
\end{verbatim}
\afterverb
%
Or assign the return value to a variable, which you can print
(or use for some other purpose) later.

\beforeverb
\begin{verbatim}
stereo = type('32')
print stereo
\end{verbatim}
\afterverb
%


\section{Type conversion functions}
\index{conversion!type}
\index{type conversion}

% from Jeff:

% comment on whether these things are _really_ functions?

% use max as an example of a built-in?

Python provides built-in functions that convert values
from one type to another.  The {\tt int} function takes any value and
converts it to an integer if it can or complains otherwise:

\beforeverb
\begin{verbatim}
>>> int('32')
32
>>> int('Hello')
ValueError: invalid literal for int(): Hello
\end{verbatim}
\afterverb
%
{\tt int} can convert floating-point values to integers, but it
doesn't round off; it chops off the fraction part:

\beforeverb
\begin{verbatim}
>>> int(3.99999)
3
>>> int(-2.3)
-2
\end{verbatim}
\afterverb
%
{\tt float} converts integers and strings to floating-point
numbers:

\beforeverb
\begin{verbatim}
>>> float(32)
32.0
>>> float('3.14159')
3.14159
\end{verbatim}
\afterverb
%
Finally, {\tt str} converts its argument to a string:

\beforeverb
\begin{verbatim}
>>> str(32)
'32'
>>> str(3.14149)
'3.14149'
\end{verbatim}
\afterverb
%



\section{Math functions}
\index{math function}
\index{function!math}

Python has a math module that provides most of the familiar
mathematical functions.  A {\bf module} is a file that contains a
collection of related functions.

\index{module}
\index{module object}

Before we can use the module, we have to import it:

\beforeverb
\begin{verbatim}
>>> import math
\end{verbatim}
\afterverb
%
This statement creates a {\bf module object} named math.  If
you print the module object, you get some information about it:

\beforeverb
\begin{verbatim}
>>> print math
<module 'math' from '/usr/lib/python2.4/lib-dynload/mathmodule.so'>
\end{verbatim}
\afterverb
%
The module object contains the functions and variables defined in the
module.  To access one of the functions, you have to specify the name
of the module and the name of the function, separated by a dot (also
known as a period).  This format is called {\bf dot notation}.

\index{dot notation}

\beforeverb
\begin{verbatim}
>>> ratio = signal_power / noise_power
>>> decibels = 10 * math.log10(ratio)

>>> radians = 0.7
>>> height = math.sin(radians)
\end{verbatim}
\afterverb
%
The first example computes the logarithm base 10 of the
signal-to-noise ratio.  The math module also provides a
function called {\tt log} that computes logarithms base {\tt e}.

The second example finds the sine of {\tt
radians}.  The name of the variable is a hint that {\tt
sin} and the other trigonometric functions ({\tt cos}, {\tt tan},
etc.)  take arguments in radians. To convert from degrees to radians,
divide by 360 and multiply by $2 \pi$:

\beforeverb
\begin{verbatim}
>>> degrees = 45
>>> radians = degrees / 360.0 * 2 * math.pi
>>> math.sin(radians)
0.707106781187
\end{verbatim}
\afterverb
%
The expression {\tt math.pi} gets the variable {\tt pi} from the math
module.  Conveniently, the value of this variable is an approximation
of $\pi$, accurate to about 15 digits.

If you know
your trigonometry, you can check the previous result by comparing it to
the square root of two divided by two:

\beforeverb
\begin{verbatim}
>>> math.sqrt(2) / 2.0
0.707106781187
\end{verbatim}
\afterverb
%

\section{Composition}
\index{composition}

So far, we have looked at the elements of a program---variables,
expressions, and statements---in isolation, without talking about how to
combine them.

One of the most useful features of programming languages is their
ability to take small building blocks and {\bf compose} them.  For
example, the argument of a function can be any kind of expression,
including arithmetic operators:

\beforeverb
\begin{verbatim}
x = math.sin(degrees / 360.0 * 2 * math.pi)
\end{verbatim}
\afterverb
%
And even function calls:

\beforeverb
\begin{verbatim}
x = math.exp(math.log(x+1))
\end{verbatim}
\afterverb
%
Almost anywhere you can put a value, you can put an arbitrary
expression, with one exception: the left side of an assignment
statement has to be a variable name.  An expression on the left side
is a syntax error.

\beforeverb
\begin{verbatim}
>>> minutes = hours * 60                 # right
>>> hours * 60 = minutes                 # wrong!
SyntaxError: can't assign to operator
\end{verbatim}
\afterverb
%
\index{SyntaxError}
\index{exception!SyntaxError}


\section{Adding new functions}

So far, we have only been using the functions that come with Python,
but it is also possible to add new functions.
A {\bf function definition} specifies the name of a new function and
the sequence of statements that execute when the function is called.

\index{function}
\index{function definition}
\index{definition!function}

Here is an example:

\beforeverb
\begin{verbatim}
def print_lyrics():
    print "I'm a lumberjack, and I'm okay."
    print "I sleep all night and I work all day."
\end{verbatim}
\afterverb
%
{\tt def} is a keyword that indicates that this is a function
definition.  The name of the function is {\tt print\_lyrics}.  The
rules for function names are the same as for variable names: letters,
numbers and some punctuation marks are legal, but the first character
can't be a number.  You can't use a keyword as the name of a function,
and you should avoid having a variable and a function with the same
name.

The empty parentheses after the name indicate that this function
doesn't take any arguments.

The first line of the function definition is called the {\bf header};
the rest is called the {\bf body}.
The header has to end with a colon and the
body has to be indented.  By convention, the indentation is always
four spaces.  The body can contain any number of statements.

The strings in the print statements are enclosed in double
quotes.  Single quotes and double quotes do the same thing.
Most people use single quotes except in cases like this where
a single quote (which is also an apostrophe) appears in the string.

If you type a function definition in interactive mode, the interpreter
prints ellipses ({\em ...}) to let you know that the definition
isn't complete:

\beforeverb
\begin{verbatim}
>>> def print_lyrics():
...     print "I'm a lumberjack, and I'm okay."
...     print "I sleep all night and I work all day."
...
\end{verbatim}
\afterverb
%
To end the function, you have to enter an empty line (this is
not necessary in a script).

Defining a function creates a variable with the same name.

\beforeverb
\begin{verbatim}
>>> print print_lyrics
<function print_lyrics at 0xb7e99e9c>
>>> print type(print_lyrics)
<type 'function'>
\end{verbatim}
\afterverb
%
The value of {\tt print\_lyrics} is a {\bf function object}, which
has type {\tt function}.

The syntax for calling the new function is the same as
for built-in functions:

\beforeverb
\begin{verbatim}
>>> print_lyrics()
I'm a lumberjack, and I'm okay.
I sleep all night and I work all day.
\end{verbatim}
\afterverb
%
Once you have defined a function, you can use it inside another
function.  For example, to repeat the previous refrain, we could write
a function called {\tt repeat\_lyrics}:

\beforeverb
\begin{verbatim}
def repeat_lyrics():
    print_lyrics()
    print_lyrics()
\end{verbatim}
\afterverb
%
And then call {\tt repeat\_lyrics}:

\beforeverb
\begin{verbatim}
>>> repeat_lyrics()
I'm a lumberjack, and I'm okay.
I sleep all night and I work all day.
I'm a lumberjack, and I'm okay.
I sleep all night and I work all day.
\end{verbatim}
\afterverb
%
But that's not really how the song goes.


\section{Definitions and uses}

Pulling together the code fragments from the previous section, the
whole program looks like this:

\beforeverb
\begin{verbatim}
def print_lyrics():
    print "I'm a lumberjack, and I'm okay."
    print "I sleep all night and I work all day."

def repeat_lyrics():
    print_lyrics()
    print_lyrics()

repeat_lyrics()
\end{verbatim}
\afterverb
%
This program contains two function definitions: {\tt print\_lyrics} and
{\tt repeat\_lyrics}.  Function definitions get executed just like other
statements, but the effect is to create the new function.  The statements
inside the function do not get executed until the function is called, and
the function definition generates no output.

As you might expect, you have to create a function before you can
execute it.  In other words, the function definition has to be
executed before the first time it is called.

\begin{ex}
Move the last line of this program
to the top, so the function call appears before the definitions. Run 
the program and see what error
message you get.
\end{ex}

\begin{ex}
Move the function call back to the bottom
and move the definition of {\tt print\_lyrics} after the definition of
{\tt repeat\_lyrics}.  What happens when you run this program?
\end{ex}


\section{Flow of execution}
\index{flow of execution}

In order to ensure that a function is defined before its first use,
you have to know the order in which statements are executed, which is
called the {\bf flow of execution}.

Execution always begins at the first statement of the program.
Statements are executed one at a time, in order from top to bottom.

Function definitions do not alter the flow of execution of the
program, but remember that statements inside the function are not
executed until the function is called.

A function call is like a detour in the flow of execution. Instead of
going to the next statement, the flow jumps to the body of
the function, executes all the statements there, and then comes back
to pick up where it left off.

That sounds simple enough, until you remember that one function can
call another.  While in the middle of one function, the program might
have to execute the statements in another function. But while
executing that new function, the program might have to execute yet
another function!

Fortunately, Python is adept at keeping track of where it is, so each
time a function completes, the program picks up where it left off in
the function that called it.  When it gets to the end of the program,
it terminates.

What's the moral of this sordid tale?  When you read a program, you
don't always want to read from top to bottom.  Sometimes it makes
more sense if you follow the flow of execution.


\section{Parameters and arguments}
\label{parameters}
\index{parameter}
\index{function!parameter}
\index{argument}
\index{function!argument}

Some of the built-in functions you have used require arguments.  For
example, when you call {\tt math.sin} you pass a number (in radians)
as an argument.  Some functions take more than one argument; {\tt
math.pow} takes two, the base and the exponent.

Inside the function, the arguments are assigned to
variables called {\bf parameters}.  Here is an example of a
user-defined function that takes an argument:

\beforeverb
\begin{verbatim}
def print_twice(bruce):
    print bruce
    print bruce
\end{verbatim}
\afterverb
%
This function assigns the argument to a parameter
named {\tt bruce}.  When the function is called, it prints the value of
the parameter, whatever it is, twice.

This function works with any value that can be printed.

\beforeverb
\begin{verbatim}
>>> print_twice('Spam')
Spam
Spam
>>> print_twice(17)
17
17
>>> print_twice(math.pi)
3.14159265359
3.14159265359
\end{verbatim}
\afterverb
%
The same rules of composition that apply to built-in functions also
apply to user-defined functions, so we can use any kind of expression
as an argument for {\tt print\_twice}:

\beforeverb
\begin{verbatim}
>>> print_twice('Spam '*4)
Spam Spam Spam Spam
Spam Spam Spam Spam
>>> print_twice(math.cos(math.pi))
-1.0
-1.0
\end{verbatim}
\afterverb
%
The argument is evaluated before the function is called, so
in the examples the expressions {\tt 'Spam '*4} and
{\tt math.cos(math.pi)} are only evaluated once.

You can also use a variable as an argument:

\beforeverb
\begin{verbatim}
>>> michael = 'Eric, the half a bee.'
>>> print_twice(michael)
Eric, the half a bee.
Eric, the half a bee.
\end{verbatim}
\afterverb
%
The name of the variable we pass as an argument ({\tt michael}) has
nothing to do with the name of the parameter ({\tt bruce}).  It
doesn't matter what the value was called back home (in the caller);
here in {\tt print\_twice}, we call everybody {\tt bruce}.


\section{Variables and parameters are local}
\index{local variable}
\index{variable!local}

When you create a variable inside a function, it is {\bf local},
which means that it only
exists inside the function.  For example:

\beforeverb
\begin{verbatim}
def cat_twice(part1, part2):
    cat = part1 + part2
    print_twice(cat)
\end{verbatim}
\afterverb
%
This function takes two arguments, concatenates them, and prints
the result twice.  Here is an example that uses it:

\beforeverb
\begin{verbatim}
>>> line1 = 'Bing tiddle '
>>> line2 = 'tiddle bang.'
>>> cat_twice(line1, line2)
Bing tiddle tiddle bang.
Bing tiddle tiddle bang.
\end{verbatim}
\afterverb
%
When {\tt cat\_twice} terminates, the variable {\tt cat}
is destroyed.  If we try to print it, we get an exception:

\index{NameError}
\index{exception!NameError}

\beforeverb
\begin{verbatim}
>>> print cat
NameError: name 'cat' is not defined
\end{verbatim}
\afterverb
%
Parameters are also local.
For example, outside {\tt print\_twice}, there is no
such thing as {\tt bruce}.


\section{Stack diagrams}
\label{stackdiagram}
\index{stack diagram}
\index{function frame}
\index{frame}

To keep track of which variables can be used where, it is sometimes
useful to draw a {\bf stack diagram}.  Like state diagrams, stack
diagrams show the value of each variable, but they also show the
function each variable belongs to.

\index{stack diagram}
\index{diagram!stack}

Each function is represented by a {\bf frame}.  A frame is a box
with the name of a function
beside it and the parameters and variables of the function inside it.
The stack diagram for the
previous example looks like this:

\beforefig
\centerline{\includegraphics{figs/stack.eps}}
\afterfig

The frames are arranged in a stack that indicates which function
called which, and so on.  In this example, {\tt print\_twice}
was called by {\tt cat\_twice}, and {\tt cat\_twice} was called by {\tt
\_\_main\_\_}, which is a special name for the topmost frame.  When
you create a variable outside of any function, it belongs to {\tt
\_\_main\_\_}.

Each parameter refers to the same value as its corresponding
argument.  So, {\tt part1} has the same value as
{\tt line1}, {\tt part2} has the same value as {\tt line2},
and {\tt bruce} has the same value as {\tt cat}.

If an error occurs during a function call, Python prints the
name of the function, and the name of the function that called
it, and the name of the function that called {\em that}, all the
way back to {\tt \_\_main\_\_}.

For example, if you try to access {\tt cat} from within {\tt
print\_twice}, you get a {\tt NameError}:

\beforeverb
\begin{verbatim}
Traceback (innermost last):
  File "test.py", line 13, in __main__
    cat_and_print_twice(line1, line2)
  File "test.py", line 5, in cat_and_print_twice
    print_twice(cat)
  File "test.py", line 9, in print_twice
    print cat
NameError: name 'cat' is not defined
\end{verbatim}
\afterverb
%
This list of functions is called a {\bf traceback}.  It tells you what
program file the error occurred in, and what line, and what functions
were executing at the time.  It also shows the line of code that
caused the error.

\index{traceback}

The order of the functions in the traceback is the same as the
order of the frames in the stack diagram.  The function that is
currently running is at the bottom.


\section{Fruitful functions and void functions}

\index{fruitful function}
\index{void function}
\index{function!fruitful}
\index{function!void} 

Some of the functions we are using,
such as the math functions, yield results; for want of a better
name, I call them {\bf fruitful functions}.
Other functions, like {\tt
print\_twice}, perform an action but don't return a value.  They
are called {\bf void functions}.

When you call a fruitful function, you almost always
want to do something with the result; for example, you might
assign it to a variable or use it as part of an expression:

\beforeverb
\begin{verbatim}
x = math.cos(radians)
golden = (math.sqrt(5) + 1) / 2
\end{verbatim}
\afterverb
%
When you call a function in interactive mode, Python displays
the result:

\beforeverb
\begin{verbatim}
>>> math.sqrt(5)
2.2360679774997898
\end{verbatim}
\afterverb
%
But in a script, if you call a fruitful function all by itself,
the return value is lost forever!

\beforeverb
\begin{verbatim}
math.sqrt(5)
\end{verbatim}
\afterverb
%
This script computes the square root of 5, but since it doesn't store
or display the result, it is not very useful.

Void functions might display something on the screen or have some
other effect, but they don't have a return value.  If you try to
assign the result to a variable, you get a special value called
{\tt None}.

\index{{\tt None}}

\beforeverb
\begin{verbatim}
>>> result = print_twice('Bing')
Bing
Bing
>>> print result
None
\end{verbatim}
\afterverb
%
The value {\tt None} is not the same as the string {\tt 'None'}. 
It is a special value that has its own type:

\beforeverb
\begin{verbatim}
>>> print type(None)
<type 'NoneType'>
\end{verbatim}
\afterverb
%
The functions we have written so far are all void.  We will start
writing fruitful functions in a few chapters.


\section{Why functions?}

It may not be clear why it is worth the trouble to divide
a program into functions.  There are a lot of reasons; here
are a few:

\begin{itemize}

\item Creating a new function gives you an opportunity to name a group
of statements, which makes your program easier to read and debug.

\item Functions can make a program smaller by eliminating repetitive
code.  Later, if you make a change, you only have
to make it in one place.

\item Dividing a long program into functions allows you to debug the
parts one at a time and then assemble them into a working whole.

\item Well-designed functions are often useful for many programs.
Once you write and debug one, you can reuse it.

\end{itemize}


\section{Debugging}

If you are using a text editor to write your scripts, you might
run into problems with spaces and tabs.  The best way to avoid
these problems is to use spaces exclusively (no tabs).  Most text
editors that know about Python do this by default, but some
don't.

Tabs and spaces are usually invisible, which makes them
hard to debug, so try to find an editor that manages indentation
for you.

Also, don't forget to save your program before you run it.  Some
development environments do this automatically, but some don't.
In that case the program you are looking at in the text editor
is not the same as the program you are running (the one on disk).

Debugging can take a long time if you keep running the same,
incorrect, program over and over!  And that brings me to the
Third Theorem of Debugging:

\begin{quote}
Make sure that the code you are looking at is
the code you are running.
\end{quote}

If you're not sure, put something like {\tt print 'hello!'} at the
beginning of the program and run it again.  If you don't see
{\tt 'hello!'}, you're not running the right program!




\section{Glossary}

\begin{description}

\item[function:] A named sequence of statements that performs some
useful operation.  Functions may or may not take arguments and may or
may not produce a result.
\index{function}

\item[function definition:]  A statement that creates a new function,
specifying its name, parameters, and the statements it executes.
\index{function definition}

\item[function object:]  A value created by a function definition.
The name of the function is a variable that refers to a function
object.
\index{function definition}

\item[header:] The first line of a function definition.
\index{header}

\item[body:] The sequence of statements inside a function definition.
\index{body}

\item[parameter:] A name used inside a function to refer to the value
passed as an argument.
\index{parameter}

\item[function call:] A statement that executes a function. It
consists of the function name followed by an argument list.
\index{function call}

\item[argument:]  A value provided to a function when the function is called.
This value is assigned to the corresponding parameter in the function.
\index{argument}

\item[local variable:]  A variable defined inside a function.  A local
variable can only be used inside its function.
\index{local variable}

\item[return value:]  The result of a function.  If a function call
is used as an expression, the return value is the value of
the expression.
\index{return value}

\item[fruitful function:] A function that returns a value.
\index{fruitful function}

\item[void function:] A function that doesn't return a value.
\index{void function}

\item[module:] A file that contains a
collection of related functions and other definitions.
\index{module}

\item[import statement:] A statement that reads a module file and creates
a module object.
\index{import}

\item[module object:] A value created by an {\tt import} statement
that provides access to the values defined in a module.
\index{module}

\item[dot notation:]  The syntax for calling a function in another
module by specifying the module name followed by a dot (period) and
the function name.
\index{dot notation}

\item[composition:] Using an expression as part of a larger expression,
or a statement as part of a larger statement.
\index{composition}

\item[flow of execution:]  The order in which statements are executed during
a program run.
\index{flow of execution}

\item[stack diagram:]  A graphical representation of a stack of functions,
their variables, and the values they refer to.
\index{stack diagram}

\item[frame:]  A box in a stack diagram that represents a function call.
It contains the local variables and parameters of the function.
\index{function frame}
\index{frame}

\item[traceback:]  A list of the functions that are executing,
printed when an exception occurs.
\index{traceback}


\end{description}


\section{Exercises}

\begin{ex}
Fermat's Last Theorem says that there are no integers
$a$, $b$, and $c$ such that

\[ a^n + b^n = c^n \]
%
for any values of $n$ greater than 2.

Write a function named {\tt check\_fermat} that takes four
parameters---{\tt a}, {\tt b}, {\tt c} and {\tt n}---and
that checks to see if Fermat's theorem holds.  If
$n$ is greater than 2 and it turns out to be true that 

\[a^n + b^n = c^n \]
%
the program should print ``Holy smokes, Fermat was wrong!''
Otherwise the program should print ``No, that doesn't work.''
\end{ex}


\begin{ex}
Python provides a built-in function called {\tt len} that
returns the length of a string, so the value of {\tt len('allen')} is 5.

Write a function named {\tt right\_justify} that takes a string
named {\tt s} as a parameter and that prints the string with enough
leading spaces so that the last letter of the string is in column 70
of the display.

\begin{verbatim}
>>> right_justify('allen')
                                                                 allen
\end{verbatim}
\end{ex}


\begin{ex}
\item Write a function that draws grids like this in any
size\footnote{Based on an exercise in Oualline, {\em Practical C
Programming, Third Edition}, O'Reilly (1997)}:

\beforeverb
\begin{verbatim}
+ - - - - - + - - - - - +
|           |           |
|           |           |
|           |           |
|           |           |
+ - - - - - + - - - - - +
|           |           |
|           |           |
|           |           |
|           |           |
+ - - - - - + - - - - - +
\end{verbatim}
\afterverb
%
Hint: to print more than one value on a line, you can print
a comma-separated sequence:

\beforeverb
\begin{verbatim}
print '+', '-'
\end{verbatim}
\afterverb
%
If the sequence ends with a comma, Python leaves the line unfinished,
so the value printed next appears on the same line.

\beforeverb
\begin{verbatim}
print '+', 
print '-'
\end{verbatim}
\afterverb
%
The output of these statements is {\tt '+ -'}.

\end{ex}





\chapter{Case study: interface design}
\label{turtlechap}

\section{TurtleWorld}

To accompany this book, I have written a suite of modules called
Swampy.  One of these modules is TurtleWorld, which provides
a set of functions for drawing lines by steering
turtles around the screen.

You can download Swampy from \url{allendowney.com/swampy}.  Move
into the directory that contains {\tt TurtleWorld.py}, start
the Python interpreter, and type:

\beforeverb
\begin{verbatim}
>>> from TurtleWorld import *
\end{verbatim}
\afterverb
%
This is a variation of the {\tt import} statement we saw before.
Instead of creating a module object, it imports the functions
in the module directly, so we can access them without using dot
notation.  For example, to create TurtleWorld, type:

\beforeverb
\begin{verbatim}
>>> TurtleWorld()
\end{verbatim}
\afterverb
%
A window should appear on the screen and the interpreter should
display something like:

\beforeverb
\begin{verbatim}
<TurtleWorld.TurtleWorld instance at 0xb7f0c2ec>
\end{verbatim}
\afterverb
%
The angle-brackets indicate that the return value from 
{\tt TurtleWorld} is an {\bf instance} of a TurtleWorld
as defined in module {\tt TurtleWorld}.  In this context,
an instance is a member of a set;
this TurtleWorld is one of the set of possible TurtleWorlds.

To create a turtle, type:

\beforeverb
\begin{verbatim}
>>> bob = Turtle()
\end{verbatim}
\afterverb
%
In this case we assign the return value from {\tt Turtle} to
a variable named {\tt bob} so we can refer to it later (we don't
really have a way to refer to the TurtleWorld).

The turtle-steering functions are {\tt fd} and {\tt bk} for
forward and backward, and {\tt lt} and {\tt rt} for left and
right turns.

To draw a right angle, type:

\beforeverb
\begin{verbatim}
>>> fd(bob, 100)
>>> rt(bob)
>>> fd(bob, 100)
\end{verbatim}
\afterverb
%
The first line (and third) tells {\tt bob} to take 100 steps
forward.  The second line tells him to turn right.  In the
TurtleWorld window you should
see the turtle move east and then south, leaving two line
segments behind.

Before you go on, use {\tt bk} and {\tt lt} to put the turtle
back where it started.


\section{Simple repetition}
\label{repetition}

If you haven't already, move into the directory that contains {\tt
TurtleWorld.py}.  Create a file named {\tt polygon.py} and type in the
code from the previous section:

\begin{verbatim}
from TurtleWorld import *
TurtleWorld()
bob = Turtle()

fd(bob, 100)
lt(bob)
fd(bob, 100)
\end{verbatim}
%
When you run the program, you should see {\tt bob} draw a right
angle, but when the program finishes, the window disappears.
Add the line

\begin{verbatim}
wait_for_user()
\end{verbatim}
%
at the {\em end} of the program and run it again.  Now the window
stays up until you close it.

Now modify the program to draw a square.  Don't turn the page until
you've got it working!



Chances are you wrote something like this (leaving out the code
that creates TurtleWorld and waits for the user):

\begin{verbatim}
fd(bob, 100)
lt(bob)

fd(bob, 100)
lt(bob)

fd(bob, 100)
lt(bob)

fd(bob, 100)
\end{verbatim}
%
We can do the same thing more concisely with a {\tt for} statement.
Add this example to {\tt polygon.py} and run it again:

\beforeverb
\begin{verbatim}
for i in range(4):
    print 'Hello!'
\end{verbatim}
\afterverb
%
You should see something like this:

\beforeverb
\begin{verbatim}
Hello!
Hello!
Hello!
Hello!
\end{verbatim}
\afterverb
%
This is the simplest use of the {\tt for} statement; we will see
more later.  But that should be enough to let you rewrite your
square-drawing program.  Don't turn the page until you do.

Here is a {\tt for} statement that draws a square:

\beforeverb
\begin{verbatim}
for i in range(4):
    fd(bob, 100)
    lt(bob)
\end{verbatim}
\afterverb
%
The syntax of a {\tt for} statement is similar to a function
definition.  It has a header that ends with a colon and an indented
body.  The body can contain any number of any kind of statement.

A {\tt for} statement is sometimes called a {\bf loop} because
the flow of execution runs through the body and then loops back
to the top.  In this case, it runs the body four times.

This version is actually a little different from the previous
square-drawing code because it makes another left turn after
drawing the last side of the square.  The extra turn takes a little
more time, but it simplifies the code if we do the same thing
every time through the loop.  This version also has the effect
of leaving the turtle back in the starting position, facing in
the starting direction.

\section{Exercises}

The following is a series of exercises using TurtleWorld.  They
are meant to be fun, but they have a point, too.  While you are
working on them, think about what the point is.

The following sections have solutions to the exercises, so
don't look until you have finished (or at least tried).

\begin{enumerate}

\item Write a function called {\tt square} that takes a parameter
named {\tt t}, which is a turtle.  It should use the turtle to draw
a square.

Write a function call that passes {\tt bob} as an argument to
{\tt square}, and then run the program again.

\item Add another parameter, named {\tt length}, to {\tt square}.
Modify the body so length of the sides is {\tt length}, and then
modify the function call to provide a second argument.  Run the
program again.  Test your program with a range of values for {\tt
length}.

\item The functions {\tt lt} and {\tt rt} make 90-degree turns by
default, but you can provide a second argument to that specifies the
number of degrees.  For example, {\tt lt(bob, 45)} turns {\tt bob} 45
degrees to the left.

Make a copy of {\tt square} and change the name to {\tt polygon}.  Add
another parameter named {\tt n} and modify the body so it draws an
n-sided regular polygon.  Hint: The angles of an n-sided regular
polygon are $360.0 / n$ degrees.


\item Write a function called {\tt circle} that takes a turtle, {\tt t},
and radius, {\tt r}, as parameters and that draws an approximate circle
by invoking {\tt polygon} with an appropriate length and number of
sides.  Test your function with a range of values of {\tt r}.

Hint: figure out the circumference of the circle and make sure that
{\tt length * n = circumference}.

Another hint: if {\tt bob} is too slow for you, you can speed
him up by changing {\tt bob.delay}, which is the time between moves,
in seconds.  {\tt bob.delay = 0.01} ought to get him moving.

\item Make a more general version of {\tt circle} called {\tt arc}
that takes an additional parameter {\tt angle}, which determines
what fraction of a circle to draw.  {\tt angle} is in units of
degrees, so when {\tt angle=360}, {\tt arc} should draw a complete
circle.

\end{enumerate}

\section{Encapsulation}

The first exercise asks you to put your square-drawing code
into a function definition and then call the function, passing
the turtle as a parameter.  Here is a solution:

\beforeverb
\begin{verbatim}
def square(t):
    for i in range(4):
        fd(t, 100)
        lt(t)

square(bob)
\end{verbatim}
\afterverb
%
The innermost statements, {\tt fd} and {\tt lt} are
indented twice to show that they are inside the {\tt for} loop,
which is inside the function definition.  The next line,
{\tt square(bob)}, is flush with the left margin, so that is the
end of both the {\tt for} loop and the function definition.

Inside the function, {\tt t} refers to the same turtle {\tt bob}
refers to, so {\tt lt(t)} has the same effect as {\tt lt(bob)}.
So why not call the parameter {\tt bob}?  The idea is that {\tt t}
can be any turtle, not just {\tt bob}, so you could create
a second turtle and pass it as an argument to {\tt square}:

\beforeverb
\begin{verbatim}
ray = Turtle()
square(ray)
\end{verbatim}
\afterverb
%
Wrapping a piece of code up in a function is called {\bf
encapsulation}.  One of the benefits of encapsulation is that it
attaches a name to the code, which serves as a kind of documentation.
Another advantage is that if you re-use the code, it is more concise
to call a function twice than to copy and paste the body!

\index{encapsulation}


\section{Generalization}

The next step is to add a {\tt length} parameter to {\tt square}.
Here is a solution:

\beforeverb
\begin{verbatim}
def square(t, length):
    for i in range(4):
        fd(t, length)
        lt(t)

square(bob, 100)
\end{verbatim}
\afterverb
%
Adding a parameter to a function is called {\bf generalization}
because it makes the function more general: in the previous
version, the square is always the same size; in this version
it can be any size.

The next step is also a generalization.  Instead of drawing
squares, {\tt polygon} draws regular polygons with any number of
sides.  Here is a solution:

\beforeverb
\begin{verbatim}
def polygon(t, length, n):
    angle = 360.0 / n
    for i in range(n):
        fd(t, length)
        lt(t, angle)

polygon(bob, 70, 7)
\end{verbatim}
\afterverb
%
This draws a 7-sides polygon with side length 70.  If you have
more than a few numeric arguments, it is easy to forget what they
are, or what order they should be in.  It is legal, and sometimes
helpful, to include the names of the parameters in the argument
list:

\beforeverb
\begin{verbatim}
polygon(bob, length=70, n=7)
\end{verbatim}
\afterverb
%
This syntax makes the program more readable.  It is also a reminder
about how arguments and parameters work: when you call a function, the
arguments are assigned to the parameters.


\section{Interface design}

The next step is to write {\tt circle}, which takes a radius,
{\tt r} as a parameter.  Here is a simple solution that uses
{\tt polygon} to draw a 50-sided polygon:

\beforeverb
\begin{verbatim}
def circle(t, r):
    circumference = 2 * math.pi * r
    n = 50
    length = circumference / n
    polygon(t, length, n)
\end{verbatim}
\afterverb
%
The first line computes the circumference of a circle with radius
{\tt r} using the formula $2 \pi r$.  Since we use {\tt math.pi}, we
have to import {\tt math}.  By convention, {\tt import} statements
are usually at the beginning of the script.

{\tt n} is the number of line segments in our approximation of a circle,
so {\tt length} is the length of each segment.  Thus, {\tt polygon}
draws a 50-sides polygon that approximates a circle with radius {\tt r}.

One limitation of this solution is that {\tt n} is a constant, which
means that for very big circles, the line segments are too long, and
for small circles, we waste time drawing very small segments.  One
solution would be to generalize the function by taking {\tt n} as
a parameter.  This would give the user (whoever calls {\tt circle})
more control, but the interface would be less clean.

\index{interface}

The {\bf interface} of a function is a summary of how it is used: what
are the parameters?  What does the function do?  And what is the return
value?  An interface is ``clean'' if it is ``as simple as
possible, but not simpler. (Einstein)''

\index{Einstein, Albert}

In this example, {\tt r} belongs in the interface because it
specifies the circle to be drawn.  {\tt n} is less appropriate
because it pertains to the details of {\em how} the circle should
be rendered.

Rather than clutter up the interface, it is better
to choose the value of {\tt n} adaptively,
depending on {\tt circumference}:

\beforeverb
\begin{verbatim}
def circle(t, r):
    circumference = 2 * math.pi * r
    n = int(circumference / 4)
    length = circumference / n
    polygon(t, length, n)
\end{verbatim}
\afterverb
%
Now the number of segments is (approximately) {\tt circumference/4},
so the length of each segment is (approximately) 4, which is small
enough that the circles look good, but big enough to be efficient,
and appropriate for any size circle.


\section{Refactoring}

When we wrote {\tt circle}, we were able to re-use {\tt polygon}
because a many-sided polygon is a good approximation of a circle.
But {\tt arc} is not as cooperative; we can't use {\tt polygon}
or {\tt circle} to draw an arc.

An alternative is to start with a copy
of {\tt polygon} and transform it into {\tt arc}.  The result
might look like this:

\beforeverb
\begin{verbatim}
def arc(t, r, angle):
    arclength = 2 * math.pi * r * angle / 360
    n = int(arclength / 4)
    length = arclength / n
    step_angle = float(angle) / n
    
    for i in range(n):
        fd(t, length)
        lt(t, step_angle)
\end{verbatim}
\afterverb
%
The second half of this function looks like {\tt polygon}, but we
can't re-use {\tt polygon} without changing the interface.  We could
generalize {\tt polygon} to take an angle as a third argument,
but then {\tt polygon} would no longer be an appropriate name!
Instead, let's call the more general function {\tt polyline}:

\beforeverb
\begin{verbatim}
def polyline(t, length, n, angle):
    for i in range(n):
        fd(t, length)
        lt(t, angle)
\end{verbatim}
\afterverb
%
Now we can rewrite {\tt polygon} and {\tt arc} to use {\tt polyline}:

\beforeverb
\begin{verbatim}
def polygon(t, length, n):
    angle = 360.0 / n
    polyline(t, length, n, angle)

def arc(t, r, angle):
    arclength = 2 * math.pi * r * angle / 360
    n = int(arclength / 4)
    length = arclength / n
    polyline(t, length, n, float(angle)/n)
\end{verbatim}
\afterverb
%
Finally, we can rewrite {\tt circle} to use {\tt arc}:

\beforeverb
\begin{verbatim}
def circle(t, r):
    arc(t, r, 360.0)
\end{verbatim}
\afterverb
%
This process---rearranging a program to improve function
interfaces and facilitate code re-use---is called {\tt refactoring}.
In this case, we noticed that there was similar code in {\tt arc} and
{\tt polygon}, so we ``factored it out'' into {\tt polyline}.

\index{refactoring}

If we had planned ahead, we might have written {\tt polyline} first
and avoided refactoring, but often you don't know enough at the
beginning of a project to design all the interfaces.  Once you start
coding, you understand the problem better.  Sometimes refactoring is a
sign that you have learned something.


\section{A development plan}

A {\bf development plan} is a process for writing programs.
The process we used
in this case study is what I call ``EGR'' for ``encapsulation,
generalization and refactoring.''  The steps of
EGR are:

\begin{enumerate}

\item Start by writing a small program with no function definitions.

\item Once you get the program working, encapsulate it in a function
and give it a name.

\item Generalize the function by adding appropriate parameters.

\item Repeat steps 1--3 until you have a set of working functions.
Copy and paste working code to avoid retyping (and re-debugging).

\item Look for opportunities to improve the program by refactoring.
For example, if you have similar code in several places, consider
factoring it into an appropriately general function.

\end{enumerate}

EGR has some drawbacks---we will see alternatives later---but it can
be useful if you don't know ahead of time how to divide the program
into functions.  This approach lets you design as you go along.


\section{docstring}
\label{docstring}

A {\bf docstring} is a string at the beginning of a function that
explains the interface (``doc'' is short for ``documentation'').  Here
is an example:

\beforeverb
\begin{verbatim}
def polyline(t, length, n, angle):
    """Draw n line segments with the given length and
    angle (in degrees) between them.  t is a turtle.
    """    
    for i in range(n):
        fd(t, length)
        lt(t, angle)
\end{verbatim}
\afterverb
%
This docstring is a triple-quoted string, also known
as a multi-line string because the triple quotes allow the string
to span more than one line.

It is terse, but it contains the essential information
someone would need to use this function.  It explains concisely what
the function does (without getting into the details of how it does
it).  It explains what effect each parameter has on the behavior of
the function and what type each parameter should be (if it is not
obvious).

Writing this kind of documentation is an important part of interface
design.  A well-designed interface should be simple to explain;
if you are having a hard time explaining one of your functions,
that might mean that the interface could be improved.


\section{Glossary}

\begin{description}

\item[instance:] A member of a set.  The TurtleWorld in this
chapter is a member of the set of TurtleWorlds.
\index{instance}

\item[loop:] A part of a program that can execute repeatedly.
\index{loop}

\item[encapsulation:] The process of transforming a sequence of
statements into a function definition.
\index{encapsulation}

\item[generalization:] The process of replacing something
unnecessarily specific (like a number) with something appropriately
general (like a variable or parameter).
\index{generalization}

\item[interface:] A description of how to use a function, including
the name and descriptions of the arguments and return value.
\index{interface}

\item[development plan:] A process for writing programs.
\index{development plan}

\item[docstring:]  A string that appears in a function definition
to document the function's interface.
\index{docstring}

\end{description}


\section{Exercises}

\begin{enumerate}

\item Write appropriate docstrings for {\tt polygon} and
{\tt circle}.

\item Draw a stack diagram that shows the state of the program
while executing {\tt circle(bob, 100)}.  You can do the
arithmetic by hand or add {\tt print} statements to the code.

\item Write an appropriately general set of functions that
can draw flowers like this:

\centerline{\includegraphics[height=1in]{figs/flowers.eps}}

\item Write an appropriately general set of functions that
can draw shapes like this:

\centerline{\includegraphics[height=0.9in]{figs/pies.eps}}

\end{enumerate}




\chapter{Conditionals and recursion}

\section{Modulus operator}
\index{integer division}
\index{division!integer}
\index{modulus operator}
\index{operator!modulus}

The {\bf modulus operator} works on integers
and yields the remainder when the first operand is divided by the
second.  In Python, the modulus operator is a percent sign ({\tt
\%}).  The syntax is the same as for other operators:

\beforeverb
\begin{verbatim}
>>> quotient = 7 / 3
>>> print quotient
2
>>> remainder = 7 % 3
>>> print remainder
1
\end{verbatim}
\afterverb
%
So 7 divided by 3 is 2 with 1 left over.

The modulus operator turns out to be surprisingly useful.  For
example, you can check whether one number is divisible by another---if
{\tt x \% y} is zero, then {\tt x} is divisible by {\tt y}.

Also, you can extract the right-most digit
or digits from a number.  For example, {\tt x \% 10} yields the
right-most digit of {\tt x} (in base 10).  Similarly {\tt x \% 100}
yields the last two digits.


\section{Boolean expressions}
\index{boolean expression}
\index{expression!boolean}
\index{logical operator}
\index{operator!logical}

A {\bf boolean expression} is an expression that is either true
or false.  The following examples use the 
operator {\tt ==}, which compares two operands and produces
{\tt True} if they are equal and {\tt False} otherwise:

\beforeverb
\begin{verbatim}
>>> 5 == 5
True
>>> 5 == 6
False
\end{verbatim}
\afterverb
%
{\tt True} and {\tt False} are special
values that belong to the type {\tt bool}; they are not strings:

\beforeverb
\begin{verbatim}
>>> type(True)
<type 'bool'>
>>> type(False)
<type 'bool'>
\end{verbatim}
\afterverb
%
The {\tt ==} operator is one of the {\bf comparison operators}; the
others are:

\beforeverb
\begin{verbatim}
      x != y               # x is not equal to y
      x > y                # x is greater than y
      x < y                # x is less than y
      x >= y               # x is greater than or equal to y
      x <= y               # x is less than or equal to y
\end{verbatim}
\afterverb
%
Although these operations are probably familiar to you, the Python
symbols are different from the mathematical symbols.  A common error
is to use a single equal sign ({\tt =}) instead of a double equal sign
({\tt ==}).  Remember that {\tt =} is an assignment operator and
{\tt ==} is a comparison operator.   There is no such thing as
{\tt =<} or {\tt =>}.


\section {Logical operators}
\index{logical operator}
\index{operator!logical}

There are three {\bf logical operators}: {\tt and}, {\tt
or}, and {\tt not}.  The semantics (meaning) of these operators is
similar to their meaning in English.  For example,
{\tt x > 0 and x < 10} is true only if {\tt x} is greater than 0
{\em and} less than 10.

{\tt n\%2 == 0 or n\%3 == 0} is true if {\em either} of the conditions
is true, that is, if the number is divisible by 2 {\em or} 3.

Finally, the {\tt not} operator negates a boolean
expression, so {\tt not (x > y)} is true if {\tt x > y} is false,
that is, if {\tt x} is less than or equal to {\tt y}.

Strictly speaking, the operands of the logical operators should be
boolean expressions, but Python is not very strict.
Any nonzero number is interpreted as ``true.''

\beforeverb
\begin{verbatim}
>>> 17 and True
True
\end{verbatim}
\afterverb
%
This flexibility can be useful, but there are some subtleties to
it that might be confusing.  You might want to avoid it (unless
you know what you are doing).


\section{Conditional execution}
\label{conditional execution}
\index{conditional branching}
\index{conditional execution}

In order to write useful programs, we almost always need the ability
to check conditions and change the behavior of the program
accordingly.  {\bf Conditional statements} give us this ability.  The
simplest form is the {\tt if} statement:

\beforeverb
\begin{verbatim}
if x > 0:
    print 'x is positive'
\end{verbatim}
\afterverb
%
The boolean expression after the {\tt if} statement is
called the {\bf condition}.  If it is true, then the indented
statement gets executed.  If not, nothing happens.

\index{condition}
\index{compound statement}
\index{statement!compound}

{\tt if} statements have the same structure as function definitions:
a header followed by an indented block.  Statements like this are
called {\bf compound statements}.

There is no limit on the number of statements that can appear in
the body, but there has to be at least one.
Occasionally, it is useful to have a body with no statements (usually
as a place keeper for code you haven't written yet).  In that
case, you can use the {\tt pass} statement, which does nothing.

\index{pass statement}
\index{statement!pass}

\beforeverb
\begin{verbatim}
if x < 0:
    pass          # need to handle negative values!
\end{verbatim}
\afterverb
%

\section{Alternative execution}
\label{alternative execution}

A second form of the {\tt if} statement is alternative execution,
in which there are two possibilities and the condition determines
which one gets executed.  The syntax looks like this:

\beforeverb
\begin{verbatim}
if x%2 == 0:
    print 'x is even'
else:
    print 'x is odd'
\end{verbatim}
\afterverb
%
If the remainder when {\tt x} is divided by 2 is 0, then we
know that {\tt x} is even, and the program displays a message to that
effect.  If the condition is false, the second set of statements is
executed.  Since the condition must be true or false, exactly one of
the alternatives will be executed.  The alternatives are called
{\bf branches}, because they are branches in the flow of execution.

\index{branch}

% As an aside, if you need to check the parity (evenness or
% oddness) of numbers often, you might encapsulate this code in a
% function:

% \beforeverb
% \begin{verbatim}
% def print_parity(x):
%     if x%2 == 0:
%         print x, 'is even'
%     else:
%         print x, 'is odd'
% \end{verbatim}
% \afterverb
%
% For any value of {\tt x}, {\tt print_parity} displays an
% appropriate message.
% When you call it, you can provide any integer expression
% as an argument.

% \beforeverb
% \begin{verbatim}
% >>> print_parity(17)
% 17 is odd
% >>> y = 17
% >>> print_parity(y+1)
% 18 is even
% \end{verbatim}
% \afterverb
%


\section{Chained conditionals}
\index{chained conditional}
\index{conditional!chained}

Sometimes there are more than two possibilities and we need more than
two branches.  One way to express a computation like that is a {\bf
chained conditional}:

\beforeverb
\begin{verbatim}
if x < y:
    print 'x is less than y'
elif x > y:
    print 'x is greater than y'
else:
    print 'x and y are equal'
\end{verbatim}
\afterverb
%
{\tt elif} is an abbreviation of ``else if.''  Again, exactly one
branch will be executed.  There is no limit on the number of {\tt
elif} statements.  If there is an {\tt else} clause, it has to be
at the end, but there doesn't have to be one.

\beforeverb
\begin{verbatim}
if choice == 'A':
    functionA()
elif choice == 'B':
    functionB()
elif choice == 'C':
    functionC()
\end{verbatim}
\afterverb
%
Each condition is checked in order.  If the first is false,
the next is checked, and so on.  If one of them is
true, the corresponding branch executes, and the statement
ends.  Even if more than one condition is true, only the
first true branch executes.  


\section{Nested conditionals}

One conditional can also be nested within another.  We could have
written the trichotomy example like this:

\beforeverb
\begin{verbatim}
if x == y:
    print 'x and y are equal'
else:
    if x < y:
        print 'x is less than y'
    else:
        print 'x is greater than y'
\end{verbatim}
\afterverb
%
The outer conditional contains two branches.  The
first branch contains a simple statement.  The second branch
contains another {\tt if} statement, which has two branches of its
own.  Those two branches are both simple statements,
although they could have been conditional statements as well.

Although the indentation of the statements makes the structure
apparent, nested conditionals become difficult to read very
quickly. In general, it is a good idea to avoid them when you can.

Logical operators often provide a way to simplify nested conditional
statements.  For example, we can rewrite the following code using a
single conditional:

\beforeverb
\begin{verbatim}
if 0 < x:
    if x < 10:
        print 'x is a positive single digit.'
\end{verbatim}
\afterverb
%
The {\tt print} statement is executed only if we make it past both
conditionals, so we can get the same effect with the {\tt and} operator:

\beforeverb
\begin{verbatim}
if 0 < x and x < 10:
    print 'x is a positive single digit.'
\end{verbatim}
\afterverb




\section{Recursion}
\label{recursion}
\index{recursion}

It is legal for one function to call another;
it is also legal for a function to call itself.  It may not be obvious
why that is a good thing, but it turns out to be one of the most
magical things a program can do.
For example, look at the following function:

\beforeverb
\begin{verbatim}
def countdown(n):
    if n <= 0:
        print 'Blastoff!'
    else:
        print n
        countdown(n-1)
\end{verbatim}
\afterverb
%
If {\tt n} is 0 or negative, it outputs the word, ``Blastoff!''
Otherwise, it outputs {\tt n} and then calls a function named {\tt
countdown}---itself---passing {\tt n-1} as an argument.

What happens if we call this function like this?

\beforeverb
\begin{verbatim}
>>> countdown(3)
\end{verbatim}
\afterverb
%
The execution of {\tt countdown} begins with {\tt n=3}, and since
{\tt n} is greater than 0, it outputs the value 3, and then calls itself...

\begin{quote}
The execution of {\tt countdown} begins with {\tt n=2}, and since
{\tt n} is greater than 0, it outputs the value 2, and then calls itself...

\begin{quote}
The execution of {\tt countdown} begins with {\tt n=1}, and since
{\tt n} is greater than 0, it outputs the value 1, and then calls itself...

\begin{quote}
The execution of {\tt countdown} begins with {\tt n=0}, and since {\tt
n} is not greater than 0, it outputs the word, ``Blastoff!'' and then
returns.
\end{quote}

The {\tt countdown} that got {\tt n=1} returns.
\end{quote}

The {\tt countdown} that got {\tt n=2} returns.
\end{quote}

The {\tt countdown} that got {\tt n=3} returns.

And then you're back in {\tt \_\_main\_\_}.  So, the
total output looks like this:

\beforeverb
\begin{verbatim}
3
2
1
Blastoff!
\end{verbatim}
\afterverb
%
A function that calls itself is {\bf recursive}; the process is
called {\bf recursion}.

\index{recursion}
\index{function!recursive}

As another example, we can write a function that prints a
string {\tt n} times.

\beforeverb
\begin{verbatim}
def print_n(s, n):
    if n <= 0:
        return
    print s
    print_n(s, n-1)
\end{verbatim}
\afterverb
%
If {\tt n <= 0} the {\tt return} statement exits the function.  The
flow of execution immediately returns to the caller, and the remaining
lines of the function are not executed.

\index{return statement}
\index{statement!return}

The rest of the function is similar to {\tt countdown}: if {\tt n} is
greater than 0, it displays {\tt s} and then calls itself to display
{\tt s} $n-1$ additional times.  So the number of lines of output
is {\tt 1 + (n - 1)} which, if you do your algebra right, comes out to
{\tt n}.

For simple examples like this, it is probably easier to use a {\tt
for} loop.  But we will see examples later that are hard to write
with a {\tt for} loop and easy to write with recursion, so it is
good to start early.



\section{Stack diagrams for recursive functions}
\index{stack diagram}
\index{function frame}
\index{frame}

In Section~\ref{stackdiagram}, we used a stack diagram to represent
the state of a program during a function call.  The same kind of
diagram can help interpret a recursive function.

Every time a function gets called, Python creates a new function
frame, which contains the function's local variables and parameters.
For a recursive function, there might be more than one frame on the
stack at the same time.

This figure shows a stack diagram for {\tt countdown} called with
{\tt n = 3}:

\beforefig
\centerline{\includegraphics{figs/stack2.eps}}
\afterfig

As usual, the top of the stack is the frame for {\tt \_\_main\_\_}.
It is empty because we did not create any variables in {\tt
\_\_main\_\_} or pass any arguments to it.

The four {\tt countdown} frames have different values for the
parameter {\tt n}.  The bottom of the stack, where {\tt n=0}, is
called the {\bf base case}.  It does not make a recursive call, so
there are no more frames.

\begin{quote}
Draw a stack diagram for {\tt print\_n} called with
{\tt s = 'Hello'} and {\tt n=4}.
\end{quote}

\index{base case}
\index{recursion!base case}


\section{Infinite recursion}
\index{infinite recursion}
\index{recursion!infinite}
\index{runtime error}
\index{error!runtime}
\index{traceback}

If a recursion never reaches a base case, it goes on making
recursive calls forever, and the program never terminates.  This is
known as {\bf infinite recursion}, and it is generally not
a good idea.  Here is a minimal program with an infinite recursion:

\beforeverb
\begin{verbatim}
def recurse():
    recurse()
\end{verbatim}
\afterverb
%
In most programming environments, a program with infinite recursion
does not really run forever.  Python reports an error
message when the maximum recursion depth is reached:

\index{exception!RuntimeError}
\index{RuntimeError}

\beforeverb
\begin{verbatim}
  File "<stdin>", line 2, in recurse
  File "<stdin>", line 2, in recurse
  File "<stdin>", line 2, in recurse
                  .   
                  .
                  .
  File "<stdin>", line 2, in recurse
RuntimeError: Maximum recursion depth exceeded
\end{verbatim}
\afterverb
%
This traceback is a little bigger than the one we saw in the
previous chapter.  When the error occurs, there are 1000
{\tt recurse} frames on the stack!


\section{Keyboard input}

The programs we have written so far are a bit rude in the sense that
they accept no input from the user.  They just do the same thing every
time.

Python provides a built-in function called {\tt raw\_input}
 that gets input from the keyboard.
When this function is
called, the program stops and waits for the user to type something.
When the user presses {\sf Return} or {\sf Enter}, the program resumes and
{\tt raw\_input} returns what the user typed as a string.

\beforeverb
\begin{verbatim}
>>> input = raw_input()
What are you waiting for?
>>> print input
What are you waiting for?
\end{verbatim}
\afterverb
%
Before calling {\tt raw\_input}, it is a good idea to print a prompt
telling the user what to input.  {\tt raw\_input} takes a prompt as an
argument:

\index{prompt}

\beforeverb
\begin{verbatim}
>>> name = raw_input('What...is your name?\n')
What...is your name?
Arthur, King of the Britons!
>>> print name
Arthur, King of the Britons!
\end{verbatim}
\afterverb
%
The sequence \verb+\n+ at the end of the prompt represents a newline,
which is a special character that causes a line break.
That's why the user's input appears below the prompt.

\index{newline}

If you expect the user to type an integer, you can try to convert
the return value to {\tt int}:

\beforeverb
\begin{verbatim}
>>> prompt = 'What...is the airspeed velocity of an unladen swallow?\n'
>>> speed = raw_input(prompt)
What...is the airspeed velocity of an unladen swallow?
17
>>> int(speed)
17
\end{verbatim}
\afterverb
%
But if the user types something other than a string of digits,
you get an exception:

\beforeverb
\begin{verbatim}
>>> speed = raw_input(prompt)
What...is the airspeed velocity of an unladen swallow?
What do you mean, an African or a European swallow?
>>> int(speed)
ValueError: invalid literal for int()
\end{verbatim}
\afterverb
%
We will see how to handle this kind of error later.


\section{Debugging}

The traceback Python displays when an error occurs contains
a lot of information, but it can be overwhelming, especially
when there are many frames on the stack.  The most
useful pieces are usually:

\begin{itemize}

\item what kind of error it was, and

\item where it occurred.

\end{itemize}

Syntax errors are usually easy to find, but there are a few
gotchas.  Whitespace errors can be tricky because spaces and
tabs are invisible and we are used to ignoring them.

\beforeverb
\begin{verbatim}
>>> x = 5
>>>  y = 6
  File "<stdin>", line 1
    y = 6
    ^
SyntaxError: invalid syntax
\end{verbatim}
\afterverb
%
In this example, the problem is that the second line is indented by
one space.  But the error message points to {\tt y}, which is
misleading.  In general, error messages indicate where the error was
discovered, but the actual error might be earlier in the code,
sometimes on a previous line.

The same is true of run time errors.  Suppose you are trying
to compute a signal-to-noise ratio in decibels.  The formula
is $SNR_{db} = 10 \log10(P_{signal} / P_{noise})$.  In Python,
you might write something like this:

\beforeverb
\begin{verbatim}
import math
signal_power = 9
noise_power = 10
ratio = signal_power / noise_power
decibels = 10 * math.log10(ratio)
print decibels
\end{verbatim}
\afterverb
%
But when you run it, you get an error message:

\beforeverb
\begin{verbatim}
Traceback (most recent call last):
  File "snr.py", line 5, in ?
    decibels = 10 * math.log10(ratio)
OverflowError: math range error
\end{verbatim}
\afterverb
%
The error message indicates line 5, but there is nothing
wrong with that line.  To find the real error, it might be
useful to print the value of {\tt ratio}, which turns out to
be 0.  The problem is in line 4, because dividing two integers
does floor division.  The solution is to represent signal power
and noise power with floating-point values.

\index{floor division}
\index{division!floor}

And that brings me to the Fourth Theorem of Debugging:

\begin{quote}
Error messages tell you where the problem was discovered, 
but that is often not where it was caused.
\end{quote}


\section{Glossary}

\begin{description}

\item[modulus operator:]  An operator, denoted with a percent sign
({\tt \%}), that works on integers and yields the remainder when one
number is divided by another.
\index{modulus operator}
\index{operator!modulus}

\item[boolean expression:]  An expression whose value is either 
{\tt True} or {\tt False}.
\index{boolean expression}
\index{expression!boolean}

\item[comparison operator:] One of the operators that compares
its operands: {\tt ==}, {\tt !=}, {\tt >}, {\tt <}, {\tt >=}, and {\tt <=}.

\item[logical operator:] One of the operators that combines boolean
expressions: {\tt and}, {\tt or}, and {\tt not}.

\item[conditional statement:]  A statement that controls the flow of
execution depending on some condition.
\index{conditional statement}
\index{statement!conditional}

\item[condition:] The boolean expression in a conditional statement
that determines which branch is executed.
\index{condition}

\item[compound statement:]  A statement that consists of a header
and a body.  The header ends with a colon (:).  The body is indented
relative to the header.
\index{compound statement}

\item[body:] The sequence of statements within a compound statement.
\index{body}

\item[branch:] One of the alternative sequences of statements in
a conditional statement.
\index{branch}

\item[chained conditional:]  A conditional statement with a series
of alternative branches.
\index{chained conditional}
\index{conditional!chained}

\item[recursion:]  The process of calling the function that is
currently executing.
\index{recursion}

\item[base case:]  A conditional branch in a
recursive function that does not make a recursive call.
\index{base case}

\item[infinite recursion:]  A function that calls itself recursively
without ever reaching the base case.  Eventually, an infinite recursion
causes a runtime error.
\index{infinite recursion}

\end{description}

\section{Exercises}

The following exercises use TurtleWorld from Chapter~\ref{turtlechap}:

\begin{ex}

Read the following function and see if you can figure out
what it does.  Then run it (see the examples in Chapter~\ref{turtlechap}).

\beforeverb
\begin{verbatim}
def draw(t, length, n):
    if n == 0:
        return
    angle = 50
    fd(t, length*n)
    lt(t, angle)
    draw(t, length, n-1)
    rt(t, 2*angle)
    draw(t, length, n-1)
    lt(t, angle)
    bk(t, length*n)
\end{verbatim}
\afterverb
\end{ex}

\begin{ex}

The Koch curve is a fractal that looks something like
this:

\beforefig
\centerline{\includegraphics[height=1in]{figs/koch.eps}}
\afterfig

To draw a Koch curve with length $x$, all you have to do is

\begin{enumerate}

\item Draw a Koch curve with length $x/3$.

\item Turn left 60 degrees.

\item Draw a Koch curve with length $x/3$.

\item Turn right 120 degrees.

\item Draw a Koch curve with length $x/3$.

\item Turn left 60 degrees.

\item Draw a Koch curve with length $x/3$.

\end{enumerate}

The only exception is if $x$ is less than 2.  In that case,
you can just draw a straight line with length $x$.

Write a function called {\tt koch} that takes a turtle and
a length as parameters, and that uses the turtle to draw a Koch
curve with the given length.

Then write a function called {\tt snowflake} that draws three
Koch curves to make the outline of a snowflake.

\end{ex}


\chapter{Fruitful functions}
\label{fruitchap}

\section{Return values}
\index{return value}

Some of the built-in functions we have used, such as the math
functions, produce results.  Calling the function generates a
value, which we usually assign to a variable or use as part of an
expression.

\beforeverb
\begin{verbatim}
e = math.exp(1.0)
height = radius * math.sin(radians)
\end{verbatim}
\afterverb
%
All of the functions we have written so far are void; they print
something or move turtles around, but their return value is {\tt
None}.

In this chapter, we are (finally) going to write fruitful functions.
The first example is {\tt area}, which returns the area of a circle
with the given radius:

\beforeverb
\begin{verbatim}
def area(radius):
    temp = math.pi * radius**2
    return temp
\end{verbatim}
\afterverb
%
We have seen the {\tt return} statement before, but in a fruitful
function the {\tt return} statement includes
a return value.  This statement means: ``Return immediately from
this function and use the following expression as a return value.''
The expression provided can be arbitrarily complicated, so we could
have written this function more concisely:

\beforeverb
\begin{verbatim}
def area(radius):
    return math.pi * radius**2
\end{verbatim}
\afterverb
%
On the other hand, {\bf temporary variables} like {\tt temp} often make
debugging easier.

\index{temporary variable}
\index{variable!temporary}

Sometimes it is useful to have multiple return statements, one in each
branch of a conditional:

\beforeverb
\begin{verbatim}
def absolute_value(x):
    if x < 0:
        return -x
    else:
        return x
\end{verbatim}
\afterverb
%
Since these {\tt return} statements are in an alternative conditional,
only one will be executed.

As soon as a return statement executes, the function
terminates without executing any subsequent statements.
Code that appears after a {\tt return} statement, or any other place
the flow of execution can never reach, is called {\bf dead code}.

\index{dead code}

In a fruitful function, it is a good idea to ensure
that every possible path through the program hits a
{\tt return} statement.  For example:

\beforeverb
\begin{verbatim}
def absolute_value(x):
    if x < 0:
        return -x
    elif x > 0:
        return x
\end{verbatim}
\afterverb
%
This program is not correct because if {\tt x} happens to be 0,
neither condition is true, and the function ends without hitting a
{\tt return} statement.  If the flow of execution gets to the end
of a function, the return value is {\tt None}, which is not
the absolute value of 0.

\index{None}

\beforeverb
\begin{verbatim}
>>> print absolute_value(0)
None
\end{verbatim}
\afterverb
%

\begin{ex}
Write a {\tt compare} function
that returns {\tt 1} if {\tt x > y},
{\tt 0} if {\tt x == y}, and {\tt -1} if {\tt x < y}.
\end{ex}


\section{Incremental development}
\label{incremental development}
\index{development plan!incremental}

As you write larger functions, you might start find yourself
spending more time debugging.

To deal with increasingly complex programs,
you might want to try a process called
{\bf incremental development}.  The goal of incremental development
is to avoid long debugging sessions by adding and testing only
a small amount of code at a time.

As an example, suppose you want to find the distance between two
points, given by the coordinates $(x_1, y_1)$ and $(x_2, y_2)$.
By the Pythagorean theorem, the distance is:

\begin{displaymath}
\mathrm{distance} = \sqrt{(x_2 - x_1)^2 + (y_2 - y_1)^2}
\end{displaymath}
%
The first step is to consider what a {\tt distance} function should
look like in Python. In other words, what are the inputs (parameters)
and what is the output (return value)?

In this case, the two points are the inputs, which you can represent
using four parameters.  The return value is the distance, which is
a floating-point value.

Already you can write an outline of the function:

\beforeverb
\begin{verbatim}
def distance(x1, y1, x2, y2):
    return 0.0
\end{verbatim}
\afterverb
%
Obviously, this version doesn't compute distances; it always returns
zero.  But it is syntactically correct, and it runs, which means that
you can test it before you make it more complicated.

To test the new function, call it with sample arguments:

\beforeverb
\begin{verbatim}
>>> distance(1, 2, 4, 6)
0.0
\end{verbatim}
\afterverb
%
I chose these values so that the horizontal distance is 3 and the
vertical distance is 4; that way, the result is 5
(the hypotenuse of a 3-4-5 triangle). When testing a function, it is
useful to know the right answer.

At this point we have confirmed that the function is syntactically
correct, and we can start adding code to the body.
A reasonable next step is to find the differences
$x_2 - x_1$ and $y_2 - y_1$.  The next version stores those values in
temporary variables and prints them.

\beforeverb
\begin{verbatim}
def distance(x1, y1, x2, y2):
    dx = x2 - x1
    dy = y2 - y1
    print 'dx is', dx
    print 'dy is', dy
    return 0.0
\end{verbatim}
\afterverb
%
If the function is working, it should display {\tt 'dx is 3'} and {\tt
'dy is 4'}.  If so, we know that the function is getting the right
arguments and performing the first computation correctly.  If not,
there are only a few lines to check.

Next we compute the sum of squares of {\tt dx} and {\tt dy}:

\beforeverb
\begin{verbatim}
def distance(x1, y1, x2, y2):
    dx = x2 - x1
    dy = y2 - y1
    dsquared = dx**2 + dy**2
    print 'dsquared is: ', dsquared
    return 0.0
\end{verbatim}
\afterverb
%
Again, you would run the program at this stage and check the output
(which should be 25).

Finally, you can use {\tt math.sqrt} to compute and return the result:

\beforeverb
\begin{verbatim}
def distance(x1, y1, x2, y2):
    dx = x2 - x1
    dy = y2 - y1
    dsquared = dx**2 + dy**2
    result = math.sqrt(dsquared)
    return result
\end{verbatim}
\afterverb
%
If that works correctly, you are done.  Otherwise, you might
want to print the value of {\tt result} before the return
statement.

The final version of the function doesn't display anything when it
runs; it only returns a value.  The {\tt print} statements we wrote
are useful for debugging, but once you get the function working, you
should remove them.  Code like that is called {\bf scaffolding}
because it is helpful for building the program but is not part of the
final product.

\index{scaffolding}

When you start out, you should add only a line or two of code at a
time.  As you gain more experience, you might find yourself writing
and debugging bigger chunks.  Either way, incremental development
can save you a lot of debugging time.

The key aspects of the process are:

\begin{enumerate}

\item Start with a working program and make small incremental changes. 
At any point, if there is an error, you should have a good idea
where it is.

\item Use temporary variables to hold intermediate values so you can
display and check them.

\item Once the program is working, you might want to remove some of
the scaffolding or consolidate multiple statements into compound
expressions, but only if it does not make the program difficult to
read.

\end{enumerate}

\begin{ex}
Use incremental development to write a function
called {\tt hypotenuse} that returns the length of the hypotenuse of a
right triangle given the lengths of the two legs as arguments.
Record each stage of the development process as you go.
\end{ex}


\section{Composition}
\index{composition}
\index{function!composition}

As you should expect by now, you can call one function from
within another.  This ability is called {\bf composition}.

As an example, we'll write a function that takes two points,
the center of the circle and a point on the perimeter, and computes
the area of the circle.

Assume that the center point is stored in the variables {\tt xc} and
{\tt yc}, and the perimeter point is in {\tt xp} and {\tt yp}. The
first step is to find the radius of the circle, which is the distance
between the two points.  Fortunately, there is a function, {\tt
distance}, that does that:

\beforeverb
\begin{verbatim}
radius = distance(xc, yc, xp, yp)
\end{verbatim}
\afterverb
%
The next step is to find the area of a circle with that radius:

\beforeverb
\begin{verbatim}
result = area(radius)
\end{verbatim}
\afterverb
%
Wrapping that up in a function, we get:

\beforeverb
\begin{verbatim}
def circle_area(xc, yc, xp, yp):
    radius = distance(xc, yc, xp, yp)
    result = area(radius)
    return result
\end{verbatim}
\afterverb
%
The temporary variables {\tt radius} and {\tt result} are useful for
development and debugging, but once the program is working, we can
make it more concise by composing the function calls:

\beforeverb
\begin{verbatim}
def circle_area(xc, yc, xp, yp):
    return area(distance(xc, yc, xp, yp))
\end{verbatim}
\afterverb
%

\section{Boolean functions}
\label{boolean}
\index{boolean function}
\index{function!boolean}

Functions can return booleans, which is often convenient for hiding
complicated tests inside functions.  For example:

\beforeverb
\begin{verbatim}
def is_divisible(x, y):
    if x % y == 0:
        return True
    else:
        return False
\end{verbatim}
\afterverb
%
It is common to give boolean functions names that sound like yes/no
questions; {\tt is\_divisible} returns either {\tt True} or {\tt False}
to indicate whether {\tt x} is divisible by {\tt y}.

Here is an example:

\beforeverb
\begin{verbatim}
>>>   is_divisible(6, 4)
False
>>>   is_divisible(6, 3)
True
\end{verbatim}
\afterverb
%
The result of the {\tt ==} operator is a boolean, so we can write the
function more concisely by returning it directly:

\beforeverb
\begin{verbatim}
def is_divisible(x, y):
    return x % y == 0
\end{verbatim}
\afterverb
%
Boolean functions are often used in conditional statements:

\beforeverb
\begin{verbatim}
if is_divisible(x, y):
    print 'x is divisible by y'
\end{verbatim}
\afterverb
%
It might be tempting to write something like:

\beforeverb
\begin{verbatim}
if is_divisible(x, y) == True:
    print 'x is divisible by y'
\end{verbatim}
\afterverb
%
But the extra comparison is unnecessary.

\begin{ex}
Write a function {\tt is\_between(x, y, z)} that
returns {\tt True} if $x \le y \le z$ or {\tt False} otherwise.
\end{ex}


\section{More recursion}
\index{recursion}
\index{complete language}
\index{language!complete}
\index{Turing, Alan}
\index{Turing Thesis}

We have only covered a small subset of Python, but you might
be interested to know that this subset is a {\em complete}
programming language, which means that anything that can be
computed can be expressed in this language.  Any program ever written
could be rewritten using only the language features you have learned
so far (actually, you would need a few commands to control devices
like the keyboard, mouse, disks, etc., but that's all).

Proving that claim is a nontrivial exercise first accomplished by Alan
Turing, one of the first computer scientists (some would argue that he
was a mathematician, but a lot of early computer scientists started as
mathematicians).  Accordingly, it is known as the Turing Thesis.  If
you take a course on the Theory of Computation, you will have a chance
to see the proof.

To give you an idea of what you can do with the tools you have learned
so far, we'll evaluate a few recursively defined mathematical
functions.  A recursive definition is similar to a circular
definition, in the sense that the definition contains a reference to
the thing being defined.  A truly circular definition is not very
useful:

\begin{description}

\item[frabjuous:] An adjective used to describe something that is frabjuous.

\end{description}

\index{frabjuous}
\index{circular definition}
\index{definition!circular}

If you saw that definition in the dictionary, you might be annoyed. On
the other hand, if you looked up the definition of the factorial
function, denoted with the symbol $!$, you might get something like
this:

\vspace{-0.35in}
\begin{eqnarray*}
&&  0! = 1 \\
&&  n! = n (n-1)!
\end{eqnarray*}
\vspace{-0.25in}

This definition says that the factorial of 0 is 1, and the factorial
of any other value, $n$, is $n$ multiplied by the factorial of $n-1$.

So $3!$ is 3 times $2!$, which is 2 times $1!$, which is 1 times
$0!$. Putting it all together, $3!$ equals 3 times 2 times 1 times 1,
which is 6.

\index{factorial function}
\index{function!factorial}

If you can write a recursive definition of something, you can usually
write a Python program to evaluate it. The first step is to decide
what the parameters should be.  In this case it should be clear
that {\tt factorial} has a single parameter:

\beforeverb
\begin{verbatim}
def factorial(n):
\end{verbatim}
\afterverb
%
If the argument happens to be 0, all we have to do is return 1:

\beforeverb
\begin{verbatim}
def factorial(n):
    if n == 0:
        return 1
\end{verbatim}
\afterverb
%
Otherwise, and this is the interesting part, we have to make a
recursive call to find the factorial of $n-1$ and then multiply it by
$n$:

\beforeverb
\begin{verbatim}
def factorial(n):
    if n == 0:
        return 1
    else:
        recurse = factorial(n-1)
        result = n * recurse
        return result
\end{verbatim}
\afterverb
%
The flow of execution for this program is similar to the flow of {\tt
countdown} in Section~\ref{recursion}.  If we call {\tt factorial}
with the value 3:

Since 3 is not 0, we take the second branch and calculate the factorial
of {\tt n-1}...

\begin{quote}
Since 2 is not 0, we take the second branch and calculate the factorial of
{\tt n-1}...


  \begin{quote}
  Since 1 is not 0, we take the second branch and calculate the factorial
  of {\tt n-1}...


    \begin{quote}
    Since 0 {\em is} 0, we take the first branch and return 1
    without making any more recursive calls.
    \end{quote}


  The return value (1) is multiplied by $n$, which is 1, and the
  result is returned.
  \end{quote}


The return value (1) is multiplied by $n$, which is 2, and the
result is returned.
\end{quote}


The return value (2) is multiplied by $n$, which is 3, and the result, 6,
becomes the return value of the function call that started the whole
process.

Here is what the stack diagram looks like for this sequence of function
calls:

\vspace{0.1in}
\beforefig
\centerline{\includegraphics{figs/stack3.eps}}
\afterfig
\vspace{0.1in}

The return values are shown being passed back up the stack.  In each
frame, the return value is the value of {\tt result}, which is the
product of {\tt n} and {\tt recurse}.

In the last frame, the local
variables {\tt recurse} and {\tt result} do not exist, because
the branch that creates them did not execute.



\section{Leap of faith}
\index{recursion}
\index{leap of faith}

Following the flow of execution is one way to read programs, but
it can quickly become labyrinthine.  An
alternative is what I call the ``leap of faith.''  When you come to a
function call, instead of following the flow of execution, you {\em
assume} that the function works correctly and returns the right
result.

In fact, you are already practicing this leap of faith when you use
built-in functions.  When you call {\tt math.cos} or {\tt math.exp},
you don't examine the bodies of those functions.  You just
assume that they work because the people who wrote the built-in
functions were good programmers.

The same is true when you call one of your own functions.  For
example, in Section~\ref{boolean}, we wrote a function called {\tt
is\_divisible} that determines whether one number is divisible by
another.  Once we have convinced ourselves that this function is
correct---examining the code and testing---we can use the function
without looking at the code again.

The same is true of recursive programs.  When you get to the recursive
call, instead of following the flow of execution, you should assume
that the recursive call works (yields the correct result) and then ask
yourself, ``Assuming that I can find the factorial of $n-1$, can I
compute the factorial of $n$?''  In this case, it is clear that you
can, by multiplying by $n$.

Of course, it's a bit strange to assume that the function works
correctly when you haven't finished writing it, but that's why
it's called a leap of faith!


\section{One more example}
\label{one more example}

\index{Fibonacci function}

After {\tt factorial}, the most common example of a recursively
defined mathematical function is {\tt fibonacci}, which has the
following definition:

\vspace{-0.25in}
\begin{eqnarray*}
&& \mathrm{fibonacci}(0) = 0 \\
&& \mathrm{fibonacci}(1) = 1 \\
&& \mathrm{fibonacci}(n) = \mathrm{fibonacci}(n-1) + \mathrm{fibonacci}(n-2);
\end{eqnarray*}
%
Translated into Python, it looks like this:

\beforeverb
\begin{verbatim}
def fibonacci (n):
    if n == 0:
        return 0
    elif  n == 1:
        return 1
    else:
        return fibonacci(n-1) + fibonacci(n-2)
\end{verbatim}
\afterverb
%
If you try to follow the flow of execution here, even for fairly
small values of $n$, your head explodes.  But according to the
leap of faith, if you assume that the two recursive calls
work correctly, then it is clear that you get
the right result by adding them together.


\section{Checking types}
\index{type checking}
\index{error checking}
\index{factorial function}

What happens if we call {\tt factorial} and give it 1.5 as an argument?

\index{RuntimeError}

\beforeverb
\begin{verbatim}
>>> factorial(1.5)
RuntimeError: Maximum recursion depth exceeded
\end{verbatim}
\afterverb
%
It looks like an infinite recursion.  But how can that be?  There is a
base case---when {\tt n == 0}.  But if {\tt n} is not an integer,
we can {\em miss} the base case and recurse forever.

\index{infinite recursion}
\index{recursion!infinite}

In the first recursive call, the value of {\tt n} is 0.5.
In the next, it is -0.5.  From there, it gets smaller and
smaller, but it will never be 0.

We have two choices.  We can try to generalize the {\tt factorial}
function to work with floating-point numbers, or we can make
{\tt factorial} check the type of its argument.  The first option
is called the gamma function and it's a little beyond the
scope of this book.  So we'll go for the
second.

\index{gamma function}

We can use the built-in function {\tt isinstance} to verify the type of the
argument.  While we're
at it, we can also make sure the argument is positive:

\index{{\tt isinstance}}

\beforeverb
\begin{verbatim}
def factorial (n):
    if not isinstance(n, int):
        print 'Factorial is only defined for integers.'
        return None
    elif n < 0:
        print 'Factorial is only defined for positive integers.'
        return None
    elif n == 0:
        return 1
    else:
        return n * factorial(n-1)
\end{verbatim}
\afterverb
%
Now we have three base cases.  The first catches nonintegers and the
second catches negative integers.  In both cases, the program prints
an error message and returns {\tt None} to indicate that something
went wrong:

\beforeverb
\begin{verbatim}
>>> factorial('fred')
Factorial is only defined for integers.
None
>>> factorial(-2)
Factorial is only defined for positive integers.
None
\end{verbatim}
\afterverb
%
If we get past both checks, then we know that $n$ is a positive
integer, and we can prove that the recursion terminates.

This program demonstrates a pattern sometimes called a {\bf guardian}.
The first two conditionals act as guardians, protecting the code that
follows from values that might cause an error.  The guardians make it
possible to prove the correctness of the code.


\section{Debugging}

Breaking a large program into smaller functions creates natural
checkpoints for debugging.  If a function is not working, there are
three possibilities to consider:

\begin{itemize}

\item There is something wrong with the arguments the function
is getting.

\item There is something wrong with the function.

\item There is something wrong with the return value or the
way it is being used.

\end{itemize}

To rule out the first possibility, you can add a {\tt print} statement
at the beginning of the function and display the values of the
parameters (and maybe their types).

If the parameters look good, add a {\tt print} statement before each
{\tt return} statement that displays the return value.  If
possible, check the result by hand.  If necessary, call the
function with special values where you know what the result should
be (as in Section~\ref{incremental development}).

If the function seems to be working, look at the function call
to make sure the return value is being used correctly (or used
at all!).

Adding print statements at the beginning and end of a function
can help make the flow of execution more visible.
For example, here is a version of {\tt factorial} with
print statements:

\beforeverb
\begin{verbatim}
def factorial(n):
    space = ' ' * (4 * n)
    print space, 'factorial', n
    if n == 0:
        print space, 'returning 1'
        return 1
    else:
        recurse = factorial(n-1)
        result = n * recurse
        print space, 'returning', result
        return result
\end{verbatim}
\afterverb
%
{\tt space} is a string of space characters that controls the
indentation of the output.  Here is the result of {\tt factorial(5)} :

\beforeverb
\begin{verbatim}
                     factorial 5
                 factorial 4
             factorial 3
         factorial 2
     factorial 1
 factorial 0
 returning 1
     returning 1
         returning 2
             returning 6
                 returning 24
                     returning 120
\end{verbatim}
\afterverb
%
If you are confused about the flow of execution, this kind of
output can be helpful.  It takes some time to develop effective
scaffolding, but according to the Fifth Theorem of Debugging:

\begin{quote}
A little bit of scaffolding can save a lot of debugging.
\end{quote}


\section{Glossary}

\begin{description}

\item[temporary variable:]  A variable used to store an intermediate value in
a complex calculation.
\index{temporary variable}
\index{variable!temporary}

\item[dead code:]  Part of a program that can never be executed, often because
it appears after a {\tt return} statement.
\index{dead code}

\item[{\tt None}:]  A special value returned by functions that
have no return statement or a return statement without an argument.
\index{None}

\item[incremental development:]  A program development plan intended to
avoid debugging by adding and testing only
a small amount of code at a time.
\index{incremental development}

\item[scaffolding:]  Code that is used during program development but is
not part of the final version.
\index{scaffolding}

\item[guardian:]  A programming pattern that uses a conditional
statement to check for and handle circumstances that
might cause an error.
\index{guardian}

\end{description}


\section{Exercises}

\begin{ex}
Draw a stack diagram for the following
program.  What does the program print?

\begin{verbatim}
def b(z):
    prod = a(z, z)
    print z, prod
    return prod

def a(x, y):
    x = x + 1
    return x * y

def c(x, y, z):
    sum = x + y + z
    pow = b(sum)**2
    return pow

x = 1
y = x + 1
print c(x, y+3, x+y)
\end{verbatim}

\end{ex}

\begin{ex}

\end{ex}


\chapter{Iteration}
\index{iteration}


\section{Multiple assignment}
\index{assignment}
\index{statement!assignment}
\index{multiple assignment}

As you may have discovered, it is legal to
make more than one assignment to the same variable.  A
new assignment makes an existing variable refer to a new
value (and stop referring to the old value).

\beforeverb
\begin{verbatim}
bruce = 5
print bruce,
bruce = 7
print bruce
\end{verbatim}
\afterverb
%
The output of this program is {\tt 5 7}, because the first time
{\tt bruce} is printed, its value is 5, and the second time, its
value is 7.  The
comma at the end of the first {\tt print} statement suppresses
the newline, which is why both outputs
appear on the same line.

\index{newline}

Here is what {\bf multiple assignment} looks like in a state diagram:

\index{state diagram}
\index{diagram!state}

\beforefig
\centerline{\includegraphics{figs/assign2.eps}}
\afterfig

With multiple assignment it is especially important to distinguish
between an assignment operation and a statement of equality.  Because
Python uses the equal sign ({\tt =}) for assignment, it is tempting to
interpret a statement like {\tt a = b} as a statement of equality. It
is not!

First, equality is a symmetric relation and assignment is not.  For
example, in mathematics, if $a = 7$ then $7 = a$.  But in Python, the
statement {\tt a = 7} is legal and {\tt 7 = a} is not.

Furthermore, in mathematics, a statement of equality is either true or
false, for all time.  If $a = b$ now, then $a$ will always equal $b$.
In Python, an assignment statement can make two variables equal, but
they don't have to stay that way:

\beforeverb
\begin{verbatim}
a = 5
b = a    # a and b are now equal
a = 3    # a and b are no longer equal
\end{verbatim}
\afterverb
%
The third line changes the value of {\tt a} but does not change the
value of {\tt b}, so they are no longer equal. 

Although multiple assignment is frequently helpful, you should use it
with caution.  If the values of variables change frequently, it can
make the code difficult to read and debug.


\section{Updating variables}
\index{update}
\label{update}

One of the most common forms of multiple assignment is an {\bf update},
where the new value of the variable depends on the old.

\beforeverb
\begin{verbatim}
x = x+1
\end{verbatim}
\afterverb
%
This means ``get the current value of {\tt x}, add one, and then
update {\tt x} with the new value.''

If you try to update a variable that doesn't exist, you get an
error, because Python evaluates the right side before it assigns
a value to {\tt x}:

\beforeverb
\begin{verbatim}
>>> x = x+1
NameError: name 'x' is not defined
\end{verbatim}
\afterverb
%
Before you can update a variable, you have to {\bf initialize}
it, usually with a simple assignment:

\index{initialize}

\beforeverb
\begin{verbatim}
>>> x = 0
>>> x = x+1
\end{verbatim}
\afterverb
%
Updating a variable by adding 1 is called an {\bf increment};
subtracting 1 is called a {\bf decrement}.

\index{increment}
\index{decrement}




\section{The {\tt while} statement}
\index{while statement}
\index{statement!while}
\index{loop!while}
\index{iteration}

Computers are often used to automate repetitive tasks.  Repeating
identical or similar tasks without making errors is something that
computers do well and people do poorly.

We have seen two programs, {\tt print\_n} and {\tt countdown}, that
use recursion to perform repetition, which is also called {\bf
iteration}.  Because iteration is so common, Python provides several
language features to make it easier.  One is the {\tt for} statement
we saw in Section~\ref{repetition}.  We'll get back to that later.

Another is the {\tt while} statement.  Here is a version of {\tt
countdown} that uses a {\tt while} statement:

\beforeverb
\begin{verbatim}
def countdown(n):
    while n > 0:
        print n
        n = n-1
    print 'Blastoff!'
\end{verbatim}
\afterverb
%
You can almost read the {\tt while} statement as if it were English.
It means, ``While {\tt n} is greater than 0,
display the value of {\tt n} and then reduce the value of
{\tt n} by 1.  When you get to 0, display the word {\tt Blastoff!}''

More formally, here is the flow of execution for a {\tt while} statement:

\begin{enumerate}

\item Evaluate the condition, yielding {\tt True} or {\tt False}.

\item If the condition is false, exit the {\tt while} statement
and continue execution at the next statement.

\item If the condition is true, execute the
body and then go back to step 1.

\end{enumerate}

This type of flow is called a {\bf loop} because the third step
loops back around to the top.  

\index{condition}
\index{loop}
\index{loop!body}
\index{body!loop}
\index{infinite loop}
\index{loop!infinite}

The body of the loop should change the value of one or more variables
so that eventually the condition becomes false and the loop
terminates.  Otherwise the loop will repeat forever, which is called
an {\bf infinite loop}.  An endless source of amusement for computer
scientists is the observation that the directions on shampoo,
``Lather, rinse, repeat,'' are an infinite loop.

In the case of {\tt countdown}, we can prove that the loop
terminates because we know that the value of {\tt n} is finite, and we
can see that the value of {\tt n} gets smaller each time through the
loop, so eventually we have to get to 0.  In other
cases, it is not so easy to tell:

\beforeverb
\begin{verbatim}
def sequence(n):
    while n != 1:
        print n,
        if n%2 == 0:        # n is even
            n = n/2
        else:               # n is odd
            n = n*3+1
\end{verbatim}
\afterverb
%
The condition for this loop is {\tt n != 1}, so the loop will continue
until {\tt n} is {\tt 1}, which makes the condition false.

Each time through the loop, the program outputs the value of {\tt n}
and then checks whether it is even or odd.  If it is even, {\tt n} is 
divided by 2.  If it is odd, the value of {\tt n} is replaced with
{\tt n*3+1}. For example, if the argument passed
to {\tt sequence} is 3, the resulting sequence is 3, 10, 5, 16, 8, 4, 2, 1.

Since {\tt n} sometimes increases and sometimes decreases, there is no
obvious proof that {\tt n} will ever reach 1, or that the program
terminates.  For some particular values of {\tt n}, we can prove
termination.  For example, if the starting value is a power of two,
then the value of {\tt n} will be even each time through the loop
until it reaches 1. The previous example ends with such a sequence,
starting with 16.

The hard question is whether we can
prove that this program terminates for {\em all positive values} of {\tt n}.
So far, no one has been able to prove it {\em or} disprove it!

\begin{ex}
Rewrite the function {\tt print\_n} from
Section~\ref{recursion} using iteration instead of recursion.
\end{ex}


\section{{\tt break}}

Sometimes you don't know it's time to end a loop until you get half
way through the body.  In that case you can use the {\tt break}
statement to jump out of the loop.

For example, suppose you want to take input from the user until they
type {\tt done}.  You could write:

\beforeverb
\begin{verbatim}
while True:
    line = raw_input('> ')
    if line == 'done':
        break
    print line

print 'Done!'
\end{verbatim}
\afterverb
%
The loop condition is {\tt True}, which is always true, so the
loop runs until it hits the break statement.

Each time through, it prompts the user with an angle bracket.
If the user types {\tt done}, the {\tt break} statement exits
the loop.  Otherwise the program echos whatever the user types
and goes back to the top of the loop.  Here's a sample run:

\beforeverb
\begin{verbatim}
> not done
not done
> done
Done!
\end{verbatim}
\afterverb
%
This way of writing {\tt while} loops is common because you
can check the condition anywhere in the loop (not just at the
top) and you can express the stop condition affirmatively
(``stop when this happens'') rather than negatively (``keep going
until that happens.'').


\section{Square roots}

Loops are often used in programs that compute
numerical results by starting with an approximate answer and
iteratively improving it.

For example, one way of computing square roots is Newton's method.
Suppose that you want to know the square root of $a$.  If you start
with almost any estimate, $x$, you can compute a better
estimate with the following formula:

\[ y = \frac{x + a/x}{2} \]
%
For example, if $a$ is 4 and $x$ is 3:

\beforeverb
\begin{verbatim}
>>> a = 4.0
>>> x = 3.0
>>> y = (x + a/x) / 2
>>> print y
2.16666666667
\end{verbatim}
\afterverb
%
Which is closer to the correct answer ($\sqrt{4} = 2$).  If we
repeat the process with the new estimate, it gets even closer:

\beforeverb
\begin{verbatim}
>>> x = y
>>> y = (x + a/x) / 2
>>> print y
2.00641025641
\end{verbatim}
\afterverb
%
After a few more updates, the estimate is almost exact:

\beforeverb
\begin{verbatim}
>>> x = y
>>> y = (x + a/x) / 2
>>> print y
2.00001024003
>>> x = y
>>> x = (x + a/x) / 2
>>> print y
2.00000000003
\end{verbatim}
\afterverb
%
In general we don't know ahead of time how many steps it takes
to get to the right answer, but we know when we get there
because the estimate
stops changing:

\beforeverb
\begin{verbatim}
>>> x = y
>>> y = (x + a/x) / 2
>>> print y
2.0
>>> x = y
>>> y = (x + a/x) / 2
>>> print y
2.0
\end{verbatim}
\afterverb
%
When {\tt y == x}, we can stop.  Here is a loop that starts
with an initial estimate, {\tt x}, and improves it until it
stops changing:

\beforeverb
\begin{verbatim}
while True:
    print x
    y = (x + a/x) / 2
    if y == x:
        break
    x = y
\end{verbatim}
\afterverb
%
For most values of {\tt a} this works fine, but in general it is
dangerous to test {\tt float} equality.
Floating-point values are only approximately right:
most rational numbers, like $1/3$, and irrational numbers, like
$\sqrt{2}$, can't be represented exactly with a {\tt float}.

Rather than checking whether {\tt x} and {\tt y} are exactly equal, it
is safer to use {\tt math.fabs} to compute the absolute value, or
magnitude, of the difference between them:

\beforeverb
\begin{verbatim}
    if math.fabs(y-x) < something_small:
        break
\end{verbatim}
\afterverb
%
Where {\tt something\_small} has a value like {\tt 0.0000001} that
determines how close is close enough.

\begin{ex}
\label{square_root}
Wrap this loop in a function called {\tt square\_root}
that takes {\tt a} as a parameter, chooses a reasonable
value of {\tt x}, and returns an estimate of the square root
of {\tt a}.
\end{ex}


\section{Algorithms}
\index{algorithm}

Newton's method is an example of an {\bf algorithm}: it is a
mechanical process for solving a category of problems (in this
case, computing square roots).

It is not easy to define an algorithm.  It might help to start
with something that is not an algorithm.  When you learned
to multiply single-digit numbers, you probably memorized the
multiplication table.  In effect, you memorized 100 specific solutions.
That kind of knowledge is not algorithmic.

But if you were ``lazy,'' you probably cheated by learning a few
tricks.  For example, to find the product of $n$ and 9, you can
write $n-1$ as the first digit and $10-n$ as the second
digit.  This trick is a general solution for multiplying any
single-digit number by 9.  That's an algorithm!

Similarly, the techniques you learned for addition with carrying,
subtraction with borrowing, and long division are all algorithms.  One
of the characteristics of algorithms is that they do not require any
intelligence to carry out.  They are mechanical processes in which
each step follows from the last according to a simple set of rules.

In my opinion, it is embarrassing that humans spend so much time in
school learning to execute algorithms that, quite literally, require
no intelligence.

On the other hand, the process of designing algorithms is interesting,
intellectually challenging, and a central part of what we call
programming.

Some of the things that people do naturally, without difficulty or
conscious thought, are the hardest to express algorithmically.
Understanding natural language is a good example.  We all do it, but
so far no one has been able to explain {\em how} we do it, at least
not in the form of an algorithm.


\section{Debugging}

As you start writing bigger programs, you might find yourself
spending more time debugging.  More code means more chances to
make an error and more place for bugs to hide.

One way to cut your debugging time is ``debugging by bisection.''
For example, if there are 100 lines in your program and you
check them one at a time, it would take 100 steps.

Instead, try to break the problem in half.  Look at the middle
of the program, or near it, for an intermediate value you
can check.  Add a {\tt print} statement (or something else
that has a verifiable effect) and run the program.

If the mid-point check is incorrect, the problem must be in the
first half of the program.  If it is correct, the problem is
in the second half.

Every time you perform a check like this, you halve the number
of lines you have to search.  After six steps (which is much
less than 100), you would be down to one or two lines of code.

At least in theory.  In practice it is not always clear what
the ``middle of the program'' is and not always possible to
check it.  It doesn't make sense to count lines and find the
exact midpoint.  Instead, think about places
in the program where there might be errors and places where it
is easy to put a check.  Then choose a spot where you
think the chances are about the same that the bug is before
or after the check.




\section{Glossary}

\begin{description}

\item[multiple assignment:] Making more than one assignment to the same
variable during the execution of a program.
\index{multiple assignment}
\index{assignment!multiple }

\item[update:] An assignment where the new value of the variable
depends on the old.
\index{update}

\item[initialize:] An assignment that gives an initial value to
a variable that will be updated.

\item[increment:] An update that increases the value of a variable
(often by one).
\index{increment}

\item[decrement:] An update that decreases the value of a variable.
\index{decrement}

\item[iteration:] Repeated execution of a set of statements using
either a recursive function call or a loop.
\index{iteration}

\item[infinite loop:] A loop in which the terminating condition is
never satisfied.
\index{infinite loop}

\end{description}


\section{Exercises}

\begin{ex}
To test the square root algorithm in this chapter, you could compare
it with {\tt math.sqrt}.  Write a function named {\tt
test\_square\_root} that prints a table like this:

\beforeverb
\begin{verbatim}
1.0 1.0           1.0           0.0
2.0 1.41421356237 1.41421356237 2.22044604925e-16
3.0 1.73205080757 1.73205080757 0.0
4.0 2.0           2.0           0.0
5.0 2.2360679775  2.2360679775  0.0
6.0 2.44948974278 2.44948974278 0.0
7.0 2.64575131106 2.64575131106 0.0
8.0 2.82842712475 2.82842712475 4.4408920985e-16
9.0 3.0           3.0           0.0

\end{verbatim}
\afterverb
%
The first column is a number, $a$; the second column is
the square root of $a$ computed with the function from
Exercise~\ref{square_root}; the third column is the square root computed
by {\tt math.sqrt}; the fourth column is the absolute value
of the difference between the two estimates.
\end{ex}


\begin{ex}

The built-in function {\tt eval} takes a string and evaluates
it using the Python interpreter.  For example:


\beforeverb
\begin{verbatim}
>>> eval('1 + 2 * 3')
7
>>> import math
>>> eval('math.sqrt(5)')
2.2360679774997898
>>> eval('type(math.pi)')
<type 'float'>
\end{verbatim}
\afterverb
%
Write a function called {\tt eval\_loop} that iteratively
prompts the user, takes the resulting input and evaluates
it using {\tt eval}, and prints the result.

It should continue until the user enters {\tt 'done'}, and then
return the value of the last expression it evaluated.

\end{ex}



\chapter{Strings}
\label{strings}


\section{A string is a sequence}
\index{sequence}
\index{character}
\index{bracket operator}
\index{operator!bracket}

A string is a {\bf sequence} of characters.  
You can access the characters one at a time with the
bracket operator:

\beforeverb
\begin{verbatim}
>>> fruit = 'banana'
>>> letter = fruit[1]
\end{verbatim}
\afterverb
%
The second statement selects character number 1 from {\tt
fruit} and assigns it to {\tt letter}.  

The expression in brackets is called an {\bf index}.  
The index {\em indicates} which character in the sequence you
want (hence the name).

But you might not get what you expect:

\beforeverb
\begin{verbatim}
>>> print letter
a
\end{verbatim}
\afterverb
%
For most people, the first letter of {\tt 'banana'} is {\tt b}, not
{\tt a}.  But for computer scientists, the index is an offset from the
beginning of the string, and the offset of the first letter is zero.

\beforeverb
\begin{verbatim}
>>> letter = fruit[0]
>>> print letter
b
\end{verbatim}
\afterverb
%
So {\tt b} is the 0th letter (``zero-eth'') of {\tt 'banana'}, {\tt a}
is the 1th letter (``one-eth''), and {\tt n} is the 2th (``two-eth'')
letter.

You can use any expression, including variables and operators, as an
index, but the value of the index has to be an integer.  Otherwise you
get:

\index{index}
\index{exception!TypeError}
\index{TypeError}

\beforeverb
\begin{verbatim}
>>> letter = fruit[1.0]
TypeError: string indices must be integers
\end{verbatim}
\afterverb
%

\section{{\tt len}}
\index{{\tt len}}
\index{string!length}

{\tt len} is a built-in function that returns the number of characters
in a string:

\beforeverb
\begin{verbatim}
>>> fruit = 'banana'
>>> len(fruit)
6
\end{verbatim}
\afterverb
%
To get the last letter of a string, you might be tempted to try something
like this:

\beforeverb
\begin{verbatim}
>>> length = len(fruit)
>>> last = fruit[length]
IndexError: string index out of range
\end{verbatim}
\afterverb
%
The reason for the {\tt IndexError} is that there is no letter in {\tt
'banana'} with the index 6.  Since we started counting at zero, the
six letters are numbered 0 to 5.  To get the last character, you have
to subtract 1 from {\tt length}:

\index{exception!IndexError}
\index{IndexError}

\beforeverb
\begin{verbatim}
>>> last = fruit[length-1]
>>> print last
a
\end{verbatim}
\afterverb
%
Alternatively, you can use negative indices, which count backward from
the end of the string.  The expression {\tt fruit[-1]} yields the last
letter, {\tt fruit[-2]} yields the second to last, and so on.

\index{index!negative}


\section{Traversal with a {\tt for} loop}
\label{for}

\index{traversal}
\index{loop!traversal}
\index{for loop}
\index{loop!for loop}

A lot of computations involve processing a string one character at a
time.  Often they start at the beginning, select each character in
turn, do something to it, and continue until the end.  This pattern of
processing is called a {\bf traversal}.  One way to write a traversal
is with a {\tt while} statement:

\beforeverb
\begin{verbatim}
index = 0
while index < len(fruit):
    letter = fruit[index]
    print letter
    index = index + 1
\end{verbatim}
\afterverb
%
This loop traverses the string and displays each letter on a line by
itself.  The loop condition is {\tt index < len(fruit)}, so
when {\tt index} is equal to the length of the string, the
condition is false, and the body of the loop is not executed.  The
last character accessed is the one with the index {\tt len(fruit)-1},
which is the last character in the string.

\begin{ex}
Write a function that takes a string as an argument
and displays the letters backward, one per line.
\end{ex}

Another way to write a traversal is with a {\tt for} loop:

\beforeverb
\begin{verbatim}
for char in fruit:
    print char
\end{verbatim}
\afterverb
%
Each time through the loop, the next character in the string is assigned
to the variable {\tt char}.  The loop continues until no characters are
left.

\index{concatenation}
\index{abecedarian}
\index{McCloskey, Robert}
\index{{\em Make Way for Ducklings}}

The following example shows how to use concatenation (string addition)
and a {\tt for} loop to generate an abecedarian series (that is, in
alphabetical order).  In Robert McCloskey's book {\em Make
Way for Ducklings}, the names of the ducklings are Jack, Kack, Lack,
Mack, Nack, Ouack, Pack, and Quack.  This loop outputs these names in
order:

\beforeverb
\begin{verbatim}
prefixes = 'JKLMNOPQ'
suffix = 'ack'

for letter in prefixes:
    print letter + suffix
\end{verbatim}
\afterverb
%
The output is:

\beforeverb
\begin{verbatim}
Jack
Kack
Lack
Mack
Nack
Oack
Pack
Qack
\end{verbatim}
\afterverb
%
Of course, that's not quite right because ``Ouack'' and
``Quack'' are misspelled.

\begin{ex}
Modify the program to fix this error.
\end{ex}



\section{String slices}
\label{slice}
\index{slice}
\index{string!slice}

A segment of a string is called a {\bf slice}.  Selecting a slice is
similar to selecting a character:

\beforeverb
\begin{verbatim}
>>> s = 'Monty Python'
>>> print s[0:5]
Monty
>>> print s[6:13]
Python
\end{verbatim}
\afterverb
%
The operator {\tt [n:m]} returns the part of the string from the 
``n-eth'' character to the ``m-eth'' character, including the first but
excluding the last.  This behavior is counterintuitive, but might
help to imagine the indices pointing {\em between} the
characters, as in the following diagram:

\beforefig
\centerline{\includegraphics{figs/banana.eps}}
\afterfig

If you omit the first index (before the colon), the slice starts at the
beginning of the string.  If you omit the second index, the slice goes to the
end of the string.  Thus:

\beforeverb
\begin{verbatim}
>>> fruit = 'banana'
>>> fruit[:3]
'ban'
>>> fruit[3:]
'ana'
\end{verbatim}
\afterverb
%
If the first index is greater than or equal to the second the result
is an {\bf empty string}, represented by two quotation marks:

\beforeverb
\begin{verbatim}
>>> fruit = 'banana'
>>> fruit[3:3]
''
\end{verbatim}
\afterverb
%
An empty string contains no characters and has length 0, but other
than that, it is the same as any other string.

\begin{ex}
Given that {\tt fruit} is a string, what does
{\tt fruit[:]} mean?
\end{ex}


\section{Strings are immutable}
\index{mutable}
\index{immutable string}
\index{string!immutable}

It is tempting to use the {\tt []} operator on the left side of an
assignment, with the intention of changing a character in a string.
For example:

\index{TypeError}
\index{error!TypeError}

\beforeverb
\begin{verbatim}
>>> greeting = 'Hello, world!'
>>> greeting[0] = 'J'
TypeError: object does not support item assignment
\end{verbatim}
\afterverb
%
The ``object'' in this case is the string and the ``item'' is
the character you tried to assign.  For now, an {\bf object} is
the same thing as a value, but we will refine that definition
later.  An {\bf item} is one of the values in a sequence.

The reason for the error is that
strings are {\bf immutable}, which means you can't change an
existing string.  The best you can do is create a new string
that is a variation on the original:

\beforeverb
\begin{verbatim}
>>> greeting = 'Hello, world!'
>>> new_greeting = 'J' + greeting[1:]
>>> print new_greeting
Jello, world!
\end{verbatim}
\afterverb
%
This example concatenates a new first letter onto
a slice of {\tt greeting}.  It has no effect on
the original string.

\index{concatenation}


\section{A {\tt find} function}
\label{find}
\index{traversal}
\index{search}
\index{pattern}
\index{computational pattern}

What does the following function do?

\beforeverb
\begin{verbatim}
def find(word, letter):
    index = 0
    while index < len(word):
        if word[index] == letter:
            return index
        index = index + 1
    return -1
\end{verbatim}
\afterverb
%
In a sense, {\tt find} is the opposite of the {\tt []} operator.
Instead of taking an index and extracting the corresponding character,
it takes a character and finds the index where that character
appears.  If the character is not found, the function returns {\tt
-1}.

This is the first example we have seen of a {\tt return} statement
inside a loop.  If {\tt word[index] == letter}, the function breaks
out of the loop and returns immediately.

If the character doesn't appear in the string, the program
exits the loop normally and  returns {\tt -1}.

This pattern of computation---traversing a sequence and returning
when we find what we are looking for---is a called a {\bf search}.

\begin{ex}
Modify {\tt find} so that it has a
third parameter, the index in {\tt word} where it should start
looking.
\end{ex}


\section{Looping and counting}
\label{counter}
\index{counter}
\index{pattern}

The following program counts the number of times the letter {\tt a}
appears in a string:

\beforeverb
\begin{verbatim}
word = 'banana'
count = 0
for letter in word:
    if letter == 'a':
        count = count + 1
print count
\end{verbatim}
\afterverb
%
This program demonstrates another pattern of computation called a {\bf
counter}.  The variable {\tt count} is initialized to 0 and then
incremented each time an {\tt a} is found.
When the loop exits, {\tt count}
contains the result---the total number of {\tt a}'s.

\begin{ex}
Encapsulate this code in a function named {\tt
count}, and generalize it so that it accepts the string and the
letter as arguments.
\end{ex}

\begin{ex}
Rewrite this function so that instead of
traversing the string, it uses the three-parameter version of {\tt
find} from the previous section.
\end{ex}


\section{{\tt string} methods}
\index{method}
\index{string method}

A {\bf method} is similar to a function---it takes arguments and
returns a value---but the syntax is different.  For example, the
method {\tt upper} takes a string and returns a new string with
all uppercase letters:

Instead of the function syntax {\tt upper(word)}, it uses
the method syntax {\tt word.upper()}.

\index{dot notation}

\beforeverb
\begin{verbatim}
>>> word = 'banana'
>>> new_word = word.upper()
>>> print new_word
BANANA
\end{verbatim}
\afterverb
%
This form of dot notation specifies the name of the method, {\tt
upper}, and the name of the string to apply the method to, {\tt
word}.  The parentheses indicate that this method has no
parameters.

A method call is called an {\bf invocation}; in this case, we would
say that we are invoking {\tt upper} on the {\tt word}.

\index{invocation}
\index{invoke}

As it turns out, there is a string method named {\tt find} that
is remarkably similar to the function we wrote:

\beforeverb
\begin{verbatim}
>>> word = 'banana'
>>> index = word.find('a')
>>> print index
1
\end{verbatim}
\afterverb
%
In this example, we invoke {\tt find} on {\tt word} and pass
the letter we are looking for as a parameter.

Actually, the {\tt find} method is more general than our function:
it can find substrings, not just characters:

\beforeverb
\begin{verbatim}
>>> word.find('na')
2
\end{verbatim}
\afterverb
%
It can take as a second argument the index where it should start:

\index{optional argument}
\index{argument!optional}

\beforeverb
\begin{verbatim}
>>> word.find('na', 3)
4
\end{verbatim}
\afterverb
%
And as a third argument where it should stop:

\beforeverb
\begin{verbatim}
>>> name = 'bob'
>>> name.find('b', 1, 2)
-1
\end{verbatim}
\afterverb
%
This search fails because {\tt b} does not
appear in the index range from {\tt 1} to {\tt 2} (not including {\tt
2}).


\begin{ex}
There is a string method called {\tt count} that is similar
to the function in the previous exercise.  Read the documentation
of this method
and write an invocation that counts the number of {\tt a}s
in {\tt 'banana'}.  Hint: there are three.
\end{ex}


\section{The {\tt in} operator}
\label{inboth}

The operators we have seen so far are all special characters like {\tt
+} and {\tt *}, but there are a few operators that are words.  {\tt
in} is a boolean operator that takes two strings and returns {\tt
True} if the first appears as a substring in the second:

\beforeverb
\begin{verbatim}
>>> 'an' in 'banana'
True
>>> 'c' in 'banana'
False
\end{verbatim}
\afterverb
%
For example, the following function prints all the
letters from {\tt word1} that also appear in {\tt word2}:

\beforeverb
\begin{verbatim}
def in_both(word1, word2):
    for letter in word1:
        if letter in word2:
            print letter
\end{verbatim}
\afterverb
%
With well-chosen variable names,
Python sometimes reads like English.  You could read
this loop, ``for (each) letter in (the first) word, if (the) letter 
(appears) in (the second) word, print (the) letter.''

Here's what you get if you compare apples and oranges:

\beforeverb
\begin{verbatim}
>>> in_both('apples', 'oranges')
a
e
s
\end{verbatim}
\afterverb
%

\section{String comparison}
\index{string comparison}
\index{comparison!string}

The comparison operators work on
strings.  To see if two strings are equal:

\beforeverb
\begin{verbatim}
if word == 'banana':
    print  'Yes, we have no bananas!'
\end{verbatim}
\afterverb
%
Other comparison operations are useful for putting words in alphabetical
order:

\beforeverb
\begin{verbatim}
if word < 'banana':
    print 'Your word,' + word + ', comes before banana.'
elif word > 'banana':
    print 'Your word,' + word + ', comes after banana.'
else:
    print 'Yes, we have no bananas!'
\end{verbatim}
\afterverb
%
Python does not handle uppercase and lowercase letters the same way
that people do.  All the uppercase letters come before all the
lowercase letters, so:

\beforeverb
\begin{verbatim}
Your word, Zebra, comes before banana.
\end{verbatim}
\afterverb
%
A common way to address this problem is to convert strings to a
standard format, such as all lowercase, before performing the
comparison.  The more difficult problem is making the program realize
that zebras are not fruit.


% \section{The {\tt string} module}

% The string module contains variables and functions that pertain to
% strings.  Some of the string functions are redundant with the
% string methods; the method syntax is a little more concise,
% so most programmers don't use string functions.

% But there are several useful variables in the string module:

% \beforeverb
% \begin{verbatim}
% Your word, Zebra, comes before banana.
% \end{verbatim}
% \afterverb
%



\section{Debugging}

When you use indices to traverse the values in a sequence,
it is tricky to get the beginning and end of the traversal
right.  Here is a function that is supposed to compare two
words and return {\tt True} if one of the words is the reverse
of the other, but it contains two errors:

\beforeverb
\begin{verbatim}
def is_reverse(word1, word2):
    if len(word1) != len(word2):
        return False
    
    i = 0
    j = len(word2)

    while j > 0:
        if word1[i] != word2[j]:
            return False
        i = i+1
        j = j-1

    return True
\end{verbatim}
\afterverb
%
The first {\tt if} statement checks whether the words are the
same length.  If not, we can return {\tt False} immediately
and then, for the rest of the function, we can assume that the words
are the same length.  This is another example of a guardian.

{\tt i} and {\tt j} are indices: {\tt i} traverses {\tt word1}
forward while {\tt j} traverses {\tt word2} backward.  If we find
two letters that don't match, we can return {\tt False} immediately.
If we get through the whole loop and all the letters match, we
return {\tt True}.

If we test this function with the words ``pots'' and ``stop'', we
expect the return value {\tt True}, but we get an IndexError:

\beforeverb
\begin{verbatim}
>>> is_reverse('pots', 'stop')
...
  File "reverse.py", line 15, in is_reverse
    if word1[i] != word2[j]:
IndexError: string index out of range
\end{verbatim}
\afterverb
%
For debugging this kind of error, my first move is to
print the values of the indices immediately before the line
where the error appears.

\beforeverb
\begin{verbatim}
    while j > 0:
        print i, j        # print here
        
        if word1[i] != word2[j]:
            return False
        i = i+1
        j = j-1
\end{verbatim}
\afterverb
%
Now when I run the program again, I get more information:

\beforeverb
\begin{verbatim}
>>> is_reverse('pots', 'stop')
0 4
...
IndexError: string index out of range
\end{verbatim}
\afterverb
%
The first time through the loop, the value of {\tt j} is 4,
which is out of range for the string {\tt 'pots'}.
The index of the last character is 3, so the
initial value for {\tt j} should be {\tt len(word2)-1}.

If I fix that error and run the program again, I get:

\beforeverb
\begin{verbatim}
>>> is_reverse('pots', 'stop')
0 3
1 2
2 1
True
\end{verbatim}
\afterverb
%
This time we get the right answer, but it looks like the loop only ran
three times, which is suspicious.  To get a better idea of what is
happening, it is useful to draw a state diagram.  During the first
iteration, the frame for {\tt is\_reverse} looks like this:

\index{state diagram}
\index{diagram!state}

\beforefig
\centerline{\includegraphics{figs/state4.eps}}
\afterfig

I took a little license by arranging the variables in the frame
and adding dotted lines to show that the values of {\tt i} and
{\tt j} indicate characters in {\tt word1} and {\tt word2}.

\begin{ex}
\label{is_reverse}
Starting with this diagram, execute the program on paper, changing the
values of {\tt i} and {\tt j} during each iteration.  Find and fix the
second error in this function.
\end{ex}



\section{Glossary}

\begin{description}

\item[object:] Something a variable can refer to.  For now,
you can use ``object'' and ``value'' interchangeably.
\index{object}

\item[sequence:] An ordered set; that is, a set of
values where each value is identified by an integer index.
\index{sequence}

\item[item:] One of the values in a sequence.
\index{item}

\item[index:] An integer value used to select an item in
a sequence, such as a character in a string.
\index{index}

\item[slice:] A part of a string specified by a range of indices.
\index{slice}

\item[empty string:] A string with no characters and length 0, represented
by two quotation marks.
\index{empty string}

\item[immutable:] The property of a sequence whose items cannot
be assigned.
\index{immutable}

\item[traverse:] To iterate through the items in a sequence,
performing a similar operation on each.
\index{traverse}

\item[search:] A pattern of traversal that stops
when it finds what it is looking for.
\index{search}

\item[counter:] A variable used to count something, usually initialized
to zero and then incremented.
\index{counter}

\item[method:] A function that is associated with an object and called
using dot notation.
\index{method}

\item[invocation:] A statement that calls a method.
\index{invocation}

\end{description}


\section{Exercises}

\begin{ex}
The following functions are all {\em intended} to check whether a
string contains any lowercase letters, but at least some of them are
wrong.  For each function, write a clear, concise description of what
the function actually does.
{\tt string.lowercase} is a string defined in the {\tt string} module
that contains all lowercase letters.

\begin{verbatim}
def any_lowercase1(s):
    for c in s:
        if c in string.lowercase:
            return True
        else:
            return False

def any_lowercase2(s):
    for c in s:
        if 'c' in string.lowercase:
            return 'True'
        else:
            return 'False'

def any_lowercase3(s):
    for c in s:
        flag = c in string.lowercase
    return flag

def any_lowercase4(s):
    flag = False
    for c in s:
        flag = flag or (c in string.lowercase)
    return flag

def any_lowercase5(s):
    for c in s:
        if not (c in string.lowercase):
            return False
    return True
\end{verbatim}

\end{ex}


\begin{ex}
\label{exrotate}
Write a function called {\tt rotate\_word}
that takes a string and an integer as parameters, and that returns
a new string that contains the letters from the original string
``rotated'' by the given amount.  To rotate a letter means to
shift it through the alphabet, wrapping around to the beginning
if necessary.  For example, 'Y' shifted by 1 is 'Z', and 'Z' shifted
by 1 is 'A'.  'A' shifted by 5 is 'F'.

For example, ``cheer'' rotated by 7 is ``jolly'' and ``melon'' rotated
by -10 is ``cubed''.  

%For example ``sleep''
%rotated by 9 is ``bunny'' and ``latex'' rotated by 7 is ``shale''.

You might want to use the built-in functions {\tt ord}, which converts
a character to a numeric code, and {\tt chr}, which converts numeric
codes to characters.
\end{ex}

\chapter{Case study: word play}

\section{Reading word lists}
\label{wordlist}

For the exercises in this chapter we need a list of English words.
There are lots of word lists available on the Web, but the one most
suitable for our purpose is one of the word lists collected (and
contributed to the public domain) by Grady Ward as part of the Moby
lexicon project.  It is a list of 113,809 official crosswords;
that is, words that are considered valid in crossword puzzles and
other word games.  In the Moby collection, the filename is
{\tt 113809of.fic}; I include a copy of this file, with the
simpler name {\tt words.txt}, along with Swampy.

This file is in plain text, so you can open it with a text
editor, but you can also read it from Python.  The built-in
function {\tt open} takes the name of the file as a parameter
and returns a {\bf file object} you can use to read the file.

\index{object!file}
\index{file object}

\beforeverb
\begin{verbatim}
>>> fin = open('words.txt')
>>> print fin
<open file 'words.txt', mode 'r' at 0xb7f4b380>
\end{verbatim}
\afterverb
%
{\tt fin} is a common name for a file object used for
input.  Mode {\tt 'r'} indicates that this file is open for
reading.

The file object provides several methods for reading, including
{\tt readline}, which reads characters from the file
until it gets to a newline, and returns the result as a
string:

\beforeverb
\begin{verbatim}
>>> fin.readline()
'aa\r\n'
\end{verbatim}
\afterverb
%
The first word in this particular list is ``aa,'' which is a kind of
lava.  The sequence \verb+\r\n+ represents two whitespace characters,
a carriage return and a newline, that separate this word from the
next.

The file object keeps track of where it is in the file, so
if you call {\tt readline} again, you get the next word:

\beforeverb
\begin{verbatim}
>>> fin.readline()
'aah\r\n'
\end{verbatim}
\afterverb
%
The next word is ``aah,'' which is a perfectly legitimate
word, so stop looking at me like that.
Or, if it's the whitespace that's bothering you,
we can get rid of it with the string method {\tt strip}:

\beforeverb
\begin{verbatim}
>>> line = fin.readline()
>>> word = line.strip()
>>> print word
aahed
\end{verbatim}
\afterverb
%
You can also use a file object as part of a {\tt for} loop.
This program reads {\tt words.txt} and prints each word, one
per line:

\beforeverb
\begin{verbatim}
fin = open('words.txt')
for line in fin:
    word = line.strip()
    print word
\end{verbatim}
\afterverb
%

\begin{ex}
Write a program that reads {\tt words.txt} and prints only the
words with more than 20 characters (not counting whitespace).
\end{ex}


\section{Exercises}

There are solutions to these exercises in the next section.
You should at least attempt each one before you read the solutions.

\begin{ex}
In 1939 Ernest Vincent Wright published a 50,000 word novel called
{\em Gadsby} that does not contain the letter 'e'.  Since 'e' is
the most common letter in English, that's not easy to do.

In fact, it is difficult to construct a solitary thought without
using that most common symbol.  It is slow going at first,
but with caution and hours of training
you can gradually gain facility.

All right, I'll stop now.

Write a function called {\tt has\_no\_e} that returns {\tt True} if
the given word doesn't have the letter ``e'' in it.

Modify your program from the previous section to print only the words
that have no ``e'' and compute the percentage of the words in the list
have no ``e.''
\end{ex}


\begin{ex} 
Write a function named {\tt avoids}
that takes a word and a string of forbidden letters, and
that returns {\tt True} if the word doesn't use any of the forbidden
letters.

Modify your program to prompt the user to enter a string
of forbidded letters and then print the number of words that
don't contain any of them.
Can you find a combination of 5 forbidden letters that
excludes the smallest number of words?
\end{ex}



\begin{ex}
Write a function named {\tt uses\_only} that takes a word and a
string of letters, and that returns {\tt True} if the word contains
only letters in the list.  Can you make a sentence using only the
letters {\tt acefhlo}?  Other than ``Hoe alfalfa?''
\end{ex}


\begin{ex} 
Write a function named {\tt uses\_all} that takes a word and a
string of required letters, and that returns {\tt True} if the word
uses all the required letters at least once.  How many words are there
that use all the vowels {\tt aeiou}?  How about {\tt aeiouy}?
\end{ex}


\begin{ex}
Write a function called {\tt is\_abecedarian} that returns
{\tt True} if the letters in a word appear in alphabetical order
(double letters are ok).  
How many abecedarian words are there?
\end{ex}

\index{abecedarian}


\begin{ex}
\label{palindrome}
A palindrome is a word that reads the same
forward and backward, like ``rotator'' and ``noon.''
Write a boolean function named {\tt is\_palindrome} that
takes a string as a parameter and returns {\tt True} if it is
a palindrome.

Modify your program from the previous section to print all
of the palindromes in the word list and then print the total
number of palindromes.
\end{ex}



\section{Search}

All of the exercises in the previous section have something
in common; they can be solved with the search pattern we saw
in Section~\ref{find}.  The simplest example is:

\beforeverb
\begin{verbatim}
def has_no_e(word):
    for letter in word:
        if letter == 'e':
            return False
    return True
\end{verbatim}
\afterverb
%
The {\tt for} loop traverses the characters in {\tt word}.  If we find
the letter ``e'', we can immediately return {\tt False}; otherwise we
have to go to the next letter.  If we exit the loop normally, that
means we didn't find an ``e'', so we return {\tt True}.

You can write this function more concisely using the {\tt in}
operator, but I wanted to start with this version because it 
demonstrates the logic of the search pattern.

{\tt avoids} is a more general version of {\tt has\_no\_e} but it
has the same structure:

\beforeverb
\begin{verbatim}
def avoids(word, forbidden):
    for letter in word:
        if letter in forbidden:
            return False
    return True
\end{verbatim}
\afterverb
%
We can return {\tt False} as soon as we find a forbidden letter;
if we get to then end of the loop, we can return {\tt True}.

{\tt uses\_only} is similar except that the sense of the condition
is reversed:

\beforeverb
\begin{verbatim}
def uses_only(word, available):
    for letter in word: 
        if letter not in available:
            return False
    return True
\end{verbatim}
\afterverb
%
Instead of a list of forbidden words, we have a list of available
words.  If we find a letter in {\tt word} that is not in
{\tt available}, we can return {\tt False}.

{\tt uses\_all} is also similar, except that we reverse the role
of the word and the string of letters:

\beforeverb
\begin{verbatim}
def uses_all(word, required):
    for letter in required: 
        if letter not in word:
            return False
    return True
\end{verbatim}
\afterverb
%
Instead of traversing the letters in {\tt word}, the loop
traverses the required letters.  If any of the required letters
do not appear in the word, we can return {\tt False}.

If you were really thinking like a computer scientist, you would
have recognized that {\tt uses\_all} was an instance of a
previously-solved problem, and you would have written:

\beforeverb
\begin{verbatim}
def uses_all(word, required):
    return uses_only(required, word)
\end{verbatim}
\afterverb
%
This is an example of a program development method called {\bf problem
recognition}, which means that you recognize the problem you are
working on as an instance of a previously-solved problem, and apply a
previously-developed solution.


\section{Looping with indices}

We could write the functions in the previous section with {\tt for}
loops, because we only needed characters in the strings; we didn't
have to do anything with the indices.

For some of the other exercises, like {\tt is\_abecedarian},
we need the indices, so it is easier to use a while loop:

\beforeverb
\begin{verbatim}
def is_abecedarian(word):
    i = 0
    while i < len(word)-1:
        if word[i+1] < word[i]:
            return False
        i = i+1
    return True
\end{verbatim}
\afterverb
%
The loop starts at {\tt i=0} and ends when {\tt i=len(word)-1}.  Each
time through the loop, it compares the $i$th character (which you can
think of as the current character) to the $i+1$th character (which you
can think of as the next).

If the next character is less than (alphabetically before) the current
one, then we have discovered a break in the abecedarian trend, as
we return {\tt False}.

If we get to the end of the loop without finding a fault, then the
word passes the test.  To convince yourself that the loop ends
correctly, consider an example like {\tt 'flossy'}.  The
length of the word is 6, so the loop stops when {\tt i} is 5, so
the last time the loop runs is when {\tt i} is 4, which is the
index of the second-to-last character.  So on the last iteration,
it compares the second-to-last character to the last, which is
what we want.

The structure for {\tt is\_palindrome} is similar except that
we need two indices; one starts at the begining and goes up;
the other starts at the end and goes down.

\beforeverb
\begin{verbatim}
def is_palindrome(word):
    i = 0
    j = len(word)-1

    while i<j:
        if word[i] != word[j]:
            return False
        i = i+1
        j = j-1

    return True
\end{verbatim}
\afterverb
%
Or, if you noticed that this is an instance of a previously-solved
problem, you might have written:

\beforeverb
\begin{verbatim}
def is_palindrome(word):
    return is_reverse(word, word)
\end{verbatim}
\afterverb
%
Assuming you did Exercise~\ref{is_reverse}.


\section{Debugging}

Testing programs is hard.  The functions in this chapter are
relatively easy to test because you can check the results by hand.
Even so, it is somewhere between difficult and impossible to choose a
set of words that test for all possible errors.

Taking {\tt has\_no\_e} as an example, there are two obvious
cases to check: words that have an 'e' should return {\tt False};
words that don't should return {\tt True}.  You should have no
trouble coming up with one of each.

Within each case, there are some less obvious subcases.  Among the
words that have an 'e', you should test words with an 'e' at the
beginning, the end, and somewhere in the middle.  You should test long
words, short words, and very short words, like the empty string.  The
empty string is an example of a {\bf special case}, which is one of
the non-obvious cases where errors often lurk.

In addition to the test cases you generate, you can also test
your program with a word list like {\tt words.txt}.  By scanning
the output, you might be able to catch errors, but be careful:
you might catch one kind of error (words that should not be
included, but are) and not another (words that should be included,
but aren't).

In general, testing can help you find bugs, but it is not easy to
generate a good set of test cases, and even if you do, you can't
be sure your program is correct.

And that brings us to the Sixth Theorem of Debugging:

\begin{quote}
Program testing can be used to show the presence of bugs, but never to
show their absence!

--- Edsger W. Dijkstra
\end{quote}

\index{Dijkstra, Edsger}

\section{Glossary}

\begin{description}

\item[file object:] A value that represents an open file.
\index{file object}
\index{object!file}

\item[problem recognition:] A way of solving a problem by
expressing it as an instance of a previously-solved problem.
\index{problem recognition}

\item[special case:] A test case that is atypical or non-obvious
(and less likely to be handled correctly).
\index{special case}

\end{description}


\chapter{Lists}
\index{list}
\index{type!list}


\section{A list is a sequence}

Like a string, a {\bf list} is a sequence of values.  In a string, the
values are characters; in a list, they can be any type.  The values in
list are called {\bf elements} or sometimes {\bf items}.

\index{element}
\index{sequence}
\index{item}

There are several ways to create a new list; the simplest is to
enclose the elements in square brackets (\verb+[+ and \verb+]+):

\beforeverb
\begin{verbatim}
[10, 20, 30, 40]
['crunchy frog', 'ram bladder', 'lark vomit']
\end{verbatim}
\afterverb
%
The first example is a list of four integers.  The second is a list of
three strings.  The elements of a list don't have to be the same type.
The following list contains a string, a float, an integer, and
(lo!) another list:

\beforeverb
\begin{verbatim}
['spam', 2.0, 5, [10, 20]]
\end{verbatim}
\afterverb
%
A list within another list is said to be {\bf nested}.

\index{nested list}
\index{list!nested}

A list that contains no elements is
called an empty list; you can create one with empty
brackets, {\tt []}.

\index{empty list}
\index{list!empty}

Lists that contain consecutive integers are common, so Python provides a
built-in function to create them:

\index{range}

\beforeverb
\begin{verbatim}
>>> range(1,5)
[1, 2, 3, 4]
\end{verbatim}
\afterverb
%
{\tt range} takes two arguments and returns a list that
contains all the integers from the first to the second, including the
first but not including the second!

With one argument, {\tt range} creates a list that starts at 0:

\beforeverb
\begin{verbatim}
>>> range(10)
[0, 1, 2, 3, 4, 5, 6, 7, 8, 9]
\end{verbatim}
\afterverb
%
If there is a third argument, it specifies the space between
successive values, which is called the ``step size.''  This example
counts from 1 to 10 by steps of 2:

\beforeverb
\begin{verbatim}
>>> range(1, 10, 2)
[1, 3, 5, 7, 9]
\end{verbatim}
\afterverb
%
As you might expect, you can assign list values to variables:

\beforeverb
\begin{verbatim}
>>> cheeses = ['Cheddar', 'Edam', 'Gouda']
>>> numbers = [17, 123]
>>> empty = []
>>> print cheeses, numbers, empty
['Cheddar', 'Edam', 'Gouda'] [17, 123] []
\end{verbatim}
\afterverb
%


% From Jeff: write sum for a nested list?


\section{Lists are mutable}
\index{list!element}
\index{access}

The syntax for accessing the elements of a list is the same as for
accessing the characters of a string---the bracket operator ({\tt
[]}).  The expression inside the brackets specifies the index.
Remember that the indices start at 0:

\beforeverb
\begin{verbatim}
>>> print cheeses[0]
Cheddar
\end{verbatim}
\afterverb
%
Unlike strings, lists are mutable.  When the bracket operator appears
on the left side of an assignment, it identifies the element of the
list that will be assigned.

\beforeverb
\begin{verbatim}
>>> numbers = [17, 123]
>>> numbers[1] = 5
>>> print numbers
[17, 5]
\end{verbatim}
\afterverb
%
You can think of a list as a relationship between indices and
elements.  This relationship is called a {\bf mapping}; each index
``maps to'' one of the elements.  Here is a state diagram showing {\tt
cheeses}, {\tt numbers} and {\tt empty}:

\index{state diagram}
\index{diagram!state}
\index{mapping}

\beforefig
\centerline{\includegraphics{figs/list_state.eps}}
\afterfig

Lists are represented by boxes with the word ``list'' outside
and the elements of the list inside.  {\tt cheeses} refers to
a list with three elements indexed 0, 1 and 2.
{\tt numbers} contains two elements; the diagram shows that the
value of the second element has been reassigned from 123 to 5.
{\tt empty} refers to a list with no elements.

The bracket operator can appear anywhere in an expression.  When it
appears on the left side of an assignment, it changes one of the
elements in the list, so the one-eth element of {\tt numbers}, which
used to be 123, is now 5.

List indices work the same way as string indices:

\begin{itemize}

\item Any integer expression can be used as an index.

\item If you try to read or write an element that does not exist, you
get an {\tt IndexError}.

\index{error!IndexError}
\index{IndexError}

\item If an index has a negative value, it counts backward from the
end of the list.

\end{itemize}


\index{list!membership}
\index{in operator}
\index{operator!in}

The {\tt in} operator also works on lists.

\beforeverb
\begin{verbatim}
>>> cheeses = ['Cheddar', 'Edam', 'Gouda']
>>> 'Edam' in cheeses
True
>>> 'Brie' in cheeses
False
\end{verbatim}
\afterverb


\section{Traversing a list}
\index{list!traversal}
\index{traversal!list}
\index{for loop}
\index{loop!for}

The most common way to traverse the elements of a list is
with a {\tt for} loop.  The syntax is the same as for strings:

\beforeverb
\begin{verbatim}
for cheese in cheeses:
    print cheese
\end{verbatim}
\afterverb
%
This works well if you only need to read the elements of the
list.  But if you want to write or update the elements, you
need the indices.  A common way to do that is to combine
the functions {\tt range} and {\tt len}:

\beforeverb
\begin{verbatim}
for i in range(len(numbers)):
    numbers[i] = numbers[i] * 2
\end{verbatim}
\afterverb
%
This loop traverses the list and updates each element.  {\tt len}
returns the number of elements in the list.  {\tt range} returns
a list of indices from 0 to $n-1$, where $n$ is the length of
the list.  Each time through the loop {\tt i} gets the index
of the next element.  The assignment statement in the body uses
{\tt i} to read the old value of the element and to assign the
new value.

A {\tt for} loop over an empty list never executes the body:

\beforeverb
\begin{verbatim}
for x in empty:
    print 'This never happens.'
\end{verbatim}
\afterverb
%
Although a list can contain another list, the nested
list still counts as a single element.  The length of this list is
four:

\beforeverb
\begin{verbatim}
['spam!', 1, ['Brie', 'Roquefort', 'Pol le Veq'], [1, 2, 3]]
\end{verbatim}
\afterverb



\section{List operations}
\index{list operation}
\index{operation!list}

The {\tt +} operator concatenates lists:

\index{concatenation!list}

\beforeverb
\begin{verbatim}
>>> a = [1, 2, 3]
>>> b = [4, 5, 6]
>>> c = a + b
>>> print c
[1, 2, 3, 4, 5, 6]
\end{verbatim}
\afterverb
%
Similarly, the {\tt *} operator repeats a list a given number of times:

\index{repetition!list}

\beforeverb
\begin{verbatim}
>>> [0] * 4
[0, 0, 0, 0]
>>> [1, 2, 3] * 3
[1, 2, 3, 1, 2, 3, 1, 2, 3]
\end{verbatim}
\afterverb
%
The first example repeats {\tt [0]} four times.  The second example
repeats the list {\tt [1, 2, 3]} three times.


\section{List slices}
\index{slice}
\index{list!slice}

The slice operator also work on lists:

\beforeverb
\begin{verbatim}
>>> t = ['a', 'b', 'c', 'd', 'e', 'f']
>>> t[1:3]
['b', 'c']
>>> t[:4]
['a', 'b', 'c', 'd']
>>> t[3:]
['d', 'e', 'f']
\end{verbatim}
\afterverb
%
If you omit the first index, the slice starts at the beginning.
If you omit the second, the slice goes to the end.  So if you
omit both, the slice is a copy of the whole list.

\beforeverb
\begin{verbatim}
>>> t[:]
['a', 'b', 'c', 'd', 'e', 'f']
\end{verbatim}
\afterverb
%
A slice operator on the left side of an assignment
can update multiple elements:

\beforeverb
\begin{verbatim}
>>> t = ['a', 'b', 'c', 'd', 'e', 'f']
>>> t[1:3] = ['x', 'y']
>>> print t
['a', 'x', 'y', 'd', 'e', 'f']
\end{verbatim}
\afterverb
%

% You can add elements to a list by squeezing them into an empty
% slice:

% \beforeverb
% \begin{verbatim}
% >>> t = ['a', 'd', 'e', 'f']
% >>> t[1:1] = ['b', 'c']
% >>> print t
% ['a', 'b', 'c', 'd', 'e', 'f']
% \end{verbatim}
% \afterverb
%
% And you can remove elements from a list by assigning the empty list to
% them:

% \beforeverb
% \begin{verbatim}
% >>> t = ['a', 'b', 'c', 'd', 'e', 'f']
% >>> t[1:3] = []
% >>> print t
% ['a', 'd', 'e', 'f']
% \end{verbatim}
% \afterverb
%
% But both of those operations can be expressed more clearly
% with list methods.


\section{List methods}
\index{list methods}
\index{methods!list}

Python provides methods that operate on lists.  For example,
{\tt append} adds a new element to the end of a list:

\beforeverb
\begin{verbatim}
>>> t = ['a', 'b', 'c']
>>> t.append('d')
>>> print t
['a', 'b', 'c', 'd']
\end{verbatim}
\afterverb
%
{\tt extend} takes a list as an argument and appends all of
the elements:

\beforeverb
\begin{verbatim}
>>> t1 = ['a', 'b', 'c']
>>> t2 = ['d', 'e']
>>> t1.extend(t2)
>>> print t1
['a', 'b', 'c', 'd', 'e']
\end{verbatim}
\afterverb
%
This example leaves {\tt t2} unmodified.

{\tt sort} arranges the elements of the list from low to high:

\beforeverb
\begin{verbatim}
>>> t = ['d', 'c', 'e', 'b', 'a']
>>> t.sort()
>>> print t
['a', 'b', 'c', 'd', 'e']
\end{verbatim}
\afterverb
%
List methods are all void; they modify the list and return {\tt None}.
If you accidentally write {\tt t = t.sort()}, you will be disappointed
with the result.


\section{Map, filter and reduce}

To add up all the numbers in a list, you can use a loop like this:

% see add.py

\beforeverb
\begin{verbatim}
def add_all(t):
    total = 0
    for x in t:
        total += x
    return total
\end{verbatim}
\afterverb
%
{\tt total} is initialized to 0.  Each time through the loop,
{\tt x} gets one element from the list.  The {\tt +=} operator
provides a short way to update a variable:

\index{{\tt +=}}
\index{operator!{\tt +=}}

\beforeverb
\begin{verbatim}
    total += x
\end{verbatim}
\afterverb
%
is equivalent to:

\beforeverb
\begin{verbatim}
    total = total + x
\end{verbatim}
\afterverb
%
As the loop executes, {\tt total} accumulates the sum of the
elements; a variable used this way is sometimes called an
{\bf accumulator}.

\index{accumulator}

Adding up the elements of a list is such a common operation
that Python provides it as a built-in function, {\tt sum}:

\beforeverb
\begin{verbatim}
>>> t = [1, 2, 3]
>>> sum(t)
6
\end{verbatim}
\afterverb
%
An operation like this that combines a sequence of elements into
a single value is sometimes called {\bf reduce}.

\index{reduce}

Sometimes you want to traverse one list while building
another.  For example, the following function takes a list of strings
and returns a new list that contains capitalized strings:

\beforeverb
\begin{verbatim}
def capitalize_all(t):
    res = []
    for s in t:
        res.append(s.capitalize())
    return res
\end{verbatim}
\afterverb
%
{\tt res} is initialized with an empty list; each time through
the loop, we append the next element.  So {\tt res} is another
kind of accumulator.

An operation like {\tt capitalize\_all} is sometimes called a {\bf
map} because it ``maps'' a function (in this case the method {\tt
capitalize}) onto each of the elements in a sequence.

Another common operation is to select some of the elements from
a list and return a sublist.  For example, the following
function takes a list of strings and returns a list that contains
only the uppercase strings:

\beforeverb
\begin{verbatim}
def only_upper(t):
    res = []
    for s in t:
        if s.isupper():
            res.append(s)
    return res
\end{verbatim}
\afterverb
%
{\tt isupper} is a string method that returns {\tt True} if
the string contains only upper case letters.

An operation like {\tt only\_upper} is called a {\bf filter} because
it selects some of the elements and filters out the others.

\index{filter}

Most common list operations can be expressed as a combination
of map, filter and reduce.  Because these operations are
so common, Python provides language features to support them,
including the built-in function {\tt reduce} and an operator
called a ``list comprehension.''  But these features are 
idiomatic to Python, so I won't go into the details.


\begin{ex}
Write a function that takes a list of numbers and returns the
cumulative sum; that is, a new list where the $i$th element
is the sum of the first $i+1$ elements from the original list.
For example, the cumulative sum of {\tt [1, 2, 3]} is
{\tt [1, 3, 6]}. 
\end{ex}


\section{Deleting elements}
\index{deletion!list}
\index{list deletion}

There are several ways to delete elements from a list.  If you
know the index of the element you want, you can use
{\tt pop}:

\beforeverb
\begin{verbatim}
>>> t = ['a', 'b', 'c']
>>> x = t.pop(1)
>>> print t
['a', 'c']
>>> print x
b
\end{verbatim}
\afterverb
%
{\tt pop} modifies the list and returns the element that was removed.

If you don't need the removed value, you can use the {\tt del}
operator:

\beforeverb
\begin{verbatim}
>>> t = ['a', 'b', 'c']
>>> del t[1]
>>> print t
['a', 'c']
\end{verbatim}
\afterverb
%

If you know the element you want to remove (but not the index), you
can use {\tt remove}:

\beforeverb
\begin{verbatim}
>>> t = ['a', 'b', 'c']
>>> t.remove('b')
>>> print t
['a', 'c']
\end{verbatim}
\afterverb
%
The return value from {\tt remove} is {\tt None}.

To remove more than one element, you can use {\tt del} with
a slice index:

\beforeverb
\begin{verbatim}
>>> t = ['a', 'b', 'c', 'd', 'e', 'f']
>>> del t[1:5]
>>> print t
['a', 'f']
\end{verbatim}
\afterverb
%
As usual, the slice selects all the elements up to, but not
including, the second index.



\section{Objects and values}
\index{object}
\index{value}

If we execute these assignment statements:

\beforeverb
\begin{verbatim}
a = 'banana'
b = 'banana'
\end{verbatim}
\afterverb
%
We know that {\tt a} and {\tt b} both refer to a
string, but we don't
know whether they refer to the {\em same} string.
There are two possible states:

\beforefig
\centerline{\includegraphics{figs/list1.eps}}
\afterfig

In one case, {\tt a} and {\tt b} refer to two different objects that
have the same value.  In the second case, they refer to the same
object.

\index{{\tt is} operator}
\index{operator!{\tt is}}

To check whether two variables refer to the same object, you can
use the {\tt is} operator.

\beforeverb
\begin{verbatim}
>>> a = 'banana'
>>> b = 'banana'
>>> a is b
True
\end{verbatim}
\afterverb
%
In this example, Python only created one string object,
and both {\tt a} and {\tt b} refer to it.

In contrast, when you create two lists, you get two objects:

\beforeverb
\begin{verbatim}
>>> a = [1, 2, 3]
>>> b = [1, 2, 3]
>>> a is b
False
\end{verbatim}
\afterverb
%
So the state diagram looks like this:

\index{state diagram}
\index{diagram!state}

\beforefig
\centerline{\includegraphics{figs/list2.eps}}
\afterfig

In this case we would say that the two lists are {\bf equivalent},
because they have the same elements, but not {\bf identical}, because
they are not the same object.  If two objects are identical, they are
also equivalent, but if they are equivalent, they are not necessarily
identical.

\index{equivalent}
\index{identical}

Until now, we have been using ``object'' and ``value''
interchangeably, but it is more precise to say that an object has a
value.  If you execute {\tt a = [1,2,3]}, {\tt a} refers to a list
object whose value is a particular sequence of elements.  If another
list has the same elements, we would say it has the same value.

\index{object}
\index{value}


\section{Aliasing}
\index{aliasing}
\index{reference!aliasing}

If {\tt a} refers to an object and you assign {\tt b = a},
then both variables refer to the same object.  For example,
if you execute:

\beforeverb
\begin{verbatim}
>>> a = [1, 2, 3]
>>> b = a
\end{verbatim}
\afterverb
%
Then {\tt a} and {\tt b} refer to the same list.
The state diagram looks like this:

\index{state diagram}
\index{diagram!state}

\beforefig
\centerline{\includegraphics{figs/list3.eps}}
\afterfig

The association of a variable with an object is called a {\bf
reference}.  In this example, there are two references to the same
object.

An object with more than one reference has, in some sense, more
than one name, so we say that the object is {\bf aliased}.

If the aliased object is mutable, 
changes made with one alias affect
the other:

\beforeverb
\begin{verbatim}
>>> b[0] = 17
>>> print a
[17, 2, 3]
\end{verbatim}
\afterverb
%
Although this behavior can be useful, it is sometimes unexpected or
undesirable.  In general, it is safer to avoid aliasing when you
are working with mutable objects.

For immutable objects like strings, aliasing is not as much of a
problem.  In this example:

\beforeverb
\begin{verbatim}
a = 'banana'
b = 'banana'
\end{verbatim}
\afterverb
%
It almost never makes a difference whether {\tt a} and {\tt b} refer
to the same string or not.


\section{List arguments}
\index{list!as argument}
\index{argument}
\index{argument!list}

When you pass a list to a function, the function gets a reference
to the list.
If the function modifies a list parameter, the caller sees the change.
For example, {\tt delete\_head} removes the first element from a list:

\beforeverb
\begin{verbatim}
def delete_head(t):
    del t[0]
\end{verbatim}
\afterverb
%
Here's how it is used:

\beforeverb
\begin{verbatim}
>>> letters = ['a', 'b', 'c']
>>> delete_head(letters)
>>> print letters
['b', 'c']
\end{verbatim}
\afterverb
%
The parameter {\tt t} and the variable {\tt letters} are
aliases for the same object.  The stack diagram looks like
this:

\index{stack diagram}
\index{diagram!stack}

\beforefig
\centerline{\includegraphics{figs/stack5.eps}}
\afterfig

Since the list is shared by two frames, I drew
it between them.

If a function returns a list, it returns a reference to the list.  For
example, {\tt tail} returns a list that contains all but the first
element of the given list:

\beforeverb
\begin{verbatim}
def tail(t):
    return t[1:]
\end{verbatim}
\afterverb
%
Here's how {\tt tail} is used:

\beforeverb
\begin{verbatim}
>>> letters = ['a', 'b', 'c']
>>> rest = tail(letters)
>>> print rest
['b', 'c']
\end{verbatim}
\afterverb
%
Because the return value was created with the slice operator, it
is a new list.  The original list is unmodified.



\section{Copying lists}
\index{list!copying}
\index{copying}

When you assign an object to a variable, Python
copies the reference to the object.

\beforeverb
\begin{verbatim}
>>> a = [1, 2, 3]
>>> b = a
\end{verbatim}
\afterverb
%
In this case {\tt a} and {\tt b} refer to the same list.

\index{slice}
\index{operator!slice}

If you want to copy the list (not just a reference to it),
you can use the slice operator:

\beforeverb
\begin{verbatim}
>>> a = [1, 2, 3]
>>> b = a[:]
>>> print b
[1, 2, 3]
\end{verbatim}
\afterverb
%
Making a slice of {\tt a} creates a new list.
In this case the slice contains all of the elements from the
original list.

\index{copy}
\index{module!copy}

Another way to make a copy is the {\tt copy} function from
the {\tt copy} module:

\beforeverb
\begin{verbatim}
>>> import copy
>>> a = [1, 2, 3]
>>> b = copy.copy(a)
>>> print b
\end{verbatim}
\afterverb
%
But it is more idiomatic to use the slice operator.


\section{Lists and strings}
\index{list}
\index{string}

A string is a sequence of characters and a list is a sequence
of values, but a list of characters is not the same as a
string.  To convert from a string to a list of characters,
you can use the {\tt list} function:

\beforeverb
\begin{verbatim}
>>> s = 'spam'
>>> t = list(s)
>>> print t
['s', 'p', 'a', 'm']
\end{verbatim}
\afterverb
%
{\tt list} breaks a string into individual letters.  If you
want to break a string into words, you can use the
{\tt split} method:

\beforeverb
\begin{verbatim}
>>> s = 'pining for the fjords'
>>> t = s.split()
>>> print t
['pining', 'for', 'the', 'fjords']
\end{verbatim}
\afterverb
%
An optional argument called a {\bf delimiter} specifies which
characters to use as word boundaries.
The following example
uses {\tt ', '} (a comma followed by a space) as the delimiter:

\index{optional argument}
\index{argument!optional}
\index{delimiter}

\beforeverb
\begin{verbatim}
>>> s = 'spam, spam, spam'
>>> delimiter = ', '
>>> s.split(delimiter)
['spam', 'spam', 'spam']
\end{verbatim}
\afterverb
%
{\tt join} is the inverse of {\tt split}.  It
takes a list of strings and
concatenates the elements.  {\tt join} is a string method,
so you have to invoke it on the delimiter and pass the
list as a parameter:

\index{concatenate}

\beforeverb
\begin{verbatim}
>>> t = ['pining', 'for', 'the', 'fjords']
>>> delimiter = ' '
>>> delimiter.join(t)
'pining for the fjords'
\end{verbatim}
\afterverb
%
In this case the delimiter is a space character, so
{\tt join} puts a space between words.  To concatenate
strings without spaces, you can use the empty string,
{\tt ''} as a delimiter. 


\section{Debugging}

When you are debugging a program, and especially if you are
working on a hard bug, there are four things to try:

\begin{description}

\item[reading:] Examine your code, read it back to yourself, and
check that it means what you meant to say.

\item[running:] Experiment by making changes and running different
versions.  Often if you display the right thing at the right place
in the program, the problem becomes obvious, but sometimes you have to
spend some time to build scaffolding.

\item[ruminating:] Take some time to think!  What kind of error
is it: syntax, run-time, logical?  What information can you get from
the error messages, or from the output of the program?  What kind of
error could cause the problem you're seeing?  What did you change
last, before the problem appeared?

\item[retreating:] At some point, the best thing to do is back
off, undoing recent changes, until you get back to a program that
works, and that you understand.  Then you can starting rebuilding.

\end{description}

Beginning programmers sometimes get stuck on one of these activities
and forget the others.  Each activity comes with its own failure
mode.

For example, reading your code might help if the problem is a
typographical error, but not if the problem is a conceptual
misunderstanding.  If you don't understand what your program does, you
can read it 100 times and never see the error, because the error is in
your head.

Running experiments can help, especially if you run small, simple
tests.  But if you run experiments without thinking or reading your
code, you might fall into a pattern I call ``random walk programming,''
which is the process of making random changes until the program
does the right thing.  Needless to say, random walk programming
can take a long time.

The way out is to take more time to think.  Debugging is like an
experimental science.  You should have at least one hypothesis about
what the problem is.  If there are two or more possibilities, try to
think of a test that would eliminate one of them.

Taking a break sometimes helps with the thinking.  So does talking.
If you explain the problem to someone else (or even yourself), you
will sometimes find the answer before you finish asking the question.

But even the best debugging techniques will fail if there are too many
errors, or if the code you are trying to fix is too big and
complicated.  Sometimes the best option is to retreat, simplifying the
program until you get to something that you understand, and that
works.

Beginning programmers are often reluctant to retreat, because
they can't stand to delete a line of code (even if it's wrong).
If it makes you feel better, copy your program into another file
before you start stripping it down.  Then you can paste the pieces
back in a little bit at a time.

To summarize, here's the Seventh Theorem of debugging:

\begin{quote}
Finding a hard bug requires reading, running, ruminating, and
sometimes retreating.  If you get stuck on one of these activities,
try the others.
\end{quote}



\section{Glossary}

\begin{description}

\item[list:] A sequence of values.
\index{list}

\item[element:] One of the values in a list (or other sequence),
also called items.
\index{element}

\item[index:] An integer value that indicates an element in a list.
\index{index}

\item[nested list:] A list that is an element of another list.
\index{nested list}

\item[list traversal:] The sequential accessing of each element in a list.
\index{list traversal}

\item[mapping:] A relationship in which each element of one set
corresponds to an element of another set.  For example, a list is
a mapping from indices to elements.
\index{mapping}

\item[accumulator:] A variable used in a loop to add up or
accumulate a result.
\index{accumulator}

\item[reduce:] A processing pattern that traverses a sequence 
and accumulates the elements into a single result.
\index{reduce}

\item[map:] A processing pattern that traverses a sequence and
performs an operation on each element.
\index{map}

\item[filter:] A processing pattern that traverses a list and
selects the elements that satisfy some criterion.
\index{filter}

\item[object:] Something a variable can refer to.  An object
has a type and a value.
\index{object}

\item[equivalent:] Having the same value.
\index{equivalent}

\item[identical:] Being the same object (which implies equivalence).
\index{identical}

\item[reference:] The association between a variable and its value.
\index{reference}

\item[aliasing:] A circumstance where two variables refer to the same
object.
\index{aliasing}

\item[delimiter:] A character or string used to indicate where a
string should be split.
\index{delimiter}

\end{description}


\section{Exercises}

\begin{ex}
The slice operator can take a third argument that determines
the step size, so {\tt t[::2]} creates a list that contains
every other element from {\tt t}.  If the step size is negative,
it goes through the list backward, so {\tt t[::-1]} creates
a list of all the elements in {\tt t} in reverse order.

Use this idiom to write a one-line
version of {\tt is\_palindrome} from Exercise~\ref{palindrome}.
\end{ex}

\begin{ex}
Write a function called {\tt is\_sorted} that takes a list as a
parameter and returns {\tt True} if the list is sorted in ascending
order and {\tt False} otherwise.  You can assume (as a precondition)
that the elements of the list can be compared with the comparison
operators {\tt <}, {\tt >}, etc.

For example, {\tt is\_sorted([1,2,2])} should return {\tt True}
and {\tt is\_sorted(['b','a'])} should return {\tt False}.
\end{ex}


\chapter{Dictionaries}
\index{dictionary}

\index{dictionary}
\index{data type!dictionary}
\index{type!dict}
\index{key}
\index{key-value pair}
\index{index}

A {\bf dictionary} is like a list, but more general.  In a list,
the indices have to be integers; in a dictionary they can
be (almost) any type.

You can think of a dictionary as a mapping between a set of
indices and a set of values.  Each index, which is called
a {\bf key}, corresponds to a value.
The association of
a key and a value is called a {\bf key-value pair} or sometimes
an {\bf item}.

As an example, we will build a dictionary that maps from English words
to Spanish words, so the keys and the values are all strings.

The function {\tt dict} creates a new dictionary with no items.

\beforeverb
\begin{verbatim}
>>> eng2sp = dict()
>>> print eng2sp
{}
\end{verbatim}
\afterverb
%
The squiggly-brackets, \verb+{}+, represent an empty dictionary.
To add items to the dictionary, you can use square brackets:

\beforeverb
\begin{verbatim}
>>> eng2sp['one'] = 'uno'
\end{verbatim}
\afterverb
%
This line creates an item that maps from the key
{\tt 'one'} to the value {\tt 'uno'}.  If we print the
dictionary again, we see a key-value pair with a colon
between the key and value:

\beforeverb
\begin{verbatim}
>>> print eng2sp
{'one': 'uno'}
\end{verbatim}
\afterverb
%
This output format is also an input format.  For example,
you can create a new dictionary with three items:

\beforeverb
\begin{verbatim}
>>> eng2sp = {'one': 'uno', 'two': 'dos', 'three': 'tres'}
\end{verbatim}
\afterverb
%
But if you print {\tt eng2sp}, you might be surprised:

\beforeverb
\begin{verbatim}
>>> print eng2sp
{'one': 'uno', 'three': 'tres', 'two': 'dos'}
\end{verbatim}
\afterverb
%
The key-value pairs are not in order, but that's not a problem because
the elements of a dictionary are never indexed with integer indices.
Instead, you use the keys to look up the corresponding values:

\beforeverb
\begin{verbatim}
>>> print eng2sp['two']
'dos'
\end{verbatim}
\afterverb
%
The key {\tt 'two'} always maps to the value {\tt 'dos'} so the order
of the items doesn't matter.

If the key isn't in the dictionary, you get an exception:

\index{exception!KeyError}
\index{KeyError}

\beforeverb
\begin{verbatim}
>>> print eng2sp['four']
KeyError: 'four'
\end{verbatim}
\afterverb
%
The {\tt len} function works on dictionaries; it returns the
number of key-value pairs:

\beforeverb
\begin{verbatim}
>>> len(eng2sp)
3
\end{verbatim}
\afterverb
%
The {\tt in} operator works on dictionaries; it tells you whether
something appears as a {\em key} in the dictionary (appearing
as a value is not good enough).

\beforeverb
\begin{verbatim}
>>> 'one' in eng2sp
True
>>> 'uno' in eng2sp
False
\end{verbatim}
\afterverb
%
To see whether something appears as a value in a dictionary, you
can use the method {\tt values}, which returns the values as
a list, and then use the {\tt in} operator:

\beforeverb
\begin{verbatim}
>>> vals = eng2sp.values()
>>> 'uno' in vals
True
\end{verbatim}
\afterverb
%
The {\tt in} operator uses different algorithms for lists
and dictionaries.  For lists, it uses a
search algorithm, as in Section~\ref{find}.  As the list gets longer,
the search time gets longer in direct proportion.
For dictionaries, Python uses an algorithm called a {\bf hashtable}
that has a remarkable property: the {\tt in} operator takes
about the same amount of time no matter how many items there 
are in a dictionary.  I won't explain how that's possible, but
you can look it up. 


\section{Dictionary as a set of counters}
\label{histogram}

Suppose you are given a string and you want to count how many
times each letter appears.  There are several ways you could do it:

\begin{enumerate}

\item You could create 26 variables, one for each letter of the
alphabet.  Then you could traverse the string and, for each
character, increment the corresponding counter, probably using
a chained conditional.

\item You could create a list with 26 elements.  Then you could
convert each character to a number (using the built-in function
{\tt ord}), use the number as an index into the list, and increment
the appropriate counter.

\item You could create a dictionary with characters as keys
and counters as the corresponding values.  The first time you
see a character, you would add an item to the dictionary.  After
that you would increment the value of an existing item.

\end{enumerate}

Each of these options performs the same computation, but each
of them implements that computation in a different way.

An {\bf implementation} is a way of performing a computation;
some implementations are better than others.  For example,
an advantage of the dictionary implementation is that we don't
have to know ahead of time which letters appear in the string
and we only have to make room for the letters that do appear.

Here is what the code might look like:

\beforeverb
\begin{verbatim}
def histogram(s):
    d = {}
    for c in s:
        if c not in d:
            d[c] = 1
        else:
            d[c] += 1
    return d
\end{verbatim}
\afterverb
%
The name of the function is {\bf histogram}, which is a statistical
term for a set of counters (or frequencies).

The first line of the
function creates an empty dictionary.  The {\tt for} loop traverses
the string.  Each time through the loop, if the character {\tt c} is
not in the dictionary, we create a new item with key {\tt c} and the
initial value 1 (since we have seen this letter once).  If {\tt c} is
already in the dictionary we increment {\tt d[c]}.

\index{histogram}

Here's how it works:

\beforeverb
\begin{verbatim}
>>> h = histogram('brontosaurus')
>>> print h
{'a': 1, 'b': 1, 'o': 2, 'n': 1, 's': 2, 'r': 2, 'u': 2, 't': 1}
\end{verbatim}
\afterverb
%
The histogram indicates that the letters {\tt 'a'} and {\tt 'b'}
appear once each; {\tt 'o'} appears twice, and so on.

\begin{ex}
Dictionaries have a method called {\tt get} that takes a key
and a default value.  If the key appears in the dictionary,
{\tt get} returns the corresponding value; otherwise it returns
the default value.  For example:

\beforeverb
\begin{verbatim}
>>> h = histogram('a')
>>> print h
{'a': 1}
>>> h.get('a', 0)
1
>>> h.get('b', 0)
0
\end{verbatim}
\afterverb
%
Use {\tt get} to write {\tt histogram} more concisely.  You
should be able to eliminate the {\tt if} statement.
\end{ex}


\section{Looping and dictionaries}
\index{dictionary!looping}

If you use a dictionary in a {\tt for} statement, it traverses
the keys of the dictionary.  For example, {\tt print\_hist}
prints each key and the corresponding value:

\beforeverb
\begin{verbatim}
def print_hist(h):
    for c in h:
        print c, h[c]
\end{verbatim}
\afterverb
%
Here's what the output looks like:

\beforeverb
\begin{verbatim}
>>> h = histogram('parrot')
>>> print_hist(h)
a 1
p 1
r 2
t 1
o 1
\end{verbatim}
\afterverb
%
Again, the keys are in no particular order.

\begin{ex}
Dictionaries have a method called {\tt keys} that returns
the keys of the dictionary, in no particular order, as a list.

Modify {\tt print\_hist} to print the keys and their values
in alphabetical order, using {\tt keys} and {\tt sort}.
\end{ex}



\section{Reverse lookup}
\index{lookup, dictionary}
\index{reverse lookup, dictionary}

Given a dictionary {\tt d} and a key {\tt k}, it is easy to
find the corresponding value {\tt v = d[k]}.  This operation
is called a {\bf lookup}.

But what if you have {\tt v} and you want to find {\tt k}?
You have two problems: first, there might be more than one
key that maps to the value {\tt v}.  Depending on the application,
you might be able to pick one, or you might have to make
a list that contains all of them.  Second, there is no
simple syntax to do a {\bf reverse lookup}; you have to search.

Here is a function that takes a value and returns the first
key that maps to that value:

\beforeverb
\begin{verbatim}
def reverse_lookup(d, v):
    for k in d:
        if d[k] == v:
            return k
    raise ValueError
\end{verbatim}
\afterverb
%
This function is yet another example of the search pattern we have
seen before, but it uses a feature we haven't seen before, {\tt raise}.
The {\tt raise} statement causes an exception; in this case it
causes a {\tt ValueError}, which generally indicates that there
is something wrong with the value of a parameter.

\index{search}
\index{pattern!search}
\index{raise}
\index{statement!raising}
\index{exception}
\index{ValueError}

If we get to the end of the loop, that means {\tt v}
doesn't appear in the dictionary as a value, so we raise an
exception.

Here is an example of a successful reverse lookup:

\beforeverb
\begin{verbatim}
>>> h = histogram('parrot')
>>> k = reverse_lookup(h, 2)
>>> print k
r
\end{verbatim}
\afterverb
%
And an unsuccessful one:

\beforeverb
\begin{verbatim}
>>> k = reverse_lookup(h, 3)
Traceback (most recent call last):
  File "<stdin>", line 1, in ?
  File "<stdin>", line 5, in reverse_lookup
ValueError
\end{verbatim}
\afterverb
%
The result when you raise an exception is the same as when
Python raises one: it prints a traceback and an error message.

\index{optional argument}
\index{argument!optional}

The {\tt raise} statement takes a detailed error message as an
optional argument.  For example:

\beforeverb
\begin{verbatim}
>>> raise ValueError, 'value does not appear in the dictionary'
Traceback (most recent call last):
  File "<stdin>", line 1, in ?
ValueError: value does not appear in the dictionary
\end{verbatim}
\afterverb
%
A reverse lookup is much slower than a forward lookup; if you
have to do it often, or if the dictionary gets big, the performance
of your program will suffer.

\begin{ex}
Modify {\tt reverse\_lookup} so that it builds and returns a list
of {\em all} keys that map to {\tt v}, or an empty list if there
are none.
\end{ex}


\section{Dictionaries and lists}

Lists can appear as values in a dictionary.  For example, if you
were given a dictionary that maps from letters to frequencies, you
might want to invert it; that is, create a dictionary that maps
from frequencies to letters.  Since there might be several letters
with the same frequency, each value in the inverted dictionary
should be a list of letters.

Here is a function that inverts a dictionary:


\beforeverb
\begin{verbatim}
def invert_dict(d):
    inv = {}
    for key in d:
        val = d[key]
        if val not in inv:
            inv[val] = [key]
        else:
            inv[val].append(key)
    return inv
\end{verbatim}
\afterverb
%
Each time through the loop, {\tt key} gets a key from {\tt d} and {\tt
val} gets the corresponding value.  If {\tt val} is not in {\tt inv},
that means we haven't seen it before, so we create a new item and
initialize it with a {\bf singleton} (a list that contains a
single element).  Otherwise we have seen this value before, so we
append the corresponding key to the list.

Here is an example:

\beforeverb
\begin{verbatim}
>>> hist = histogram('parrot')
>>> print hist
{'a': 1, 'p': 1, 'r': 2, 't': 1, 'o': 1}
>>> inv = invert_dict(hist)
>>> print inv
{1: ['a', 'p', 't', 'o'], 2: ['r']}
\end{verbatim}
\afterverb
%
And here is a diagram showing {\tt hist} and {\tt inv}:

\index{state diagram}
\index{diagram!state}

\beforefig
\centerline{\includegraphics{figs/dict1.eps}}
\afterfig

A dictionary is represented as a box with the type {\tt dict} above it
and the key-value pairs inside.  If the values are integers, floats or
strings, I usually draw them inside the box, but I usually draw lists
outside the box, just to keep the diagram simple.

Lists can be values in a dictionary, as this example shows, but they
cannot be keys.  Here's what happens if you try:

\beforeverb
\begin{verbatim}
>>> t = [1, 2, 3]
>>> d = {}
>>> d[t] = 'oops'
Traceback (most recent call last):
  File "<stdin>", line 1, in ?
TypeError: list objects are unhashable
\end{verbatim}
\afterverb
%
I mentioned earlier that a dictionary is implemented using
a hashtable and that means that the keys have to be {\bf hashable}.

\index{hash function}
\index{hashable}

A {\bf hash} is a function that takes a value (of any kind)
and returns an integer.  Dictionaries uses these integers,
called hash values, to store and look up key-value pairs.

This system works fine if the keys are immutable.  But if the
keys are mutable, like lists, bad things happen.  For example,
when you create a key-value pair, Python hashes the key and 
stores it in the corresponding location.  If you modify the
key and then hash it again, it would go to a different location.
In that case you might have two entries for the same key,
or you might not be able to find a key.  Either way, the
dictionary wouldn't work correctly.

That's why the keys have to be hashable, and why mutable types like
lists aren't.  The simplest way to get around this limitation is to
use tuples, which we will see in the next chapter.

Since dictionaries are mutable, they can't be used as keys,
but they {\em can} be used as values.



\section{Hints}
\index{hint}
\index{Fibonacci function}

If you played with the {\tt fibonacci} function from
Section~\ref{one more example}, you might have noticed that the bigger
the argument you provide, the longer the function takes to run.
Furthermore, the run time increases very quickly.

To understand why, consider this {\bf call graph} for
{\tt fibonacci} with {\tt n=4}:

\beforefig
\centerline{\includegraphics[height=2in]{figs/fibonacci.eps}}
\afterfig

A call graph shows a set function frames, with lines connecting each
frame to the frames of the functions it calls.  At the top of the
graph, {\tt fibonacci} with {\tt n=4} calls {\tt fibonacci} with {\tt
n=3} and {\tt n=2}.  In turn, {\tt fibonacci} with {\tt n=3} calls
{\tt fibonacci} with {\tt n=2} and {\tt n=1}.  And so on.

\index{function frame}
\index{frame}
\index{call graph}

Count how many times {\tt fibonacci(0)} and {\tt fibonacci(1)} are
called.  This is an inefficient solution to the problem, and it gets
worse as the argument gets bigger.

One solution is to keep track of values that have already been
computed by storing them in a dictionary.  A previously computed value
that is stored for later use is called a {\bf hint}.  Here is
an implementation of {\tt fibonacci} using hints:

\beforeverb
\begin{verbatim}
previous = {0:0, 1:1}

def fibonacci(n):
    if n in previous:
        return previous[n]

    res = fibonacci(n-1) + fibonacci(n-2)
    previous[n] = res
    return res
\end{verbatim}
\afterverb
%
{\tt previous} keeps track of the Fibonacci
numbers we already know.  We start with only
two items: 0 maps to 0 and 1 maps to 1.

Whenever {\tt fibonacci} is called, it checks {\tt previous}.
If the result is already there, it can return
immediately.  Otherwise it has to 
compute the new value, add it to the dictionary, and return it.

{\tt previous} is created outside the function, so it belongs
to the special frame called {\tt \_\_main\_\_}.  Variables
in {\tt \_\_main\_\_} are sometimes called {\bf global} because
they can be accessed from any function.  Unlike local variables,
which disappear when their function ends, global variables 
persist from one function call to the next.

Using this version of {\tt fibonacci}, you can compute
{\tt fibonacci(40)} in an eyeblink.  But if you compute
{\tt fibonacci(50)}, you get:

\beforeverb
\begin{verbatim}
>>> fibonacci(50)
20365011074L
\end{verbatim}
\afterverb
%
The {\tt L} at the end of the result indicates that the result
is too big to fit into a Python integer.  Python
converted it to a long integer.


\section{Long integers}
\index{long integer}
\index{data type!long integer}
\index{type!long}
\index{integer!long}

Python provides a type called {\tt long} that can handle any size
integer.  There are two ways to create a {\tt long} value.  One is
to write an integer with a capital {\tt L} at the end:

\beforeverb
\begin{verbatim}
>>> type(1L)
<type 'long'>
\end{verbatim}
\afterverb
%
The other is to use the {\tt long} function to convert a value.  
{\tt
long} can accept any numerical type and even strings of digits:

\index{type coercion}
\index{coercion!type}

\beforeverb
\begin{verbatim}
>>> long(1)
1L
>>> long(3.1415)
3L
>>> long('42')
42L
\end{verbatim}
\afterverb
%
The mathematical operators work on long integers, and the function
in the {\tt math} module, too, so in general any code that
works with {\tt int} will also work with {\tt long}.

Any time the result of a computation is too big to be represented with
an integer, Python converts the result as a long integer:

\beforeverb
\begin{verbatim}
>>> 1000 * 1000
1000000
>>> 100000 * 100000
10000000000L
\end{verbatim}
\afterverb
%
In the first case the result has type {\tt int}; in the
second case it is {\tt long}.


\section{Debugging}

As you work with bigger datasets it can become unwieldy to
debug by printing and checking data by hand.  Here are some
suggestions for debugging large datasets:

\begin{description}

\item[Scale down the input:] If possible, reduce the size of the
dataset.  For example if the program reads a text file, start with
just the first 10 lines, or with the smallest example you can find.
You can either edit the files themselves, or (better) modify the
program so it reads only the first {\tt n} lines.

If there is an error, you can reduce {\tt n} to the smallest
value that manifests the error, and then increase it gradually
as you find and correct errors.

\item[Check summaries and types:] Instead of printing and checking the
entire dataset, consider printing summaries of the data: for example,
the number of items in a dictionary or the total of a list of numbers.

A common cause of run-time errors is a value that is not the right
type.  For debugging this kind of error, it is often enough to print
the type of a value, which is often smaller than the value itself.

\item[Write self-checks:]  Sometimes you can write code to check
for errors automatically.  For example, if you are computing the
average of a list of number, you could check that the result is
not greater than the largest element in the list or less than
the smallest.  This is called a ``sanity check'' because it detects
results that are ``insane.''

Another kind of check compares the results of two different
computations to see if they are consistent.  This is called a
``consistency check.''

\end{description}

  



\section{Glossary}

\begin{description}

\item[dictionary:] A mapping from a set of keys to their
corresponding values.
\index{dictionary}

\item[key-value pair:] The representation of the mapping from
a key to a value.
\index{key-value pair}

\item[item:] Another name for a key-value pair.
\index{item!dictionary}

\item[key:] An object that appears in a dictionary as the
first part of a key-value pair.
\index{key}

\item[value:] An object that appears in a dictionary as the
second part of a key-value pair.  This is more specific than
our previous use of the word ``value.''
\index{value}

\item[implementation:] A way of performing a computation.
\index{implementation}

\item[hashtable:] The algorithm used to implement Python
dictionaries.
\index{hashtable}

\item[hash function:] A function used by a hashtable to compute the
location for a key.
\index{hash function}

\item[hashable:] A type that has a hash function.  Immutable
types like integers,
floats and strings are hashable; mutable types like lists and
dictionaries are not.
\index{hashable}

\item[lookup:] A dictionary operation that takes a key and finds
the corresponding value.
\index{lookup}

\item[reverse lookup:] A dictionary operation that takes a value and finds
one or more keys that map to it.
\index{reverse lookup}

\item[singleton:] A list (or other sequence) with a single element.
\index{singleton}

\item[call graph:] A diagram that shows every frame created during
the execution of a program, with an arrow from each caller to
each callee. 
\index{call graph}
\index{diagram!call graph}

\item[histogram:] A set of counters.
\index{histogram}

\item[hint:] A computed value stored to avoid unnecessary future 
computation.
\index{hint}

\item[global variable:]  A variable defined outside a function.  Global
variables can be accessed from any function.
\index{global variable}



\end{description}

\section{Exercises}

\begin{ex}
\label{anagram}
Two words are anagrams if you can rearrange the letters from one
to spell the other.  Write a function called {\tt is\_anagram}
that takes two strings and returns True if they are anagrams.
\end{ex}

\begin{ex}
Write a function named {\tt has\_duplicates} that takes a list
as a parameter and that returns {\tt True} if there is any object
that appears more than once in the list, and {\tt False} otherwise.
\end{ex}

\begin{ex}
\label{exrotatepairs}
Write a function that takes the file name of a word list (see
Section~\ref{wordlist}) as a parameter and searches for all pairs of
words that are rotations of each other.  The function
{\tt rotate\_word} is described in Exercise~\ref{exrotate}.
\end{ex}





\chapter{Tuples}
\label{tuplechap}
\index{tuple}

\section{Tuples are immutable}
\index{tuple}
\index{data type!tuple}
\index{type!tuple}
\index{data type!immutable}

A tuple is a sequence of values.  The values can be any type, and
they are indexed by integers, so in that respect tuples are a lot
like lists.  The important difference is that tuples are immutable.

\index{mutable}
\index{immutable}

Syntactically, a tuple is a
comma-separated list of values:

\beforeverb
\begin{verbatim}
>>> tuple = 'a', 'b', 'c', 'd', 'e'
\end{verbatim}
\afterverb
%
Although it is not necessary, it is common to enclose tuples in
parentheses:

\beforeverb
\begin{verbatim}
>>> t = ('a', 'b', 'c', 'd', 'e')
\end{verbatim}
\afterverb
%
To create a tuple with a single element, you have to include the final
comma:

\beforeverb
\begin{verbatim}
>>> t1 = ('a',)
>>> type(t1)
<type 'tuple'>
\end{verbatim}
\afterverb
%
Without the comma, Python treats {\tt ('a')} as a string in
parentheses:

\beforeverb
\begin{verbatim}
>>> t2 = ('a')
>>> type(t2)
<type 'str'>
\end{verbatim}
\afterverb
%
Another way to create a tuple is the function {\tt tuple}.  With
no argument, it creates an empty tuple:

\beforeverb
\begin{verbatim}
>>> t = tuple()
>>> print t
()
\end{verbatim}
\afterverb
%
If the argument is a sequence (string, list or tuple), the result
is a tuple with the elements of the sequence:

\beforeverb
\begin{verbatim}
>>> t = tuple('lupins')
>>> print t
('l', 'u', 'p', 'i', 'n', 's')
\end{verbatim}
\afterverb
%
Most list operators also work on tuples.  The bracket operator
indexes an element:

\beforeverb
\begin{verbatim}
>>> t = ('a', 'b', 'c', 'd', 'e')
>>> print t[0]
'a'
\end{verbatim}
\afterverb
%
And the slice operator selects a range of elements.

\beforeverb
\begin{verbatim}
>>> print t[1:3]
('b', 'c')
\end{verbatim}
\afterverb
%
But if you try to modify one of the elements of the tuple, you get
an error:

\index{exception!TypeError}
\index{TypeError}

\beforeverb
\begin{verbatim}
>>> t[0] = 'A'
TypeError: object doesn't support item assignment
\end{verbatim}
\afterverb
%
You can't modify the elements of a tuple, but you can replace
one tuple with another:

\beforeverb
\begin{verbatim}
>>> t = ('A',) + t[1:]
>>> print t
('A', 'b', 'c', 'd', 'e')
\end{verbatim}
\afterverb
%

\section{Tuple assignment}
\label{tuple assignment}
\index{tuple assignment}
\index{assignment!tuple}

It is often useful to swap the values of two variables.
With conventional assignments, you have to use a temporary
variable.  For example, to swap {\tt a} and {\tt b}:

\beforeverb
\begin{verbatim}
>>> temp = a
>>> a = b
>>> b = temp
\end{verbatim}
\afterverb
%
This solution is cumbersome; {\bf tuple assignment} is more elegant:

\beforeverb
\begin{verbatim}
>>> a, b = b, a
\end{verbatim}
\afterverb
%
The left side is a tuple of variables; the right side is a tuple of
expressions.  Each value is assigned to its respective variable.  
All the expressions on the right side are evaluated before any
of the assignments.

The number of variables on the left and the number of
values on the right have to be the same:

\index{exception!ValueError}
\index{ValueError}

\beforeverb
\begin{verbatim}
>>> a, b = 1, 2, 3
ValueError: too many values to unpack
\end{verbatim}
\afterverb
%
More generally, the right side can be any kind of sequence
(string, list or tuple).  For example, to split an email address
into a user name and a domain, you could write:

\beforeverb
\begin{verbatim}
>>> addr = 'monty@python.org'
>>> uname, domain = addr.split('@')
\end{verbatim}
\afterverb
%
The return value from {\tt split} is a list with two elements;
the first element is assigned to {\tt uname}, the second to
{\tt domain}.

\beforeverb
\begin{verbatim}
>>> print uname
monty
>>> print domain
python.org
\end{verbatim}
\afterverb
%

\section{Tuples as return values}
\index{tuple}
\index{value!tuple}
\index{return value!tuple}
\index{function!tuple as return value}

Strictly speaking, a function can only return one value, but
if the value is a tuple, the effect is the same as returning
multiple values.  For example, if you want to divide two integers
and compute the quotient and remainder, it is inefficient to
compute {\tt x/y} and then {\tt x\%y}.  It is better to compute
them both at the same time.

\index{divmod}

The built-in function {\tt divmod} takes two arguments and
returns a tuple of two values, the quotient and remainder.
You can store the result as a tuple:

\beforeverb
\begin{verbatim}
>>> t = divmod(7, 3)
>>> print t
(2, 1)
\end{verbatim}
\afterverb
%
Or use tuple assignment to store the elements separately:

\beforeverb
\begin{verbatim}
>>> quot, rem = divmod(7, 3)
>>> print quot
2
>>> print rem
1
\end{verbatim}
\afterverb
%
Here is an example of a function that returns a tuple:

\beforeverb
\begin{verbatim}
def min_max(t):
    return min(t), max(t)
\end{verbatim}
\afterverb
%
{\tt max} and {\tt min} are built-in functions that find
the largest and smallest elements of a sequence.  {\tt max\_min}
computes both and returns a tuple of two values.

\index{max}
\index{min}


\section{Lists and tuples}
\index{list}

{\tt zip} is a built-in function that takes two or more 
sequences and ``zips'' them into a list of tuples, where
each tuple contains one element from each sequence.

This example zips a string and a list:

\beforeverb
\begin{verbatim}
>>> s = 'abc'
>>> t = [0, 1, 2]
>>> zip(s, t)
[('a', 0), ('b', 1), ('c', 2)]
\end{verbatim}
\afterverb
%
The result is a list of tuples where each tuple contains
a character from the string and the corresponding element from
the list.

If the sequences are not the same length, the result gets the
length of the shorter one.

\beforeverb
\begin{verbatim}
>>> zip('Anne', 'Elk')
[('A', 'E'), ('n', 'l'), ('n', 'k')]
\end{verbatim}
\afterverb
%
You can use tuple assignment to traverse a list of tuples:

\beforeverb
\begin{verbatim}
t = [('a', 0), ('b', 1), ('c', 2)]
for letter, number in t:
    print number, letter
\end{verbatim}
\afterverb
%
Each time through the loop, Python selects the next tuple in
the list and assigns the elements to {\tt letter} and 
{\tt number}.  The output of this loop is:

\beforeverb
\begin{verbatim}
0 a
1 b
2 c
\end{verbatim}
\afterverb
%
If you combine {\tt zip}, {\tt for} and tuple assignment, you get a
standard idiom for traversing two (or more) sequences at the same
time.  For example, {\tt has\_match} takes two sequences, {\tt t1} and
{\tt t2}, and returns {\tt True} if there is an index {\tt i}
such that {\tt t1[i] == t2[i]}:

\beforeverb
\begin{verbatim}
def has_match(t1, t2):
    for x, y in zip(t1, t2):
        if x == y:
            return True
    return False
\end{verbatim}
\afterverb
%
If you need to traverse the elements of a sequence and their
indices, you can use the built-in function {\tt enumerate}:

\beforeverb
\begin{verbatim}
for index, element in enumerate('abc'):
    print index, element
\end{verbatim}
\afterverb
%
The output of this loop is:

\beforeverb
\begin{verbatim}
0 a
1 b
2 c
\end{verbatim}
\afterverb
%
Again.


\section{Dictionaries and tuples}
\index{dictionary}

Dictionaries have a method called {\tt items} that returns
a list of tuples, where each tuple is a key-value pair.

\beforeverb
\begin{verbatim}
>>> d = {'a':0, 'b':1, 'c':2}
>>> t = d.items()
>>> print t
[('a', 0), ('c', 2), ('b', 1)]
\end{verbatim}
\afterverb
%
As you should expect from a dictionary, the items are in no
particular order.

Conversely, you can use a list of tuples to initialize
a new dictionary:

\beforeverb
\begin{verbatim}
>>> t = [('a', 0), ('c', 2), ('b', 1)]
>>> d = dict(t)
>>> print d
{'a': 0, 'c': 2, 'b': 1}
\end{verbatim}
\afterverb
%
Combining this feature with {\tt zip} yields a concise way
to create a dictionary:

\beforeverb
\begin{verbatim}
>>> d = dict(zip('abc', range(3)))
>>> print d
{'a': 0, 'c': 2, 'b': 1}
\end{verbatim}
\afterverb
%
The dictionary method {\tt update} also takes a list of tuples
and adds them, as key-value pairs, to an existing dictionary.

Combining {\tt items}, tuple assignment and {\tt for}, you
get the idiom for traversing the keys and values of a dictionary:

\beforeverb
\begin{verbatim}
for key, value in d.items():
    print value, key
\end{verbatim}
\afterverb
%
The output of this loop is:

\beforeverb
\begin{verbatim}
0 a
2 c
1 b
\end{verbatim}
\afterverb
%
Again.

It is common to use tuples as keys in dictionaries (primarily because
you can't use lists).  For example, a telephone directory might map
from last-name, first-name pairs to telephone numbers.  Assuming
that we have defined {\tt last}, {\tt first} and {\tt number}, we
could write:

\beforeverb
\begin{verbatim}
directory[last,first] = number
\end{verbatim}
\afterverb
%
The expression in brackets is a tuple.  We could use tuple
assignment to traverse this dictionary.

\beforeverb
\begin{verbatim}
for last, first in directory:
    print first, last, directory[last,first]
\end{verbatim}
\afterverb
%
This loop traverses the keys in {\tt directory}, which are tuples.  It
assigns the elements of each tuple to {\tt last} and {\tt first}, then
prints the name and corresponding telephone number.

There are two ways to represent tuples in a state diagram.  The more
detailed version shows the indices and elements just as they appear in
a list.  For example, the tuple {\tt ('Cleese', 'John')} would appear:

\index{state diagram}
\index{diagram!state}

\beforefig
\centerline{\includegraphics{figs/tuple1.eps}}
\afterfig

But in a larger diagram you might want to leave out the
details.  For example, a diagram of the telephone directory might
appear:

\beforefig
\centerline{\includegraphics{figs/dict2.eps}}
\afterfig

Here the tuples are shown using Python syntax as a graphical
shorthand.

The telephone number in the diagram is the complaints line for the
BBC, so please don't call it.



\section{Sorting tuples}

The comparison operators work with tuples and other sequences;
Python starts by comparing the first element from each
sequence.  If they are equal, it goes on to the next elements,
and so on, until it finds elements that differ.  Subsequent
elements are not considered (even if they are really big).

\beforeverb
\begin{verbatim}
>>> (0, 1, 2) < (0, 3, 4)
True
>>> (0, 1, 2000000) < (0, 3, 4)
True
\end{verbatim}
\afterverb
%
The {\tt sort} function works the same way.  It sorts 
primarily by first element, but in the case of a tie, it sorts
by second element, and so on.  Here is an example that sorts
and prints the key-value pairs of a dictionary:

\beforeverb
\begin{verbatim}
>>> d = {'a': 0, 'c': 2, 'b': 1}
>>> t = d.items()
>>> t.sort()
>>> print t
[('a', 0), ('b', 1), ('c', 2)]
\end{verbatim}
\afterverb
%
To sort by value (rather than key), you can build a list of
value-key pairs.  One way to do that is to traverse the
dictionary items and append tuples onto a list:

\beforeverb
\begin{verbatim}
def value_key_pairs(d):
    res = []
    for key, value in d.items():
        res.append((value, key))
    return res
\end{verbatim}
\afterverb
%
The argument for {\tt append} has two sets of parentheses:
one because its an argument and the other because it is a tuple.

\begin{ex}
Draw a diagram that shows the final state of {\tt value\_key\_pairs}
with {\tt d = \{'a': 0, 'c': 2, 'b': 1\}}.
\end{ex}


\section{Sequences of sequences}
\index{sequence}

I have focused on lists of tuples, but almost all of the examples in
this chapter also work with lists of lists, tuples of tuples, and
tuples of lists.  To avoid enumerating the possible combinations, it
is sometimes easier to talk about sequences of sequences.

In many contexts, the different kinds of sequences (strings, lists and
tuples) can be used interchangeably.  So how and why do you choose one
over the others.

To start with the obvious, strings are more limited than other
sequences because the elements have to be characters.  They are
also immutable.  If you need the ability to change the characters
in a string (as opposed to creating a new string), you might
want to use a list of characters instead.

Lists are more common than tuples, mostly because they are mutable.
But there are a few cases where you might prefer tuples:

\begin{itemize}

\item In some contexts, like a {\tt return} statement, it is
syntactically simpler to create a tuple than a list.

\item If you want to use a sequence as a dictionary key, you
have to use an immutable type like a tuple or string.

\item If you are passing a sequence as an argument to a function,
using tuples reduces the potential for unexpected behavior
due to aliasing.

\end{itemize}

Because tuples are immutable, they don't provide methods
like {\tt sort} and {\tt reverse}, which modify existing lists.
But Python provides the built-in functions {\tt sorted}
and {\tt reversed}, which take any sequence as a parameter
and return a new list with the same elements in a different
order.


%\section{Debugging}

%Using immutable types to eliminate aliasing.

%The best way to avoid a bug is to make it impossible.




\section{Glossary}

\begin{description}

\item[tuple:] An immutable sequence of elements.
\index{tuple}

\item[tuple assignment:] An assignment with a sequence on the
right side and a tuple of variables on the left.  The right
side is evaluated and then its elements are assigned to the
variables on the left.
\index{tuple assignment}
\index{assignment!tuple}

\end{description}


\section{Exercises}

\begin{ex}
Write a function called {\tt most\_frequent} that takes a string
and prints the 3 most common letters in the string.
\end{ex}



\begin{ex}
\label{anagrams}

Write a program
that reads a word list from a file (see Section~\ref{wordlist}) and
prints all the sets of words that are anagrams.

Here is an example of what the output might look like:

\beforeverb
\begin{verbatim}
['deltas', 'desalt', 'lasted', 'salted', 'slated', 'staled']
['retainers', 'ternaries']
['generating', 'greatening']
['resmelts', 'smelters', 'termless']
\end{verbatim}
\afterverb
%
Hint: you might want to build a dictionary that maps from a
set of letters to a list of words that can be spelled with those
letters.  The question is, how can you represent the set of
letters in a way that can be used as a key?


% see anagram_sets.py

\end{ex}

\begin{ex}
Modify the previous program so that it prints the largest set
of anagrams first, followed by the second largest set, and so on.
\end{ex}




\chapter{Case study: data structure selection}

\section{DSU}
\label{DSU}
\index{DSU}

{\bf DSU} stands for ``decorate, sort, undecorate'' and refers to
a pattern that is often useful for sorting lists according to
some attribute of the elements.

For example, if you have a dictionary that maps from mothers
to lists of their children, you might want to sort the mothers
by their number of children.  Here is a function that does
that:

\beforeverb
\begin{verbatim}
def sort_by_children(mothers):
    t = []
    for mother, children in mothers.items():
       t.append((len(children), mother))

    t.sort()

    res = []
    for number, mother in t:
        res.append(mother)
    return res
\end{verbatim}
\afterverb
%
The first loop assigns each mother to {\tt mother} and each list of
children to {\tt children}.  It builds a list of tuples, where each
tuple is the number of children and a mother.

{\tt sort} compares the first element, number of children, first, and
only considers the second element to break ties.  The result is a list
of tuples sorted in increasing order by number of children.

The second loop traverses the list of tuples and builds a list of
mothers, sorted by parity (which in this context means number of
children).

This pattern is called ``decorate, sort, undecorate'' because
the first loop ``decorates'' the list of mothers with by pairing
each mother with her parity, and the last loop ``undecorates''
the sorted list by removing the parity information.


\section{Word frequency analysis}
\label{analysis}

As usual, you should at least attempt the following exercises
before you read my solutions.

\begin{ex}
Write a program that reads a file, breaks each line into
words, strips whitespace and punctuation from the words, and
converts them to lowercase.
\end{ex}


\begin{ex}
Go to Project Gutenberg (\url{gutenberg.net}) and download 
your favorite out-of-copyright book in plain text format.

Modify your program from the previous exercise to read the book
you downloaded, skip over the header information at the beginning
of the file, and process the rest of the words as before.

Then modify the program to count the total number of words in
the book, and the number of times each word is used.

Print the number of different words used in the book.  Compare
different books by different authors, written in different eras.
Which author uses the most extensive vocabulary?
\end{ex}


\begin{ex}
Modify the program from the previous exercise to print the
20 most frequently-used words in the book.
\end{ex}


\begin{ex}
Modify the previous program to read a word list (see
Section~\ref{wordlist}) and then print all the words in the book that
are not in the word list.  How many of them are typos?  How many of
them are common words that {\em should} be in the word list, and how
many of them are really obscure?
\end{ex}


\section{Random numbers}
\index{random number}
\index{number!random}

Most computer programs do the same thing every time they execute,
given the same inputs, so they are said to be {\bf deterministic}.
Determinism is usually a good thing, since we expect the same
calculation to yield the same result.  For some applications, though,
we want the computer to be unpredictable.  Games are an obvious
example, but there are more.

Making a program truly nondeterministic turns out to be not so easy,
but there are ways to make it at least seem nondeterministic.  One of
them is to use algorithms that generate
{\bf pseudorandom} numbers.  Pseudorandom numbers are not truly
random because they are generated by a deterministic computation,
but just by looking at the numbers it is all but impossible to
distinguish them from random.

The {\tt random} module provides functions that generate
pseudorandom numbers (which I will simply call ``random'' from
here on).

The function {\tt random} returns a random float
between 0.0 and 1.0 (including 0.0 but not 1.0).  Each time you
call {\tt random}, you get the next number in a long series.  To see a
sample, run this loop:

\beforeverb
\begin{verbatim}
import random

for i in range(10):
    x = random.random()
    print x
\end{verbatim}
\afterverb
%
The function {\tt randint} takes parameters {\tt low} and
{\tt high} and returns an integer between {\tt low} and
{\tt high} (including both).


\beforeverb
\begin{verbatim}
>>> random.randint(5, 10)
5
>>> random.randint(5, 10)
9
\end{verbatim}
\afterverb
%
To choose an element from a sequence at random, you can use
{\tt choice}:

\beforeverb
\begin{verbatim}
>>> t = [1, 2, 3]
>>> random.choice(t)
2
>>> random.choice(t)
3
\end{verbatim}
\afterverb
%
The {\tt random} module also provides functions to generate
random values from continuous distributions including
Gaussian, exponential, gamma, and a few more.

\begin{ex}
Write a function named {\tt choose\_from\_hist} that takes
a histogram as defined in Section~\ref{histogram} and returns a 
random value from the histogram, chosen with probability
in proportion to frequency.  For example, for this histogram:

\beforeverb
\begin{verbatim}
>>> t = ['a', 'a', 'b']
>>> h = histogram(t)
>>> print h
{'a': 2, 'b': 1}
\end{verbatim}
\afterverb
%
your function should {\tt 'a'} with probability $2/3$ and {\tt 'b'}
with probability $1/3$.
\end{ex}


\section{Word histogram}

Here is a program that reads a file and builds a histogram of the
words in the file:

\beforeverb
\begin{verbatim}
import string

def process_file(filename):
    h = {}
    fp = open(filename)
    for line in fp:
        process_line(line, h)
    return h

def process_line(line, h):
    line = line.replace('-', ' ')
    
    for word in line.split():
        word = word.strip(string.punctuation + string.whitespace)
        word = word.lower()

        h[word] = h.get(word, 0) + 1

hist = process_file('emma.txt')
\end{verbatim}
\afterverb
%
This
program reads {\tt emma.txt}, which contains the text of
{\em Emma} by Jane Austin.

{\tt process\_file} loops through the lines of the file,
passing them one at a time to {\tt process\_line}.  The histogram
{\tt h} is being used as an accumulator.

\index{accumulator}

{\tt process\_line} uses the string method {\tt replace} to replace
hyphens with spaces before using {\tt split} to break the line into a
list of strings.  It traverses the list of words and uses {\tt strip}
and {\tt lower} to remove punctuation and convert to lower case.  (It
is a shorthand to say that strings are ``converted;'' remember that
string are immutable, so methods like {\tt strip} and {\tt lower}
return new strings.)

Finally, {\tt process\_line} updates the histogram by creating a new
item or incrementing an existing one.

To count the total number of words in the file, we can add up
the frequencies in the histogram:

\beforeverb
\begin{verbatim}
def total_words(h):
    return sum(h.values())
\end{verbatim}
\afterverb
%
The number of different words is just the number of items in
the dictionary:

\beforeverb
\begin{verbatim}
def different_words(h):
    return len(h)
\end{verbatim}
\afterverb
%
Here is some code to print the results:

\beforeverb
\begin{verbatim}
print 'Total number of words:', total_words(hist)
print 'Number of different words:', different_words(hist)
\end{verbatim}
\afterverb
%
And the results:

\beforeverb
\begin{verbatim}
Total number of words: 161073
Number of different words: 7212
\end{verbatim}
\afterverb
%

\section{Most common words}

To find the most common words, we can apply the DSU pattern;
{\tt most\_common} takes a histogram and returns a list of
word-frequency tuples, sorted in reverse order by frequency:

\beforeverb
\begin{verbatim}
def most_common(h):
    t = []
    for key, value in h.items():
        t.append((value, key))

    t.sort()
    t.reverse()
    return t
\end{verbatim}
\afterverb
%
Here is a loop that prints the ten most common words:

\beforeverb
\begin{verbatim}
t = most_common(hist)
print 'The most common words are:'
for freq, word in t[0:10]:
    print word, '\t', freq
\end{verbatim}
\afterverb
%
And here are the results from {\em Emma}:

\beforeverb
\begin{verbatim}
The most common words are:
to      5242
the     5204
and     4897
of      4293
i       3191
a       3130
it      2529
her     2483
was     2400
she     2364
\end{verbatim}
\afterverb
%

\section{Optional arguments}

\index{optional argument}
\index{argument!optional}

We have seen built-in functions that take a variable number of
arguments.  For example, {\tt range} can take one, two,
or three arguments.

It is possible to write user-defined functions with optional
arguments, too.  For example, here is a function that prints
the most common words in a histogram

\beforeverb
\begin{verbatim}
def print_most_common(hist, num=10)
    t = most_common(hist)
    print 'The most common words are:'
    for freq, word in t[0:num]:
        print word, '\t', freq
\end{verbatim}
\afterverb

The first parameter is required; the second is optional.
The {\bf default value} of {\tt num} is 10.

If you only provide one argument:

\beforeverb
\begin{verbatim}
print_most_common(hist)
\end{verbatim}
\afterverb

{\tt num} gets the default value.  If you provide two arguments:

\beforeverb
\begin{verbatim}
print_most_common(hist, 20)
\end{verbatim}
\afterverb

{\tt num} gets the value of the argument instead.  In other
words, the optional argument {\bf overrides} the default value.

If a function has both required and optional parameters, all
the required parameters have to come first, followed by the
optional ones.


\section{Dictionary subtraction}

Finding the words from the book that are not in the word list
from {\tt words.txt} is a problem you might recognize as set
subtraction; that is, we want to find all the words from one
set (the words in the book) that are not in another set (the
words in the list).

{\tt subtract} takes dictionaries {\tt d1} and {\tt d2} and returns a
new dictionary that contains all the keys from {\tt d1} that are not
in {\tt d2}.  Since we don't really care about the values, we
set them all to None.


\beforeverb
\begin{verbatim}
def subtract(d1, d2):
    res = {}
    for key in d1:
        if key not in d2:
            res[key] = None
    return res
\end{verbatim}
\afterverb
%
To find the words in the book that are not in {\tt words.txt},
we can use {\tt process\_file} to build a histogram for
{\tt words.txt}, and then subtract:

\beforeverb
\begin{verbatim}
words = process_file('words.txt')
diff = subtract(hist, words)

print "The words in the book that aren't in the word list are:"
for word in diff.keys():
    print word,
\end{verbatim}
\afterverb
%
Here are some of the results from {\em Emma}:

\beforeverb
\begin{verbatim}
The words in the book that aren't in the word list are:
 rencontre jane's blanche woodhouses disingenuousness 
friend's venice apartment ...
\end{verbatim}
\afterverb
%
Some of these words are names and possessives.  Others, like
``rencontre,'' are no longer in common use.  But a few are common
words that should really be on the list!


\section{Random words}
\label{randomwords}

To choose a random word from the histogram, the simplest algorithm
it to build a list with multiple copies of each word, according
to the observed frequency, and then choose from the list:

\beforeverb
\begin{verbatim}
def random_word(h):
    t = []
    for word, freq in h.items():
        t.extend([word] * freq)

    return random.choice(t)
\end{verbatim}
\afterverb
%
The expression {\tt [word] * freq} creates a list with {\tt freq}
copies of the string {\tt word} (actually, to be more precise, the
elements are references to the same string).  The {\tt extend}
method is similar to {\tt append} except that the argument is
a sequence.

This algorithm works, but it is wildly inefficient; each time you
choose a random word, it rebuilds the list, which is as big as
the original book.

If you generate a series of words from the book, you can get a
sense of the vocabulary, but it probably won't make much sense:

\beforeverb
\begin{verbatim}
this the small regard harriet which knightley's it most things
\end{verbatim}
\afterverb
%
The next section is about generating random text that makes more
sense.

\section{Markov analysis}

A series of random words seldom makes a sentence because there
is no correlation between successive words.  For example, in
a real sentence you would expect an article like ``the'' to
be followed by an adjective or a noun, and probably not a verb
or adverb.

One way to measure these kinds of relationships is Markov
analysis, which characterizes, for a given sequence of words,
the probability of the word that comes next.  For example,
the song {\em Eric, the Half a Bee} begins:

\begin{quote}
Half a bee, philosophically,
Must, ipso facto, half not be.
But half the bee has got to be
Vis a vis, its entity. D'you see?

But can a bee be said to be
Or not to be an entire bee
When half the bee is not a bee
Due to some ancient injury?
\end{quote}

In this text,
the phrase ``half the'' is always followed by the word ``bee,''
but the phrase ``the bee'' might be followed by either
``has'' or ``is''.

The result of Markov analysis is a mapping from each prefix
(like ``half the'' and ``the bee'') to all possible suffixes
(like ``has'' and ``is'').

Given this mapping, you can generate a random text by
starting with any prefix and choosing at random from the
possible suffixes.  Next, you can combine the end of the
prefix and the new suffix to form the next prefix, and repeat.

For example, if you start with the prefix ``Half a,'' then the
next word has to be ``bee,'' because the prefix only appears
once in the text.  The next prefix is ``a bee,'' so the
next suffix might be ``philosophically,'' ``be'' or ``due.''

In this example the length of the prefix is always two, but
you can do Markov analysis with any prefix length.  The length
of the prefix is called the ``order'' of the analysis.

\begin{ex}
Write a program to read a text from a file and perform Markov
analysis.  The result should be a dictionary that maps from
prefixes to a collection of possible suffixes.  The collection
might be a list, tuple, or dictionary; it is up to you to make
an appropriate choice.  You can text your program with prefix
length two, but you should write the program in a way that makes
it easy to try other lengths.
\end{ex}

\begin{ex}
Add a function to the previous program to generate random text
based on the Markov analysis.  Here is an example from {\em Emma}
with prefix length 2:

\begin{quote}
He was very clever, be it sweetness or be angry, ashamed or only
amused, at such a stroke. She had never thought of Hannah till you
were never meant for me?" "I cannot make speeches, Emma:" he soon cut
it all himself.
\end{quote}

For this example, I left the punctuation attached to the words.
The result is almost syntactically correct, but not quite.
Semantically, it almost makes sense, in places, but not quite.

What happens if you increase the prefix length?  Does the random
text make more sense?
\end{ex}


\begin{ex}
Once your program is working, you might want to try a mash-up:
if you analyze text from two or more books, the random
text you generate will blend the vocabulary and phrases from
the sources in interesting ways.

\end{ex}


\section{Data structures}

Using Markov analysis to generate random text is fun, but there
is also a point to this exercise: data structure selection.

Lists, dictionaries and tuples, and their combinations, like
lists of tuples, are known as {\bf data structures}.  When you
design programs, one of the decision you have to make is
which data structures to use.

For example, in your solution to the previous exercises, you
had to choose:

\begin{itemize}

\item How to represent the prefixes.

\item How to represent the collection of possible suffixes.

\item How to represent the mapping from each prefix to
the collection of possible suffixes.

\end{itemize}

The last one is the easiest; the only mapping type we have
seen is a dictionary, so it is the natural choice.

For the prefixes, the most obvious options are string,
list of strings, or tuple of strings.  For the suffixes,
one option is a list another is a histogram (dictionary that
maps strings to integers).

How should you choose?  The first step is to think about
the operations you will need to implement for each data structure.
For the prefixes, we need to be able to remove words from
the beginning and add to the end.  For example, if the current
prefix is ``Half a,'' and the next word is ``bee,'' you need
to be able to form the next prefix, ``a bee.''

Your first choice might be a list, since it is easy to add
and remove elements, but we also need to be able to use the
prefixes as keys in a dictionary, so that rules out lists.
With tuples, you can't append or remove, but you can use
the addition operator to form a new tuple:

\beforeverb
\begin{verbatim}
def shift(prefix, word):
    return prefix[1:] + (word,)
\end{verbatim}
\afterverb
%
{\tt shift} takes a tuple of words, {\tt prefix}, and a string, 
{\tt word}, and forms a new tuple that has all the words
in {\tt prefix} except the first, and {\tt word} added to
the end.

For the collection of suffixes, the operations we need to
perform include adding a new suffix (or increasing the frequency
of an existing one), and choosing a random suffix.

Adding a new suffix is equally easy for the list implementation
or the histogram.  Choosing a random element from a list
is easy; choosing from a histogram is harder and not very
efficient (see Section~\ref{randomwords}).

So far we have been talking mostly about ease of implementation,
but there are other factors to consider in choosing data structures.
One is run time.  Sometimes there is a theoretical reason to expect
one data structure to be faster than other; for example, I mentioned
that the {\tt in} operator is faster for dictionaries than for lists,
at least when the number of elements is large.

But often you don't know ahead of time which implementation will
be faster.  One option is to implement both of them and see which
is better.  This approach is called {\bf benchmarking}.  A practical
alternative is to choose the data structure that is
easiest to implement, and then see if it is fast enough for the
intended application.  If so, there is no need to go on.  If not,
there are tools, like the {\tt profile} module, that can identify
the places in a program that take the most time.

\index{benchmarking}
\index{profile module}
\index{module!profile}

The other factor to consider is storage space.  For example, using a
histogram for the collection of suffixes might take less space because
you only have to store each word once, no matter how many times it
appears in the text.  In some cases, saving space can also make your
program run faster, and in the extreme, your program might not run at
all if you run out of memory.  But for many applications, space is a
secondary consideration after run time.

One final thought: in this discussion, I have implied that
we would use one data structure for both analysis and generation.  But
since these are separate phases, it would also be possible to use one
structure for analysis and then convert to another structure for
generation.  This would be a net win if the time saved during
generation exceeded the time spent in conversion.

%\section{Debugging}

%Scaling up and scaling down.

%If you are looking for a needle in a haystack, choose a small
%haystack.


\section{Glossary}

\begin{description}

\item[DSU:] Abbreviation of ``decorate-sort-undecorate,'' a processing
pattern that involves building a list of tuples, sorting, and (often)
extracting part of the result.
\index{DSU}

\item[deterministic:] Pertaining to a program that does the same
thing each time it runs, given the same inputs.
\index{deterministic}

\item[pseudorandom:] Pertaining to a sequence of numbers that appear
to be random, but are generated by a deterministic program.
\index{pseudorandom}

\item[default value:] The value given to an optional parameter if no
argument is provided.
\index{default value}

\item[override:] To replace a default value with an argument.
\index{override}

\item[data structure:] Any collection of values, including sequences
and dictionaries.
\index{data structure}

\item[benchmarking:] The process of choosing between data structures
by implementing alternatives and testing them on a sample of the
possible inputs.  
\index{benchmarking}

\end{description}



\chapter{Files}
\index{file}
\index{type!file}


\section{Persistence}

Most of the programs we have seen so far are transient in the
sense that they run for a short time and produce some output,
but when they end, their data disappears.  If you run the program
again, it starts with a clean slate.

Other programs are {\bf persistent}: they run for a long time
(or all the time); they keep at least some of their data
in non-volatile storage (a hard drive, for example); and
if they shut down and restart, they pick up where they left off.

Examples of persistent programs are operating systems, which
run pretty much whenever a computer is on, and web servers,
which run all the time, waiting for requests to come in on
the network.

One of the simplest ways for programs to maintain their data
is by reading and writing text files.  We have already seen
programs that read text files; in this chapters we will see programs
that write them.

An alternative is to store the state of the program in a database.
In this chapter I will present a simple database and a module,
{\tt pickle}, that makes it easy to store program data.


\section{Reading and writing}

A text file is a sequence of characters stored on a permanent
medium like a hard drive, flash memory, or CD-ROM.  
To read a file, you can use {\tt open} to create a file
object:

\beforeverb
\begin{verbatim}
>>> fin = open('words.txt')
>>> print fin
<open file 'words.txt', mode 'r' at 0xb7eb2380>
\end{verbatim}
\afterverb
%
Mode {\tt 'r'} means that this file is open for reading.
The file object provides several methods for reading data,
including {\tt readline}:

\beforeverb
\begin{verbatim}
>>> line = fin.readline()
>>> print line
aa
\end{verbatim}
\afterverb
%
The file object keeps track of where it is in the file,
so if you invoke {\tt readline} again, it picks up from where
it left off.  You can also use a file object in a for loop, as we saw
in Section~\ref{wordlist}.

To write a file, you have to create a file object with mode
{\tt 'w'} as a second parameter:

\beforeverb
\begin{verbatim}
>>> fout = open('output.txt', 'w')
>>> print fout
<open file 'output.txt', mode 'w' at 0xb7eb2410>
\end{verbatim}
\afterverb
%
If the file already exists, opening it in write mode clears out
the old data and starts fresh, so be careful!
If the file doesn't exist, a new one is created.

The {\tt write} method puts data into the file.

\beforeverb
\begin{verbatim}
>>> line1 = "This here's the wattle,\n"
>>> fout.write(line1)
\end{verbatim}
\afterverb
%
Again, the file object keeps track of where it is, so if
you call {\tt write} again, it add the new data to the end.

\beforeverb
\begin{verbatim}
>>> line2 = "the emblem of our land.\n"
>>> fout.write(line2)
\end{verbatim}
\afterverb
%
When you are done writing, you have to close the file.

\beforeverb
\begin{verbatim}
>>> fout.close()
\end{verbatim}
\afterverb
%

\section{Format operator}
\index{format operator}
\index{format string}
\index{operator!format}

% EDIT THIS SECTION!!!

The argument of {\tt write} has to be a string, so if we want
to put other values in a file, we have to convert them to
strings.  The easiest way to do that is with {\tt str}:

\beforeverb
\begin{verbatim}
>>> x = 52
>>> f.write (str(x))
\end{verbatim}
\afterverb
%
An alternative is to use the {\bf format operator}, {\tt \%}.  When
applied to integers, {\tt \%} is the modulus operator.  But
when the first operand is a string, {\tt \%} is the format operator.

The first operand is the {\bf format string}, and the second operand
is a tuple of expressions.  The result is a string that contains
the values of the expressions, formatted according to the format
string.

As an example, the {\bf format sequence} {\tt '\%d'} means that
the first expression in the tuple should be formatted as an
integer ({\tt d} stands for ``decimal''):

\beforeverb
\begin{verbatim}
>>> camels = 42
>>> '%d' % camels
'42'
\end{verbatim}
\afterverb
%
The result is the string {\tt '42'}, which is not to be confused
with the integer value {\tt 42}.

A format sequence can appear anywhere in the format string,
so you can embed a value in a sentence:

\beforeverb
\begin{verbatim}
>>> camels = 42
>>> 'I have spotted %d camels.' % camels
'I have spotted 42 camels.'
\end{verbatim}
\afterverb
%
The format sequence {\tt '\%g'} formats the next element in the tuple
as a floating-point number (don't ask why), and {\tt '\%s'} formats
the next item as a string:

\beforeverb
\begin{verbatim}
>>> 'In %d years I have spotted %g %s.' % (3, 0.1, 'camels')
'In 3 years I have spotted 0.1 camels.'
\end{verbatim}
\afterverb
%
By default, the floating-point format prints six decimal places.

The number of elements in the tuple has to match the number
of format sequences in the string.  Also, the types of the
elements have to match the format sequences:

\index{exception!TypeError}
\index{TypeError}

\beforeverb
\begin{verbatim}
>>> '%d %d %d' % (1, 2)
TypeError: not enough arguments for format string
>>> '%d' % 'dollars'
TypeError: illegal argument type for built-in operation
\end{verbatim}
\afterverb
%
In the first example, there aren't enough elements; in the
second, the element is the wrong type.

You can specify the number of digits as part of the format sequence.
For example, the sequence {\tt '\%8.2f'}
formats a floating-point number to be 8 characters long, with
2 digits after the decimal point:

\beforeverb
\begin{verbatim}
>>> '%8.2f' % 3.14159
'    3.14'
\end{verbatim}
\afterverb
%
The result takes up eight spaces with two
digits after the decimal point.  


\section{Filenames and paths}
\label{paths}

Files are organized into {\bf directories} (also called ``folders'').
Every running program has a ``current directory,'' which is the
default directory for most operations.  For example, when you create a
new file with {\tt open}, the new file goes in the current directory.
And when you open a file for reading, Python looks for it in the
current directory.

The module {\tt os} provides functions for working with files and
directories (``os'' stands for ``operating system'').  {\tt os.getcwd}
returns the name of the current directory:

\beforeverb
\begin{verbatim}
>>> import os
>>> cwd = os.getcwd()
>>> print cwd
/home/dinsdale
\end{verbatim}
\afterverb
%
{\tt cwd} stands for ``current working directory.''  The result in
this example is {\tt /home/dinsdale}, which is the home directory of a
user named {\tt dinsdale}.

A string like {\tt cwd} that identifies a file is called a {\bf path}.
A {\bf relative path} starts from the current directory;
an {\bf absolute path} starts from the topmost directory in the
file system.

The paths we have seen so far are simple filenames, so they are
relative to the current directory.  To find the absolute path to
a file, you can use {\tt abspath}, which is in the module {\tt os.path}.

\beforeverb
\begin{verbatim}
>>> os.path.abspath('memo.txt')
'/home/dinsdale/memo.txt'
\end{verbatim}
\afterverb
%
{\tt os.path.exists} checks
whether the file (or directory) specified by a path exists:

\beforeverb
\begin{verbatim}
>>> os.path.exists('memo.txt')
True
\end{verbatim}
\afterverb
%
If it exists, {\tt os.path.isdir} checks whether it's a directory:

\beforeverb
\begin{verbatim}
>>> os.path.isdir('memo.txt')
False
>>> os.path.isdir('music')
True
\end{verbatim}
\afterverb
%
Similarly, {\tt os.path.isfile} checks whether it's a file.

{\tt os.listdir} returns a list of the files (and other directories)
in the given directory:

\beforeverb
\begin{verbatim}
>>> os.listdir(cwd)
['music', 'photos', 'memo.txt']
\end{verbatim}
\afterverb
%
To demonstrate these functions, the following example
``walks'' through a directory, prints
the names of all the files, and calls itself recursively on
all the directories.

\beforeverb
\begin{verbatim}
def walk(dir):
    for name in os.listdir(dir):
        path = os.path.join(dir, name)

        if os.path.isfile(path):
            print path
        else:
            walk(path)
\end{verbatim}
\afterverb
%
{\tt os.path.join} takes a directory and a file name and joins
them into a complete path.  

\begin{ex}
Modify {\tt walk} so that instead of printing the names of
the files, it returns a list of names.
\end{ex}


\section{Catching exceptions}
\label{catch}

A lot of things can go wrong when you try to read and write
files.  If you try to open a file that doesn't exist, you get an
{\tt IOError}:

\index{exception!IOError}
\index{IOError}

\beforeverb
\begin{verbatim}
>>> fin = open('bad_file')
IOError: [Errno 2] No such file or directory: 'bad_file'
\end{verbatim}
\afterverb
%
If you don't have permission to access a file:

\beforeverb
\begin{verbatim}
>>> fout = open('/etc/passwd', 'w')
IOError: [Errno 13] Permission denied: '/etc/passwd'
\end{verbatim}
\afterverb
%
And if you try to open a directory for reading, you get

\beforeverb
\begin{verbatim}
>>> fin = open('/home')
IOError: [Errno 21] Is a directory
\end{verbatim}
\afterverb
%
To avoid these errors, you could use functions like {\tt os.path.exists}
and {\tt os.path.isfile}, but it would take a lot of time and code
to check all the possibilities (based on the last error message,
there are {\em at least} 21 things that can go wrong).

\index{exception!catching}
\index{{\tt try} statement}

It is better to go ahead and try, and deal with problems if they
happen, which is exactly what the {\tt try} statement does.  The
syntax is similar to an {\tt if} statement:

\beforeverb
\begin{verbatim}
try:    
    fin = open('bad_file')
    for line in fin:
        print line
    fin.close()
except:
    print 'Something went wrong.'
\end{verbatim}
\afterverb
%
Python starts by executing the {\tt try} clause.  If all goes
well, it skips the {\tt except} clause and proceeds.  If an
exception occurs, it jumps out of the {\tt try} clause and
executes the {\tt except} clause.

Handling an exception with a {\tt try} statement is called {\bf
catching} an exception.  In this example, the {\tt except} clause
prints an error message that is not very helpful.  In general,
catching an exception gives you a chance to fix the problem, or try
again, or at least end the program gracefully.


\section{Databases}

A {\bf database} is a file that is organized for storing data.
Most databases are organized like a dictionary in the sense
that they map from keys to values.  The biggest difference
is that the database is on disk (or other non-volatile storage),
so it persists after the program ends.

The module {\tt anydbm} provides an interface for creating
and updating database files.  As an example, I'll create a database
that contains captions for image files.

Opening a database is similar
to opening other files:

\beforeverb
\begin{verbatim}
>>> import anydbm
>>> db = anydbm.open('captions.db', 'c')
\end{verbatim}
\afterverb
%
The mode {\tt 'c'} means that the database should be created if
it doesn't already exist.  The result is a database object
that can be used (for most operations) like a dictionary.
If you create a new item, {\tt anydbm} updates the database file.

\beforeverb
\begin{verbatim}
>>> db['cleese.png'] = 'Photo of John Cleese.'
\end{verbatim}
\afterverb
%
When you access one of the items, {\tt anydbm} reads the file:

\beforeverb
\begin{verbatim}
>>> print db['cleese.png']
Photo of John Cleese.
\end{verbatim}
\afterverb
%
If you make another assignment to an existing key, {\tt anydbm} replaces
the old value:

\beforeverb
\begin{verbatim}
>>> db['cleese.png'] = 'Photo of John Cleese doing a silly walk.'
>>> print db['cleese.png']
Photo of John Cleese doing a silly walk.
\end{verbatim}
\afterverb
%
Many dictionary methods, like {\tt keys} and {\tt items}, also
work with database objects.  So does iteration with a {\tt for}
statement.

\beforeverb
\begin{verbatim}
for key in db:
     print key
\end{verbatim}
\afterverb
%
As with other files, you should close the database when you are
done:

\beforeverb
\begin{verbatim}
>>> db.close()
\end{verbatim}
\afterverb
%


\section{Pickling}

A limitation of {\tt anydbm} is that the keys and values have
to be strings.  If you try to use any other type, you get an
error.

But the {\tt pickle} module can help.  It translates
almost any type of object into a string, suitable for storage in a
database, and then translates strings back into objects.

{\tt pickle.dumps} takes an object as a parameter and returns
a string representation ({\tt dumps} is short for ``dump string''):

\beforeverb
\begin{verbatim}
>>> import pickle
>>> t = [1, 2, 3]
>>> pickle.dumps(t)
'(lp0\nI1\naI2\naI3\na.'
\end{verbatim}
\afterverb
%
The format isn't obvious to human readers; it is meant to be
easy for {\tt pickle} to interpret.  {\tt pickle.loads}
(``load string'') reconstitutes the object:

\beforeverb
\begin{verbatim}
>>> t1 = [1, 2, 3]
>>> s = pickle.dumps(t1)
>>> t2 = pickle.loads(s)
>>> print t2
[1, 2, 3]
\end{verbatim}
\afterverb
%
Although the new object has the same value as the old, it is
not (in general) the same object:

\beforeverb
\begin{verbatim}
>>> t == t2
True
>>> t is t2
False
\end{verbatim}
\afterverb
%
In other words, pickling and then unpickling has the same effect
as copying the object.

You can use {\tt pickle} to store non-strings in a database.
In fact, this combination is so common that it has been
encapsulated in a module called {\tt shelve}.  

\begin{ex}
If you did Exercise~\ref{anagrams}, modify your solution so that
it creates a database that maps from each word in the list to
a list of words that use the same set of letters.

Write a different program that opens the database and prints
the contents in a human-readable format.
\end{ex}


%\section{Debugging}

%Print and then write.



\section{Glossary}

\begin{description}

\item[persistent:] Pertaining to a program that runs indefinitely
and keeps at least some of its data in permanent storage.
\index{persistent}

\item[format operator:] An operator, {\tt \%}, that takes a format
string and a tuple and generates a string that includes
the elements of the tuple formatted as specified by the format string.
\index{format operator}
\index{operator!format}

\item[format string:] A string, used with the format operator, that
contains format sequences.  
\index{format string}

\item[format sequence:] A sequence of characters in a format string,
like {\tt \%d} that specifies how a value should be formatted.
\index{format sequence}

\item[text file:] A sequence of characters stored in non-volatile
storage like a hard drive.
\index{text file}

\item[directory:] A named collection of files, also called a folder.
\index{directory}

\item[path:] A string that identifies a file.
\index{path}

\item[relative path:] A path that starts from the current directory.
\index{relative path}

\item[absolute path:] A path that starts from the topmost directory
in the file system.
\index{absolute path}

\item[catch:] To prevent an exception from terminating
a program using the {\tt try}
and {\tt except} statements.
\index{catch}

\item[database:] A file whose contents are organized like a dictionary
with keys that correspond to values.
\index{database}

\end{description}




\chapter{Classes and objects}
\index{class}
\index{object}


\section{User-defined types}
\label{point}
\index{user-defined type}
\index{type!user-defined}

We have used many of Python's built-in types; now we are going
to define a new type.  As an example, we will create a type
called {\tt Point} that represents a point in two-dimensional
space.

In mathematical notation, points are often written in
parentheses with a comma separating the coordinates. For example,
$(0, 0)$ represents the origin, and $(x, y)$ represents the
point $x$ units to the right and $y$ units up from the origin.

There are several ways we might represent points in Python:

\begin{itemize}

\item We could store the coordinates separately in two
variables, {\tt x} and {\tt y}.

\item We could store the coordinates as elements in a list
or tuple.

\item We could create a new type to represent points as
objects.

\end{itemize}

Creating a new type
is (a little) more complicated than the other options, but
it has advantages that will be apparent soon.

A user-defined type is also called a {\bf class}.
A class definition looks like this:

\beforeverb
\begin{verbatim}
class Point:
    """represents a point in 2-D space"""
\end{verbatim}
\afterverb
%
This header indicates that the new class is called {\tt Point}.
The body is a docstring that explains what the class is for.
You can define variables and functions inside a class definition,
but we will get back to that later.

\index{docstring}

Defining a class named {\tt Point} creates a class object,
also named {\tt Point}.

\beforeverb
\begin{verbatim}
>>> print Point
__main__.Point
>>> type(Point)
<type 'classobj'>
\end{verbatim}
\afterverb
%
Because {\tt Point} is defined at the top level, its ``full
name'' is {\tt \_\_main\_\_.Point}.

\index{object!class}
\index{class object}

The class object is like a factory for creating objects.  To create a
Point, you call {\tt Point} as if it were a function.

\beforeverb
\begin{verbatim}
>>> blank = Point()
>>> print blank
<__main__.Point instance at 0xb7e9d3ac>
\end{verbatim}
\afterverb
%
The return value is a reference to a Point object, which we
assign to {\tt blank}.  
Creating a new object is called
{\bf instantiation}, and the object is and {\bf instance} of
the class.

\index{instance}
\index{instantiation}


\section{Attributes}
\index{attribute}

You can assign values to an instance using dot notation:

\beforeverb
\begin{verbatim}
>>> blank.x = 3.0
>>> blank.y = 4.0
\end{verbatim}
\afterverb
%
This syntax is similar to the syntax for selecting a variable from a
module, such as {\tt math.pi} or {\tt string.uppercase}.  In this case,
though, we are assigning values to named elements of an object.
These elements are called {\bf attributes}.

The following diagram shows the result of these assignments.
A state diagram that shows an object and its attributes is
called an {\bf object diagram}:

\index{state diagram}
\index{diagram!state}
\index{object diagram}
\index{diagram!object}

\beforefig
\centerline{\includegraphics{figs/point.eps}}
\afterfig

The variable {\tt blank} refers to a Point object, which
contains two attributes.  Each attribute refers to a
floating-point number.

We can read the value of an attribute using the same syntax:

\beforeverb
\begin{verbatim}
>>> print blank.y
4.0
>>> x = blank.x
>>> print x
3.0
\end{verbatim}
\afterverb
%
The expression {\tt blank.x} means, ``Go to the object {\tt blank}
refers to and get the value of {\tt x}.'' In this case, we assign that
value to a variable named {\tt x}.  There is no conflict between
the variable {\tt x} and the attribute {\tt x}.

You can use dot notation as part of any expression.  For example:

\beforeverb
\begin{verbatim}
>>> print '(%g, %g)' % (blank.x, blank.y)
(3.0, 4.0)
>>> distance = math.sqrt(blank.x**2 + blank.y**2)
>>> print distance
5.0
\end{verbatim}
\afterverb
%
You can pass an instance as an argument in the usual way.
For example:

\beforeverb
\begin{verbatim}
def print_point(p):
    print '(%g, %g)' % (p.x, p.y)
\end{verbatim}
\afterverb
%
{\tt print\_point} takes a point as an argument and displays it in
mathematical notation.  To invoke it, you can pass {\tt blank} as
an argument:

\beforeverb
\begin{verbatim}
>>> print_point(blank)
(3.0, 4.0)
\end{verbatim}
\afterverb
%
Inside the function, {\tt p} is an alias for {\tt blank}, so if
the function modifies {\tt p}, {\tt blank} changes.

\begin{ex}
Write a function called {\tt distance} that it takes two Points
as arguments and returns the distance between them.
\end{ex}



\section{Rectangles}
\index{rectangle}

Sometimes it is obvious what the attributes of an object should be,
but other times you have to make decisions.  For example, imagine you
are designing a class to represent rectangles.  What attributes would
you use to specify the location and size of a rectangle?  You can
ignore angle; to keep things simple, assume that the rectangle is
either vertical or horizontal.

There are at least two possibilities: 

\begin{itemize}

\item You could specify one corner of the rectangle
(or the center), the width, and the height.

\item You could specify two opposing corners.

\end{itemize}

At this point it is hard to say whether either is better than
the other, so we'll implement the first one, just as an example.

Here is the class definition:

\beforeverb
\begin{verbatim}
class Rectangle:
    """represent a rectangle. 
       attributes: width, height, corner.
    """
\end{verbatim}
\afterverb
%
The docstring lists the attribute names.  {\tt width} and
{\tt height} are numbers; {\tt corner} is a Point object that
specifies the lower-left corner.

To represent a rectangle, you have to instantiate a Rectangle
object and assign values to the attributes:

\beforeverb
\begin{verbatim}
box = Rectangle()
box.width = 100.0
box.height = 200.0
box.corner = Point()
box.corner.x = 0.0
box.corner.y = 0.0
\end{verbatim}
\afterverb
%
The expression {\tt box.corner.x} means,
``Go to the object {\tt box} refers to and select the attribute named
{\tt corner}; then go to that object and select the attribute named
{\tt x}.''

The figure shows the state of this object:

\index{state diagram}
\index{diagram!state}
\index{object diagram}
\index{diagram!object}

\beforefig
\centerline{\includegraphics{figs/rectangle.eps}}
\afterfig


\section{Instances as return values}
\index{instance}
\index{return value}

Functions can return instances.  For example, {\tt find\_center}
takes a {\tt Rectangle} as an argument and returns a {\tt Point}
that contains the coordinates of the center of the {\tt Rectangle}:

\beforeverb
\begin{verbatim}
def find_center(box):
    p = Point()
    p.x = box.corner.x + box.width/2.0
    p.y = box.corner.y + box.height/2.0
    return p
\end{verbatim}
\afterverb
%
Here is an example that passes {\tt box} as an argument and assign
the resulting Point to {\tt center}:

\beforeverb
\begin{verbatim}
>>> center = find_center(box)
>>> print_point(center)
(50.0, 100.0)
\end{verbatim}
\afterverb
%

\section{Objects are mutable}
\index{object!mutable}
\index{mutable!object}

We can change the state of an object by making an assignment to one of
its attributes.  For example, to change the size of a rectangle
without changing its position, you can modify the values of {\tt
width} and {\tt height}:

\beforeverb
\begin{verbatim}
box.width = box.width + 50
box.height = box.width + 100
\end{verbatim}
\afterverb
%
You can also write functions that modify objects.  For example,
{\tt grow\_rectangle} takes a Rectangle object and two numbers,
{\tt dwidth} and {\tt dheight}, and adds the numbers to the
width and height of the rectangle:

\beforeverb
\begin{verbatim}
def grow_rectangle(rect, dwidth, dheight) :
    rect.width += dwidth
    rect.height += dheight
\end{verbatim}
\afterverb
%
Here is an example that demonstrates the effect:

\beforeverb
\begin{verbatim}
>>> print box.width
100.0
>>> print box.height
200.0
>>> grow_rectangle(box, 50, 100)
>>> print box.width
150.0
>>> print box.height
300.0
\end{verbatim}
\afterverb
%
Inside the function, {\tt rect} is an
alias for {\tt box}, so if the function modifies {\tt rect}, 
{\tt box} changes.

\begin{ex}
Write a function named {\tt move\_rectangle} that takes
a Rectangle and two numbers named {\tt dx} and {\tt dy}.  It
should change the location of the rectangle by adding {\tt dx}
to the {\tt x} coordinate of {\tt corner} and adding {\tt dy}
to the {\tt y} coordinate of {\tt corner}.
\end{ex}


\section{Copying}
\label{embedded}
\index{aliasing}
\index{copying}
\index{copy module}
\index{module!copy}

Aliasing can make a program difficult to read because changes
made in one place might have unexpected effects in another place.
It is hard to keep track of all the variables that might refer
to a given object.

Copying an object is often an alternative to aliasing.
The {\tt copy} module contains a function called {\tt copy} that
can duplicate any object:

\beforeverb
\begin{verbatim}
>>> p1 = Point()
>>> p1.x = 3.0
>>> p1.y = 4.0

>>> import copy
>>> p2 = copy.copy(p1)
\end{verbatim}
\afterverb
%
{\tt p1} and {\tt p2} contain the same data, but they are
not the same Point.

\beforeverb
\begin{verbatim}
>>> print_point(p1)
(3.0, 4.0)
>>> print_point(p2)
(3.0, 4.0)
>>> p1 is p2
False
>>> p1 == p2
False
\end{verbatim}
\afterverb
%
The {\tt is} operator indicates that {\tt p1} and {\tt p2} are not the
same object, which is what we expected.  But you might have expected
{\tt ==} to yield {\tt True} because these points contain the same
data.  In that case, you will be disappointed to learn that for
instances, the default behavior of the {\tt ==} operator is the same
as the {\tt is} operator; it checks object identity, not object
equivalence.

This behavior can be changed, so for many objects defined in Python
modules, the {\tt ==} operator checks equivalence (in whatever
sense is appropriate).  But the default is to check identity.

If you use {\tt copy.copy} to duplicate a Rectangle, you will find
that it copies the Rectangle object but not the embedded Point.

\beforeverb
\begin{verbatim}
>>> box2 = copy.copy(box)
>>> box2 is box
False
>>> box2.corner is box.corner
True
\end{verbatim}
\afterverb
%
Here is what the object diagram looks like:

\index{state diagram}
\index{diagram!state}
\index{object diagram}
\index{diagram!object}

\vspace{0.1in}
\beforefig
\centerline{\includegraphics{figs/rectangle2.eps}}
\afterfig
\vspace{0.1in}

This operation is called a {\bf shallow copy} because it copies the
object and any references it contains, but not the embedded objects.

For most applications, this is not what you want.  In this example,
invoking {\tt grow\_rectangle} on one of the Rectangles would not
affect the other, but invoking {\tt move\_rectangle} on either would
affect both!  This behavior is confusing and error-prone.

Fortunately, the {\tt copy} module contains a method named {\tt
deepcopy} that copies not only the object but also 
the objects it refers to, and they objects {\em they} refer to,
and so on.
You will not be surprised to learn that this operation is
called a {\bf deep copy}.

\beforeverb
\begin{verbatim}
>>> box3 = copy.deepcopy(box)
>>> box3 is box
False
>>> box3.corner is box.corner
False
\end{verbatim}
\afterverb
%
{\tt box3} and {\tt box} are completely separate objects.


\begin{ex}
Write a version {\tt move\_rectangle} that it creates and
returns a new Rectangle instead of modifying the old one.
\end{ex}


\section{Debugging}

When you start working with objects, you are likely to encounter
some new exceptions.  If you try to access an attribute
that doesn't exist, you get an {\tt AttributeError}:

\index{exception!Attribute}
\index{Attribute}

\beforeverb
\begin{verbatim}
>>> p = Point(3, 4)
>>> print p.z
AttributeError: Point instance has no attribute 'z'
\end{verbatim}
\afterverb
%
If you are not sure what type an object is, you can ask:

\beforeverb
\begin{verbatim}
>>> type(p)
<type 'instance'>
\end{verbatim}
\afterverb
%
This result tells us that {\tt p} is an object, but not what
kind.  But all objects have a special attribute named
{\tt \_\_class\_\_} that refers to the object's class.

\beforeverb
\begin{verbatim}
>>> print p.__class__
__main__.Point
\end{verbatim}
\afterverb
%
If you are not sure whether an object has a particular attribute,
you can use the built-in function {\tt hasattr}:

\beforeverb
\begin{verbatim}
>>> hasattr(p, 'x')
True
>>> hasattr(p, 'z')
False
\end{verbatim}
\afterverb
%
The first argument can be any object; the second argument is a {\em
string} that contains the name of the attribute.

Another way to access the attributes of an object is through the
special attribute {\tt \_\_dict\_\_}, which is a dictionary that maps
from attribute names (as strings) and values:

\beforeverb
\begin{verbatim}
>>> print p.__dict__
{'y': 4, 'x': 3}
\end{verbatim}
\afterverb
%
For purposes of debugging, you might find it useful to keep this
function handy:

\beforeverb
\begin{verbatim}
def print_attributes(obj):
    for attr in obj.__dict__:
        print attr, getattr(obj, attr)
\end{verbatim}
\afterverb
%
{\tt print\_attributes} traverses the items in the object's dictionary
print each attrbute name and its corresponding value.

The built-in function {\tt getattr} takes an object and an attribute
name (as a string) and returns the attribute's value.



\section{Glossary}

\begin{description}

\item[class:] A user-defined type.  A class definition creates a new
class object.
\index{class}

\item[class object:] An object that contains information about a
user-defined time.  The class object can be used to create instances
of the type.
\index{class object}

\item[instance:] An object that belongs to a class.
\index{instance}

\item[attribute:] One of the named values associated with an object.
\index{attribute}

\item[shallow copy:] To copy the contents of an object, including
any references to embedded objects;
implemented by the {\tt copy} function in the {\tt copy} module.
\index{shallow copy}

\item[deep copy:] To copy the contents of an object as well as any
embedded objects, and any objects embedded in them, and so on;
implemented by the {\tt deepcopy} function in the {\tt copy} module.
\index{deep copy}

\item[object diagram:] A diagram that shows objects, their
attributes, and the values of the attributes.
\index{object diagram}
\index{diagram!object}

\end{description}

\section{Exercises}



\chapter{Classes and functions}
\label{time}
\index{function}
\index{method}


\section{Time}

As another example of a user-defined type, we'll define a class called
{\tt Time} that records the time of day.  The class definition looks
like this:

\beforeverb
\begin{verbatim}
class Time:
    """represents the time of day
       attributes: hour, minute, second"""
\end{verbatim}
\afterverb
%
We can create a new {\tt Time} object and assign
attributes for hours, minutes, and seconds:

\beforeverb
\begin{verbatim}
time = Time()
time.hour = 11
time.minute = 59
time.second = 30
\end{verbatim}
\afterverb
%
The state diagram for the {\tt Time} object looks like this:

\index{state diagram}
\index{diagram!state}
\index{object diagram}
\index{diagram!object}

\beforefig
\centerline{\includegraphics{figs/time.eps}}
\afterfig

\begin{ex}
\label{printtime}
Write a function {\tt print\_time} that takes a 
Time object and prints it in the form {\tt hour:minute:second}.
\end{ex}

\begin{ex}
\label{after}
Write a boolean function {\tt after} that
takes two Time objects, {\tt t1} and {\tt t2}, and
returns {\tt True} if {\tt t1} follows {\tt t2} chronologically and
{\tt False} otherwise.
\end{ex}


\section{Pure functions}
\index{pure function}
\index{function type!pure}

In the next few sections, we'll write two versions of a function
called {\tt add\_time}, which calculates the sum of two Time objects.
They demonstrate two kinds of functions: pure functions and
modifiers.  They also demonstrate a development plan I'll call
{\bf prototype and patch}, which is a way of tackling a complex
problem by starting with a simple prototype and incrementally
dealing with the complications.

Here is a simple prototype of {\tt add\_time}:

\beforeverb
\begin{verbatim}
def add_time(t1, t2):
    sum = Time()
    sum.hour = t1.hour + t2.hour
    sum.minute = t1.minute + t2.minute
    sum.second = t1.second + t2.second
    return sum
\end{verbatim}
\afterverb
%
The function creates a new {\tt Time} object, initializes its
attributes, and returns a reference to the new object.  This is called
a {\bf pure function} because it does not modify any of the objects
passed to it as arguments and it has no side effects, such as
displaying a value or getting user input.

To test this function, I'll create two Time objects: {\tt start}
contains the start time of a movie, like {\em Monty Python and the
Holy Grail}, and {\tt duration} contains the run time of the movie,
which is one hour 35 minutes.

{\tt add\_time} figures out when the movie will be done.

\beforeverb
\begin{verbatim}
>>> start = Time()
>>> start.hour = 9
>>> start.minute = 45
>>> start.second =  0

>>> duration = Time()
>>> duration.hour = 1
>>> duration.minute = 35
>>> duration.second = 0

>>> done = add_time(start, duration)
>>> print_time(done)
10:80:00
\end{verbatim}
\afterverb
%
The result, {\tt 10:80:00} might not be what you were hoping
for.  The problem is that this function does not deal with cases where the
number of seconds or minutes adds up to more than sixty.  When that
happens, we have to ``carry'' the extra seconds into the minute column
or the extra minutes into the hour column.

Here's an improved version:

\beforeverb
\begin{verbatim}
def add_time(t1, t2):
    sum = Time()
    sum.hour = t1.hour + t2.hour
    sum.minute = t1.minute + t2.minute
    sum.second = t1.second + t2.second

    if sum.second >= 60:
        sum.second -= 60
        sum.minute += 1

    if sum.minute >= 60:
        sum.minute -= 60
        sum.hour += 1

    return sum
\end{verbatim}
\afterverb
%
Although this function is correct, it is starting to get big.
We will see a shorter alternative later.


\section{Modifiers}
\label{increment}
\index{modifier}
\index{function type!modifier}

Sometimes it is useful for a function to modify the objects it gets as
parameters.  In that case, the changes are visible to the caller.
Functions that work this way are called {\bf modifiers}.

{\tt increment}, which adds a given number of seconds to a {\tt Time}
object, can be written naturally as a
modifier.  Here is a rough draft:

\beforeverb
\begin{verbatim}
def increment(time, seconds):
    time.second += seconds

    if time.second >= 60:
        time.second -= 60
        time.minute += 1

    if time.minute >= 60:
        time.minute -= 60
        time.hour += 1
\end{verbatim}
\afterverb
%
The first line performs the basic operation; the remainder deals
with the special cases we saw before.

Is this function correct?  What happens if the parameter {\tt seconds}
is much greater than sixty?  In that case, it is not enough to carry
once; we have to keep doing it until {\tt time.second} is less than sixty.
One solution is to replace the {\tt if} statements with {\tt while}
statements.  That would make the function correct, but not
very efficient.

\begin{ex}
Write a correct version of {\tt increment} that
doesn't contain any loops.
\end{ex}


Anything that can be done with modifiers can also be done with pure
functions.  In fact, some programming languages only allow pure
functions.  There is some evidence that programs that use pure
functions are faster to develop and less error-prone than programs
that use modifiers.  But modifiers are convenient at times,
and functional programs tend to be less efficient.

In general, I recommend that you write pure functions whenever it is
reasonable and resort to modifiers only if there is a compelling
advantage.  This approach might be called a {\bf functional
programming style}.

\index{functional programming style}


\begin{ex}
Write a ``pure'' version of {\tt increment} that creates and returns
a new Time object rather than modifying the parameter.
\end{ex}


\section{Prototyping versus planning}
\label{prototype}
\index{prototype development}

In this chapter, I demonstrated development plan called ``prototype
and patch.''  For each function, I wrote a rough draft that performed
the basic calculation and then tested, correcting flaws along the way.

Although this approach can be effective, especially if you don't yet
have a deep understanding of the problem.  But incremental patching
can generate code that is unnecessarily complicated---since it deals
with many special cases---and unreliable---since it is hard to know if
you have found all the errors.

An alternative is {\bf planned development}, in which high-level insight
into the problem can make the programming much easier.  In this case,
the insight is that a Time object is really a three-digit number
in base 60!  The {\tt second} attribute is the ``ones column,'' the
{\tt minute} attribute is the ``sixties column,'' and the {\tt hour}
attribute is the ``thirty-six hundreds column.''

When we wrote {\tt add\_time} and {\tt increment}, we were effectively
doing addition in base 60, which is why we had to carry from one
column to the next.

This observation suggests another approach to the whole problem---we
can convert Time objects to integers and take advantage of the fact
that the computer knows how to do integer arithmetic.  

Here is
the function that converts Times to integers:

\beforeverb
\begin{verbatim}
def time_to_int(time):
    minutes = time.hour * 60 + time.minute
    seconds = minutes * 60 + time.second
    return seconds
\end{verbatim}
\afterverb
%
And here is the function that converts integers to Times
(recall that {\tt divmod} divides the first argument by the second
and returns the quotient and remainder as a tuple).

\index{divmod}

\beforeverb
\begin{verbatim}
def int_to_time(seconds):
    time = Time()
    minutes, time.second = divmod(seconds, 60)
    time.hour, time.minute = divmod(minutes, 60)
    return time
\end{verbatim}
\afterverb
%
You might have to think a bit, and run some tests, to convince
yourself that these functions are correct.  But once they are
debugged, you can use them to 
rewrite {\tt add\_time}:

\beforeverb
\begin{verbatim}
def add_time(t1, t2):
    seconds = time_to_int(t1) + time_to_int(t2)
    return int_to_time(seconds)
\end{verbatim}
\afterverb
%
This version is shorter than the original, and easier to verify.

\begin{ex}
Rewrite {\tt increment} using {\tt time\_to\_int} and {\tt int\_to\_time}.
\end{ex}

In some ways, converting from base 60 to base 10 and back is harder
than just dealing with times.  Base conversion is more abstract; our
intuition for dealing with times is better.

But if we have the insight to treat times as base 60 numbers and make
the investment of writing the conversion functions ({\tt
time\_to\_int} and {\tt int\_to\_time}), we get a program that is
shorter, easier to read and debug, and more reliable.

It is also easier to add features later.  For example, imagine
subtracting two Times to find the duration between them.  The
na\"{\i}ve approach would be to implement subtraction with borrowing.
Using the conversion functions would be easier and more likely to be
correct.

\index{generalization}

Ironically, sometimes making a problem harder (or more general) makes it
easier (because there are fewer special cases and fewer opportunities
for error).


% \section{Debugging}

% A Time object is well-formed if the values of {\tt minutes} and {\tt
% seconds} are between 0 and 60 (including 0 but not 60) and if 
% {\tt hours} is positive.  {\tt hours} and {\tt minutes} should be
% integral values, but we might allow {\tt seconds} to have a
% fraction part.

% These kind of requirements are called {\bf invariants}...

% Self-testing, checking whether {\tt time\_to\_int} and {\tt
% int\_to\_time} are consistent.



\section{Glossary}

\begin{description}

\item[prototype and patch:] A development plan that involves
writing a rough draft of a program, testing, and correcting errors as
they are found.
\index{prototype and patch}

\item[planned development:] A development plan that involves
high-level insight into the problem and more planning than incremental
development or prototype development.
\index{planned development}

\item[pure function:] A function that does not modify any of the objects it
receives as arguments.  Most pure functions are fruitful.
\index{pure function}

\item[modifier:] A function that changes one or more of the objects it
receives as arguments.  Most modifiers are fruitless.
\index{modifier}

\item[functional programming style:] A style of program design in which the
majority of functions are pure.
\index{functional programming style}

\end{description}


\section{Exercises}

\begin{ex}
Write afunction called {\tt mul\_time} that takes a Time object
and a number and returns a new Time object that contains
the product of the original Time and the number.

Then use {\tt mul\_time} to write a function that takes a Time
object that represents the finishing time in a race, and a number
that represents the distance, and returns a Time object that represents
the average pace (time per mile).

\end{ex}


\chapter{Classes and methods}


\section{Object-oriented features}
\index{object-oriented programming language}
\index{object-oriented programming}

Python is an {\bf object-oriented programming language}, which means
that it provides features that support object-oriented
programming.

It is not easy to define object-oriented programming, but we have
already seen some of its characteristics:

\begin{itemize}

\item Programs are made up of object definitions and function
definitions, and most of the computation is expressed in terms
of operations on objects.

\item Each object definition corresponds to some object or concept
in the real world, and the functions that operate on that object
correspond to the ways real-world objects interact.

\end{itemize}

For example, the {\tt Time} class defined in Chapter~\ref{time}
corresponds to the way people record the time of day, and the
functions we defined correspond to the kinds of things people do with
times.  Similarly, the {\tt Point} and {\tt Rectangle} classes
correspond to the mathematical concepts of a point and a rectangle.

So far, we have not taken advantage of the features Python provides to
support object-oriented programming.  Strictly speaking, these
features are not necessary.  For the most part, they provide an
alternative syntax for things we have already done, but in many cases,
the alternative is more concise and more accurately conveys the
structure of the program.

For example, in the {\tt Time} program, there is no obvious
connection between the class definition and the function definitions
that follow.  With some examination, it is apparent that every function
takes at least one {\tt Time} object as an argument.

This observation is the motivation for {\bf methods}; a method is
a function that is associated with a particular class.  For example,
we have seen methods for strings, lists, dictionaries and tuples.
In this chapter, we will define methods for user-defined types.

\index{method}
\index{function}

Methods are semantically the same as functions, but there are
two syntactic differences:

\begin{itemize}

\item Methods are defined inside a class definition in order
to make the relationship between the class and the method explicit.

\item The syntax for invoking a method is different from the
syntax for calling a function.

\end{itemize}

In the next few sections, we will take the functions from the previous
two chapters and transform them into methods.  This transformation is
purely mechanical; you can do it simply by following a sequence of
steps.  If you are comfortable converting from one form to another,
you will be able to choose the best form for whatever you are doing.


\section{{\tt print\_time}}
\label{print_time}
\index{printing!object}

In Chapter~\ref{time}, we defined a class named
{\tt Time} and in Exercise~\ref{printtime}, you 
wrote a function named {\tt print\_time}:

\beforeverb
\begin{verbatim}
class Time:
    """represents the time of day
       attributes: hour, minute, second"""

def print_time(time):
    print '%.2d:%.2d:%.2d' % (time.hour, time.minute, time.second)
\end{verbatim}
\afterverb
%
To call this function, you have to pass a {\tt Time} object as an
argument:

\beforeverb
\begin{verbatim}
>>> start = Time()
>>> start.hour = 9
>>> start.minute = 45
>>> start.second = 00
>>> print_time(start)
09:45:00
\end{verbatim}
\afterverb
%
To make {\tt print\_time} a method, all we have to do is
move the function definition inside the class definition.  Notice
the change in indentation.

\beforeverb
\begin{verbatim}
class Time:
    def print_time(time):
        print '%.2d:%.2d:%.2d' % (time.hour, time.minute, time.second)
\end{verbatim}
\afterverb
%
Now there are two ways to call {\tt print\_time}.  The first
(and less common) way is to use function syntax:

\beforeverb
\begin{verbatim}
>>> Time.print_time(start)
09:45:00
\end{verbatim}
\afterverb
%
In this use of dot notation, {\tt Time} is the name of the class,
and {\tt print\_time} is the name of the method.  {\tt start} is
passed as a parameter.

The second (and more concise) way is to use method syntax:

\beforeverb
\begin{verbatim}
>>> start.print_time()
09:45:00
\end{verbatim}
\afterverb
%
In this use of dot notation, {\tt print\_time} is the name of the
method (again), and {\tt start} is the object the method is
invoked on, which is called the {\bf subject}.  Just as the
subject of a sentence is what the sentence is about, the subject
of a method invocation is what the method is about.

Inside the method, the subject is assigned to the first
parameter, so in this case {\tt start} is assigned
to {\tt time}.

By convention, the first parameter of a method is
called {\tt self}, so it would be more common to write
{\tt print\_time} like this:

\beforeverb
\begin{verbatim}
class Time:
    def print_time(self):
        print '%.2d:%.2d:%.2d' % (self.hour, self.minute, self.second)
\end{verbatim}
\afterverb
%
The reason for this convention is convoluted, but it is based on a
useful metaphor:

The syntax for a function call, {\tt print\_time(start)}, suggests that
the function is the active agent.  It says something like, ``Hey {\tt
print\_time}!  Here's an object for you to print.''

In object-oriented programming, the objects are the active
agents.  A method invocation like {\tt start.print\_time()}
says ``Hey {\tt start}!  Please print yourself.''  

This change in perspective might be more polite, but
it is not obvious that it is useful.  In the examples we
have seen so far, it may not be.  But sometimes shifting
responsibility from the functions onto the objects
makes it possible to write more versatile functions,
and makes it easier to maintain and reuse code.

\begin{ex}
\label{convert}
Rewrite {\tt time\_to\_int}
(from Section~\ref{prototype}) as a method.  It is probably not
appropriate to rewrite {\tt int\_to\_time} as a method; it's not
clear what object you would invoke it on!
\end{ex}


\section{Another example}

Here's a version of {\tt increment} (from Section~\ref{increment})
rewritten as a method:

\beforeverb
\begin{verbatim}
# inside class Time:

    def increment(self, seconds):
        seconds += self.time_to_int()
        return int_to_time(seconds)
\end{verbatim}
\afterverb
%
This version assumes that {\tt time\_to\_int} is written
as a method, as in Exercise~\ref{convert}.  Also, note that
it is a pure function, not a modifier.

Here's how you would invoke {\tt increment}:

\beforeverb
\begin{verbatim}
>>> start.print_time()
09:45:00
>>> end = start.increment(1337)
>>> end.print_time()
10:07:17
\end{verbatim}
\afterverb
%
The subject, {\tt start}, gets assigned to the first parameter,
{\tt self}.  The argument, {\tt 1337}, gets assigned to the
second parameter, {\tt seconds}.

This mechanism can be confusing, especially if you make an error.
For example, if you invoke {\tt increment} with two arguments, you
get:

\index{exception!TypeError}
\index{TypeError}

\beforeverb
\begin{verbatim}
>>> end = start.increment(1337, 460)
TypeError: increment() takes exactly 2 arguments (3 given)
\end{verbatim}
\afterverb
%
The error message is initially confusing, because there are
only two arguments in parentheses.  But the subject is also
considered an argument, so all together that's three.

\begin{ex}
Convert {\tt time\_to\_int} 
(from Section~\ref{prototype}) to a method in the
{\tt Time} class.
\end{ex}


\section{A more complicated example}

{\tt after} (from Exercise~\ref{after}) is slightly more complicated
because it takes two Time objects as parameters.  In this case it is
conventional to name the first parameter {\tt self} and the second
parameter {\tt other}:

\beforeverb
\begin{verbatim}
# inside class Time:

    def after(self, other):
        return self.time_to_int() > other.time_to_int()
\end{verbatim}
\afterverb
%
To use this method, you have to invoke it on one object and pass
the other as an argument:

\beforeverb
\begin{verbatim}
>>> end.after(start)
True
\end{verbatim}
\afterverb
%
One nice thing about this syntax is that it has the same word
order as English, subject-verb-object.



\section{The {\tt init} method}
\index{{\tt init} method}
\index{method!{\tt init}}

The {\tt init} method (short for ``initialization'') is
a special method that gets invoked when an object is instantiated.  
Its full name is {\tt \_\_init\_\_} (two underscore characters,
followed by {\tt init}, and then two more underscores).  An
init method for the {\tt Time} class might look like this:

\beforeverb
\begin{verbatim}
# inside class Time:

    def __init__(self, hour=0, minute=0, second=0):
        self.hour = hour
        self.minute = minute
        self.second = second
\end{verbatim}
\afterverb
%
It is common for the parameters of {\tt \_\_init\_\_}
to have the same names as the attributes.  The statement

\beforeverb
\begin{verbatim}
        self.hour = hour
\end{verbatim}
\afterverb
%
stores the value of the parameter {\tt hour} as an attribute
in the new Time object {\tt self}.

The parameters are optional, so if you call {\tt Time} with
no arguments, you get the default values.

\beforeverb
\begin{verbatim}
>>> time = Time()
>>> time.print_time()
00:00:00
\end{verbatim}
\afterverb
%
If you provide one argument, it overrides {\tt hour}:

\beforeverb
\begin{verbatim}
>>> time = Time (9)
>>> time.print_time()
09:00:00
\end{verbatim}
\afterverb
%
If you provide two arguments, they override {\tt hour} and
{\tt minute}.

\beforeverb
\begin{verbatim}
>>> time = Time(9, 45)
>>> time.print_time()
09:45:00
\end{verbatim}
\afterverb
%
And if you provide three arguments, they override all three
default values.


\begin{ex}
Write an init method for the {\tt Point} class that takes
{\tt x} and {\tt y} as optional parameters and assigns
them to the corresponding attributes.
\end{ex}


\section{The {\tt str} method}
\index{Point class}
\index{class!Point}

{\tt \_\_str\_\_} is a special method name, like {\tt \_\_init\_\_},
that is supposed to return a string representation of an object.

For example, here is a {\tt str} method for Time objects:

\beforeverb
\begin{verbatim}
# inside class Time:

    def __str__(self):
        return '%.2d:%.2d:%.2d' % (self.hour, self.minute, self.second)
\end{verbatim}
\afterverb
%
When you {\tt print} an object, Python invokes the {\tt str} method:

\beforeverb
\begin{verbatim}
>>> time = Time(9, 45)
>>> print time
09:45:00
\end{verbatim}
\afterverb
%
When I write a new class, I almost always start by writing {\tt
\_\_init\_\_}, which makes it easier to instantiate objects, and {\tt
\_\_str\_\_}, which is almost always useful for debugging.


\begin{ex}
Write a {\tt str} method for the {\tt Point} class.  Create
a Point object and print it.
\end{ex}


\section{Operator overloading}
\label{operator overloading}
\index{operator overloading}

By defining other special methods, you can specify the behavior
of operators on user-defined types.  For example, if you define
an {\tt add} method for the {\tt Time} class, you can use the
{\tt +} operator on Time objects.

Here is what the definition might look like:

\beforeverb
\begin{verbatim}
# inside class Time:

    def __add__(self, other):
        seconds = self.time_to_int() + other.time_to_int()
        return int_to_time(seconds)
\end{verbatim}
\afterverb
%
And here is how you could use it:

\beforeverb
\begin{verbatim}
>>> start = Time(9, 45)
>>> duration = Time(1, 35)
>>> print start + duration
11:20:00
\end{verbatim}
\afterverb
%
When you apply the {\tt +} operator to Time objects, Python invokes
{\tt \_\_add\_\_}.  When you print the result, Python invokes {\tt
\_\_str\_\_}.  So there is quite a lot happening behind the scenes!

Changing the behavior of an operator so that it works with
user-defined types is called {\bf operator overloading}.  For every
operator in Python there is a corresponding special method, like {\tt
\_\_add\_\_}.

\begin{ex}
Write an {\tt add} method for the Point class.  
\end{ex}


\section{Type-based dispatch}

In the previous section we added two Time objects, but you
also might want to add an integer to a Time object.  The
following is an alternative version of {\tt \_\_add\_\_}
that checks the type of {\tt other} and invokes either
{\tt add\_time} or {\tt increment}:

\beforeverb
\begin{verbatim}
# inside class Time:

    def __add__(self, other):
        if isinstance(other, Time):
            return self.add_time(other)
        else:
            return self.increment(other)

    def add_time(self, other):
        seconds = self.time_to_int() + other.time_to_int()
        return int_to_time(seconds)

    def increment(self, seconds):
        seconds += self.time_to_int()
        return int_to_time(seconds)
\end{verbatim}
\afterverb
%
The built-in function {\tt isinstance} takes a value and a
class object, and returns {\tt True} if the value is an instance
of the class.

If {\tt other} is a Time object, {\tt \_\_add\_\_} invokes
{\tt add\_time}.  Otherwise it assumes that the seconds parameter
is a number and invokes {\tt increment}.  This operation is
called a {\bf type-based dispatch} because it dispatches the
computation to different methods based on the type of the
arguments.

\index{type-based dispatch}

Here are examples that use the {\tt +} operator with different
types:

\beforeverb
\begin{verbatim}
>>> start = Time(9, 45)
>>> duration = Time(1, 35)
>>> print start + duration
11:20:00
>>> print start + 1337
10:07:17
\end{verbatim}
\afterverb
%
Unfortunately, this implementation of addition is not commutative.
If the integer is the first operand, you get

\beforeverb
\begin{verbatim}
>>> print 1337 + start
TypeError: unsupported operand type(s) for +: 'int' and 'instance'
\end{verbatim}
\afterverb
%
The problem is, instead of asking the Time object to add an integer,
Python is asking an integer to add a Time object, and it doesn't know
how to do that.  But there is a clever solution for this problem,
the {\tt radd} method, which stands for ``right-side add.''  This
method is invoked when a Time object appears on the right side of
the {\tt +} operator.  Here's the definition:

\beforeverb
\begin{verbatim}
# inside class Time:

    def __radd__(self, other):
        return self.__add__(other)
\end{verbatim}
\afterverb
%
And here's how it's used:

\beforeverb
\begin{verbatim}
>>> print 1337 + start
10:07:17
\end{verbatim}
\afterverb
%

\begin{ex}
Write an {\tt add} method for Points that works with either a
Point object or a tuple:  

\begin{itemize}

\item If the second operand is a Point, the method should return a new
Point whose $x$ coordinate is the sum of the $x$ coordinates of the
operands, and likewise for the $y$ coordinates.

\item If the second operand is a tuple, the method should add the
first element of the tuple to the $x$ coordinate and the second
element to the $y$ coordinate, and return a new Point with the result. 

\end{itemize}

\end{ex}

\section{Polymorphism}
\index{polymorphism}

Type-based dispatch is useful when it is necessary, but (fortunately)
it is not always necessary.  Often you can avoid it by writing functions
that work correctly for arguments with different types.

Many of the functions we wrote for strings will actually
work for any kind of sequence.
For example, in Section~\ref{histogram}
we used {\tt histogram} to count the number of times each letter
appears in a word.

\beforeverb
\begin{verbatim}
def histogram(s):
    d = {}
    for c in s:
        if c not in d:
            d[c] = 1
        else:
            d[c] = d[c]+1
    return d
\end{verbatim}
\afterverb
%
This function also works for lists, tuples, and even dictionaries,
as long as the elements of {\tt s} are hashable, so they can be used
as keys in {\tt d}.

\beforeverb
\begin{verbatim}
>>> t = ['spam', 'egg', 'spam', 'spam', 'bacon', 'spam']
>>> histogram(t)
{'bacon': 1, 'egg': 1, 'spam': 4}
\end{verbatim}
\afterverb
%
Functions that can work with several types are called
{\bf polymorphic}.

Many of the built-in functions are
polymorphic.  For example, {\tt sum} works with any kind
of sequence, as long as the elements support the addition
operator.

\beforeverb
\begin{verbatim}
>>> t = [1, 2.0, 42L]
>>> print sum(t)
45.0
\end{verbatim}
\afterverb
%
Since Time objects provide an {\tt add} method, they work
with {\tt sum}:

\beforeverb
\begin{verbatim}
>>> t1 = Time(7, 43)
>>> t2 = Time(7, 41)
>>> t3 = Time(7, 37)
>>> total = sum([t1, t2, t3])
>>> print total
23:01:00
\end{verbatim}
\afterverb
%
In general, if all of the operations inside a function 
work with a given type, then the function works with that type.

The best kind of polymorphism is the unintentional kind, where
you discover that a function you have already written can be
applied to a type you never planned for.


%\section{Debugging}

%Minimum debuggable units (instantiate and print).

%Build debugging scaffolding as part of the program.



\section{Glossary}

\begin{description}

\item[object-oriented language:] A language that provides
features, such as user-defined classes and inheritance, that facilitate
object-oriented programming.
\index{object-oriented language}

\item[object-oriented programming:] A style of programming in which
data and the operations that manipulate it are organized into classes
and methods.
\index{object-oriented programming}

\item[method:] A function that is defined inside a class definition and
is invoked on instances of that class.
\index{method}

\item[subject:] The object a method is invoked on.
\index{subject}

\item[operator overloading:] Changing the behavior of an operator like
{\tt +} so it works with a user-defined type.
\index{overloading}
\index{operator!overloading}

\item[type-based dispatch:] A programming pattern that check the type
of an operand and invokes different functions for different types.
\index{type-based dispatch}

\item[polymorphic:] Pertaining to a function that can work with more than one
type.
\index{polymorphic}

\end{description}


\chapter{Inheritance}

In this chapter we will develop classes to represent playing cards,
decks of cards, and poker hands.  If you don't play poker, don't
worry; I'll tell you what you need to know for the exercises.

But if you are not familiar with common playing cards, now would be a
good time to get a deck, or else this chapter might not make much
sense.


\section{Card objects}
\index{Card}
\index{class!Card}

There are fifty-two cards in a deck, each of which belongs to one of
four suits and one of thirteen ranks.  The suits are Spades, Hearts,
Diamonds, and Clubs (in descending order in bridge).  The ranks are
Ace, 2, 3, 4, 5, 6, 7, 8, 9, 10, Jack, Queen, and King.  Depending on
the game that you are playing, an Ace may be higher than King
or lower than 2.

\index{rank}
\index{suit}

If we want to define a new object to represent a playing card, it is
obvious what the attributes should be: {\tt rank} and
{\tt suit}.  It is not as obvious what type the attributes
should be.  One possibility is to use strings containing words like
{\tt "Spade"} for suits and {\tt "Queen"} for ranks.  One problem with
this implementation is that it would not be easy to compare cards to
see which had a higher rank or suit.

\index{encode}
\index{encrypt}
\index{map to}

An alternative is to use integers to {\bf encode} the ranks and suits.
In this context, ``encode'' means that we are going to define a mapping
between numbers and suits, or between numbers and ranks.  This
kind of encoding is not meant to be a secret (that
would be ``encryption'').

For example, this table shows the suits and the corresponding integer
codes:

\beforefig
\begin{tabular}{l c l}
Spades & $\mapsto$ & 3 \\
Hearts & $\mapsto$ & 2 \\
Diamonds & $\mapsto$ & 1 \\
Clubs & $\mapsto$ & 0
\end{tabular}
\afterfig

This code makes it easy to compare cards; because higher suits map to
higher numbers, we can compare suits by comparing their codes.

The mapping for ranks is fairly obvious; each of the numerical ranks
maps to the corresponding integer, and for face cards:

\beforefig
\begin{tabular}{l c l}
Jack & $\mapsto$ & 11 \\
Queen & $\mapsto$ & 12 \\
King & $\mapsto$ & 13 \\
\end{tabular}
\afterfig

I am using the $\mapsto$ symbol to make is clear that these mappings
are not part of the Python program.  They are part of the program
design, but they don't appear explicitly in the code.

The class definition for {\tt Card} looks like this:

\beforeverb
\begin{verbatim}
class Card:
    """represents a standard playing card."""

    def __init__(self, suit=0, rank=2):
        self.suit = suit
        self.rank = rank
\end{verbatim}
\afterverb
%
As usual, the {\tt init} method takes an optional
parameter for each attribute.  The default card is
the 2 of Clubs.

To create a Card, you call {\tt Card} with the
suit and rank of the card you want.

\beforeverb
\begin{verbatim}
threeOfClubs = Card(3, 1)
\end{verbatim}
\afterverb
%
In the next section we'll figure out which card that is.


\section{Class attributes}
\index{class attribute}
\index{attribute!class}

In order to print Card objects in a way that people can easily
read, we need a mapping from the integer codes to the corresponding
ranks and suits.  A natural way to
do that is with lists of strings.  We assign these lists to {\bf class
attributes}:

\beforeverb
\begin{verbatim}
# inside class Card:

    suit_names = ['Clubs', 'Diamonds', 'Hearts', 'Spades']
    rank_names = [None, 'Ace', '2', '3', '4', '5', '6', '7', 
              '8', '9', '10', 'Jack', 'Queen', 'King']

    def __str__(self):
        return '%s of %s' % (Card.rank_names[self.rank],
                             Card.suit_names[self.suit])
\end{verbatim}
\afterverb
%
Because {\tt suit\_names} and {\tt rank\_names}
are defined outside of any method, they are class attributes;
that is, they are associated with the class {\tt Card} rather
than with a particular Card instance.

Attributes like {\tt suit} and {\tt rank} are more precisely called
{\bf instance attributes} because they are associated with a
particular instance.

Both kinds of attribute are accessed using dot notation.  For
example, in {\tt \_\_str\_\_}, {\tt self} is a Card object,
and {\tt self.rank} is its rank.  Similarly, {\tt Card}
is a class object, and {\tt Card.rank\_names} is a
list of strings associated with the class.

Every card has its own {\tt suit} and {\tt rank}, but there
is only one copy of {\tt suit\_names} and {\tt rank\_names}.

Finally, the expression \verb+Card.rank_names[self.rank]+ means ``use
the attribute {\tt rank} from the object {\tt self} as an index into
the list {\tt rank\_names} from the class {\tt Card}, and select the
appropriate string.''

The first element of {\tt rank\_names} is {\tt None} because there
is no card with rank zero.  By including {\tt None} as a place-keeper,
we get a mapping with the nice property that the index 2 maps to the
string {\tt '2'}, and so on.

With the methods we have so far, we can create and print cards:

\beforeverb
\begin{verbatim}
>>> card1 = Card(1, 11)
>>> print card1
Jack of Diamonds
\end{verbatim}
\afterverb
%
Here is a diagram that shows the {\tt Card} class object
and one Card instance:

\index{state diagram}
\index{diagram!state}
\index{object diagram}
\index{diagram!object}

\beforefig
\centerline{\includegraphics{figs/card1.eps}}
\afterfig

{\tt Card} is a class object, so it has type {\tt classobj}.  {\tt
card1} has type {\tt Card}.  (To save space, I didn't draw the
contents of {\tt suit\_names} and {\tt rank\_names}).


\section{Comparing cards}
\label{comparecard}
\index{operator!conditional}
\index{conditional operator}

For built-in types, there are conditional operators
({\tt <}, {\tt >}, {\tt ==}, etc.)
that compare
values and determine when one is greater than, less than, or equal to
another.  For user-defined types, we can override the behavior of
the built-in operators by providing a method named
{\tt \_\_cmp\_\_}.  

The {\tt cmp} method takes two parameters, {\tt self} and {\tt other},
and returns a positive number if the first object is greater, a
negative number if the second object is greater, and 0 if they are
equal to each other.

\index{override}
\index{operator overloading}

The correct ordering for cards is not obvious.
For example, which
is better, the 3 of Clubs or the 2 of Diamonds?  One has a higher
rank, but the other has a higher suit.  In order to compare
cards, you have to decide whether rank or suit is more important.

The answer might depend on what game you are playing, but to keep
things simple, we'll make the arbitrary choice that suit is more
important, so all of the Spades outrank all of the Diamonds,
and so on.

With that decided, we can write {\tt \_\_cmp\_\_}:

\beforeverb
\begin{verbatim}
# inside class Card:

    def __cmp__(self, other):
        # check the suits
        if self.suit > other.suit: return 1
        if self.suit < other.suit: return -1

        # suits are the same... check ranks
        if self.rank > other.rank: return 1
        if self.rank < other.rank: return -1

        # ranks are the same... it's a tie
        return 0    
\end{verbatim}
\afterverb
%
You can write this more concisely using tuple comparison:

\beforeverb
\begin{verbatim}
# inside class Card:

    def __cmp__(self, other):
        t1 = self.suit, self.rank
        t2 = other.suit, other.rank
        return cmp(t1, t2)
\end{verbatim}
\afterverb
%
The built-in function {\tt cmp} has the same interface as
the method {\tt \_\_cmp\_\_}: it takes two values and returns
a positive number if the first is larger, a negative number
of the second is larger, and 0 if they are equal.



\begin{ex}
Write a {\tt \_\_cmp\_\_} method for Time objects.  Hint: you
can use tuple comparison, but you also might consider using
integer subtraction.

%    def __cmp__(self, other):
%        return time_to_int(self) - time_to_int(other)

%If {\tt self} is later than {\tt other}, the result is
%a positive number.  If {\tt other} is later, the result
%is negative.  And if {\tt self} and {\tt other} are equal
%(but not necessarily identical)
%the result is zero.

\end{ex}


\section{Decks}
\index{list!of objects}
\index{object!list of}
\index{deck}

Now that we have Card objects, the next
step is to define a class to represent decks.  Since a
deck is made up of cards, a natural choice is for 
each Deck object to contain a
list of cards as an attribute.

\index{initialization method}
\index{method!initialization}

The following is a class definition for {\tt Deck}.  The
{\tt init} method creates the attribute {\tt cards} and generates
the standard set of fifty-two cards:

\index{composition}
\index{loop!nested}

\beforeverb
\begin{verbatim}
class Deck:

    def __init__(self):
        self.cards = []
        for suit in range(4):
            for rank in range(1, 14):
                card = Card(suit, rank)
                self.cards.append(card)
\end{verbatim}
\afterverb
%
The easiest way to populate the deck is with a nested loop.  The outer
loop enumerates the suits from 0 to 3.  The inner loop enumerates the
ranks from 1 to 13.  Each iteration of the inner loop
creates a new Card with the current suit and rank,
and appends it to {\tt self.cards}.

\index{{\tt append}}


\section{Printing the deck}
\label{printdeck}
\index{printing!deck object}

Here is a {\tt str} method for {\tt Deck}:

\beforeverb
\begin{verbatim}
#inside class Deck:

    def __str__(self):
        res = []
        for card in self.cards:
            res.append(str(card))
        return '\n'.join(res)
\end{verbatim}
\afterverb
%
This method demonstrates an efficient way to accumulate a large
string, by building a list of strings and then using {\tt join}.
The built-in function {\tt str} invokes the {\tt \_\_str\_\_}
method on each card and returns the string representation.

Since we invoke {\tt join} on a newline character, the cards
are separated by newlines.  Here's what the result looks like:

\beforeverb
\begin{verbatim}
>>> deck = Deck()
>>> print deck
Ace of Clubs
2 of Clubs
3 of Clubs
...
10 of Spades
Jack of Spades
Queen of Spades
King of Spades
\end{verbatim}
\afterverb
%
Even though the result appears on 52 lines, it is
one long string that contains newlines.


\section{Add, remove, shuffle and sort}

To deal cards, we would like a method that
removes a card from the deck and returns it.
The list method {\tt pop} provides a convenient way to do that:

\beforeverb
\begin{verbatim}
#inside class Deck:

    def pop_card(self):
        return self.cards.pop()
\end{verbatim}
\afterverb
%
Since {\tt pop} removes the {\em last} card in the list, we are
in effect dealing from the bottom of the deck.

To add a card, we can use the list method {\tt append}:

\beforeverb
\begin{verbatim}
#inside class Deck:

    def add_card(self, card):
        self.cards.append(card)
\end{verbatim}
\afterverb
%
A method like this that uses another function without doing
much real work is sometimes called a {\bf veneer}.  The metaphor
comes from woodworking, where it is common to glue a thin
layer of good quality wood to the surface of a cheaper piece of
wood.

\index{veneer}

In this case we are defining a ``thin'' method that expresses
a list operation in terms that are appropriate for decks.

As another example, we can write a Deck method named {\tt shuffle}
using the function {\tt shuffle} from the {\tt random} module:

\index{shuffle}

\beforeverb
\begin{verbatim}
# inside class Deck:
            
    def shuffle(self):
        random.shuffle(self.cards)
\end{verbatim}
\afterverb
%
Don't forget to import {\tt random}.

\index{sort}

\begin{ex}
Write a Deck method named {\tt sort} that uses the list method
{\tt sort} to sort the cards in a {\tt Deck}.  {\tt sort} uses
the {\tt \_\_cmp\_\_} method we defined to determine sort order.
\end{ex}



\section{Inheritance}
\index{inheritance}
\index{object-oriented programming}
\index{parent class}
\index{child class}
\index{subclass}

The language feature most often associated with object-oriented
programming is {\bf inheritance}.  Inheritance is the ability to
define a new class that is a modified version of an existing
class.

It is called ``inheritance'' because the new class inherits the
methods of the existing class.  Extending this metaphor, the existing
class is called the {\bf parent} class and the new class is
called the {\bf child}.

As an example, let's say we want a class to represent a ``hand,''
that is, the set of cards held by one player.  A hand is similar to a
deck: both are made up of a set of cards, and both require operations
like adding and removing cards.

A hand is also different from a deck; there are operations we want for
hands that don't make sense for a deck.  For example, in poker we
might compare two hands to see which one wins.  In bridge, we might
compute a score for a hand in order to make a bid.

This relationship between classes---similar, but different---lends
itself to inheritance.  

The definition of a child class is like other class definitions,
but the name of the parent class appears in parentheses:

\index{parent class}
\index{class!parent}

\beforeverb
\begin{verbatim}
class Hand(Deck):
    """represents a hand of playing cards"""
\end{verbatim}
\afterverb
%
This definition indicates that {\tt Hand} inherits from {\tt Deck};
that means we can use methods like {\tt pop\_card} and {\tt add\_card}
for Hands as well as Decks.

{\tt Hand} also inherits the {\tt init} method from {\tt Deck}, but
it doesn't really do what we want: instead of populating the hand
with 52 new cards, the {\tt init} method for Hands should initialize
{\tt cards} with an empty list.

If we provide an
{\tt init} method in the {\tt Hand} class, it overrides the one
in the {\tt Deck} class:

\beforeverb
\begin{verbatim}
# inside class Hand:

    def __init__(self, label=''):
        self.cards = []
        self.label = label
\end{verbatim}
\afterverb
%
So when you create a Hand, Python invokes this {\tt init}
method:

\beforeverb
\begin{verbatim}
>>> hand = Hand('new hand')
>>> print hand.cards
[]
>>> print hand.label
new hand
\end{verbatim}
\afterverb
%
But the other methods are inherited from {\tt Deck}, so we can use
{\tt pop\_card} and {\tt add\_card} to deal a card:

\beforeverb
\begin{verbatim}
>>> deck = Deck()
>>> card = deck.pop_card()
>>> hand.add_card(card)
>>> print hand
King of Spades
\end{verbatim}
\afterverb
%
The next natural step is to encapsulate this code in a method
called {\tt move\_cards}:

\beforeverb
\begin{verbatim}
#inside class Deck:

    def move_cards(self, hand, num):
        for i in range(num):
            hand.add_card(self.pop_card())
\end{verbatim}
\afterverb
%
{\tt move\_cards} takes two arguments, a Hand object and the number of
cards to deal.  It modifies both {\tt self} and {\tt hand}, and
returns {\tt None}.

In some games, cards are moved from one hand to another,
or from a hand back to the deck.  You can use {\tt move\_cards}
for any of these operations: {\tt self} can be either a Deck
or a Hand, and {\tt hand}, despite the name, can also be a {\tt Deck}.

\begin{ex}
Write a Deck method called {\tt deal\_hands} that takes two
parameters, the number of hands and the number of cards per
hand, and that creates new Hand objects, deals the appropriate
number of cards per hand, and returns a list of Hand objects.
\end{ex}

Inheritance is a useful feature.  Some programs that would be
repetitive without inheritance can be written more elegantly
with it.  Inheritance can facilitate code reuse, since you can
customize the behavior of parent classes without having to modify
them.  In some cases, the inheritance structure reflects the natural
structure of the problem, which makes the program easier to
understand.

On the other hand, inheritance can make programs difficult to read.
When a method is invoked, it is sometimes not clear where to find its
definition.  The relevant code may be scattered among several modules.
Also, many of the things that can be done using inheritance can be
done as well or better without it.  


%\section{Writing modules for import}




\section{Class diagrams}

So far we have seen stack diagrams, which show the state of
a program, and object diagrams, which show the attributes
of an object and their values.  These diagrams represent a snapshot
in the execution of a program, so they change as the program
runs.

They are also highly detailed, and for some applications, too
detailed.  A class diagrams is a more abstract representation
of the structure of a program.  Instead of showing individual
objects, it shows classes and the relationships between them.

There are several kinds of relationship between classes:

\begin{itemize}

\item Objects in one class might contain references to objects
in another class.  For example, each Rectangle contains a reference
to a Point, and each Deck contains references to many Cards.
This kind of relationship is called {\bf HAS-A}, as in, ``a Rectangle
has a Point.''

\item One class might inherit from another.  This relationship
is called {\bf IS-A}, as in, ``a Hand is a kind of a Deck.''

\item Once class might depend on another in the sense that changes
in one class would require changes in the other.

\end{itemize}

A {\bf class diagram} is a graphical representation of these
relationships between classes.  For example, this diagram
shows the relationships between {\tt Card}, {\tt Deck} and
{\tt Hand}.

\beforefig
\centerline{\includegraphics{figs/class1.eps}}
\afterfig

The arrow with a hollow triangle head represents an IS-A
relationship; in this case it indicates that Hand inherits
from Deck.

The standard arrow head represents a HAS-A
relationshop; in this case a Deck has references to Card
objects.

The star ({\tt *}) near the arrow head is a 
{\bf multiplicity}; it indicates how many Cards a Deck has.
A multiplicity can be a simple number, like {\tt 52}, a range,
like {\tt 5..7} or a star, which indicates that a Deck can
have any number of Cards.



\section{Glossary}

\begin{description}

\item[encode:]  To represent one set of values using another
set of values by constructing a mapping between them.
\index{encode}

\item[class attribute:] An attribute associated with a class
object.  Class attributes are defined inside
a class definition but outside any method.
\index{class attribute}
\index{attribute!class}

\item[instance attribute:] An attribute associated with an
instance of a class.
\index{instance attribute}
\index{attribute!instance}

\item[veneer:] A method or function that provides a different
interface to another function without doing much computation.
\index{veneer}

\item[inheritance:] The ability to define a new class that is a
modified version of a previously defined class.
\index{inheritance}

\item[parent class:] The class from which a child class inherits.
\index{parent class}

\item[child class:] A new class created by inheriting from an
existing class; also called a ``subclass.''
\index{child class}

\item[IS-A relationship:] The relationship between a child class
and its parent class.
\index{IS-A}

\item[HAS-A relationship:] The relationship between two classes
where instances of one class contain references to instances of
the other.
\index{HAS-A}

\item[class diagram:] A diagram that shows the classes in a program
and the relationships between them.
\index{class diagram}
\index{diagram!class}

\item[multiplicity:] A notation in a class diagram that shows, for
a HAS-A relationship, how many references there are to instances
of another class.
\index{multiplicity}

\end{description}


\section{Exercises}

The following are the possible hands in poker, in increasing order
of value (and decreasing order of probability):

\begin{description}

\item[pair:] two cards with the same rank
\vspace{-0.05in}

\item[two pair:] two pairs of cards with the same rank
\vspace{-0.05in}

\item[three of a kind:] three cards with the same rank
\vspace{-0.05in}

\item[straight:] five cards with ranks in sequence (aces can
be high or low, so {\tt Ace-2-3-4-5} is a straight and so is {\tt
10-Jack-Queen-King-Ace}, but {\tt Queen-King-Ace-2-3} is not.)
\vspace{-0.05in}

\item[flush:] five cards with the same suit
\vspace{-0.05in}

\item[full house:] three cards with one rank, two cards with another
\vspace{-0.05in}

\item[four of a kind:] four cards with the same rank
\vspace{-0.05in}

\item[straight flush:] five cards in sequence (as defined above) and
with the same suit
\vspace{-0.05in}

\end{description}
%
The goal of these exercises is to estimate
the probability of drawing these various hands.

\begin{enumerate}

\item Download the following files from \url{thinkpython.com/code}:

\begin{description}

\item[{\tt Card.py}]: A complete version of the {\tt Card},
{\tt Deck} and {\tt Hand} classes in this chapter.

\item[{\tt PokerHand.py}]: An incomplete implementation of a class
that represents a poker hand, and some code that tests it.

\end{description}
%
\item If you run {\tt PokerHand.py}, it deals six 7-card poker hands
and checks to see if any of them contains a flush.  Read this
code carefully before you go on.

\item Add methods to {\tt PokerHand.py} named {\tt has\_pair},
{\tt has\_twopair}, etc. that return True or False according to
whether or not the hand meets the relevant criteria.  Your code should
work correctly for ``hands'' that contain any number of cards
(although 5 and 7 are the most common sizes).

\item Write a method named {\tt classify} that figures out
the highest-value classification for a hand and sets the
{\tt label} attribute accordingly.  For example, a 7-card hand
might contain a flush and a pair; it should be labeled ``flush''.

\item When you are convinced that your classification methods are
working, the next step is to estimate the probablities of the various
hands.  Write a function in {\tt PokerHand.py} that shuffles a deck of
cards, divides it into hands, classifies the hands, and counts the
number of times various classifications appear.

\item Print a table of the classifications and their probabilities.
Run your program with larger and larger numbers of hands until the
output values converge to a reasonable degree of accuracy.

\end{enumerate}


\chapter{Case study: Tkinter}
\index{Tkinter}

Most of the programs we have seen so far are text-based, but
many programs use {\bf graphical user interfaces}, also
known as {\bf GUIs}.

Python provides several choices for writing GUI-based programs,
including wxPython, Tkinter, and Qt.  Each has pros and cons, which
is why Python has not converged on a standard.

The one I will present in this chapter is Tkinter because I think
it is the easiest to get started with.  Most of the concepts
in this chapter apply to the other GUI modules, too.

One drawback of Tkinter is that it is based on another language,
called Tk (Tkinter is the Python interface to Tk, hence the name).
The documentation of Tkinter is written in ...

I have written a module called {\tt Gui.py} that provides
a simplified interface to Tkinter.  I will present ...

There are several books and web pages about Tkinter.  One of
the best online resources is {\em An Introduction to Tkinter}
by Fredrik Lundh.


\section{Widgets}

Graphical user interfaces are made up of elements called
{\bf widgets}.  Common widgets include:

\begin{description}

\item[Button:] A widget, containing text or an image, that
performs an action when pressed.

\item[Canvas:] A region that can display lines, rectangles,
circles and other shapes.

\item[Entry:] A region where users can type text.

\item[Scrollbar:] A widget that controls the view of another
widget.

\item[Frame:] A container, often invisible, that contains other
widgets.

\end{description}

To create a GUI, you have to import {\tt Gui} and instantiate
a Gui object:

\beforeverb
\begin{verbatim}
>>> from Gui import *
>>> g = Gui()
\end{verbatim}
\afterverb
%
When you run this code, a window should appear with an empty
gray square and the title {\sf tk}.  To change the title, you
can invoke {\tt title} on {\tt g}:

\beforeverb
\begin{verbatim}
>>> g.title('The Title')
\end{verbatim}
\afterverb
%
The empty gray square is a Frame.  When you create a new widget,
it is added to this Frame.


\section{Buttons and callbacks}

The method {\tt bu} creates a Button widget:

\beforeverb
\begin{verbatim}
>>> button = g.bu(text='Press me.')
\end{verbatim}
\afterverb
%
The return value from {\tt bu} is a Button object.  The button
that appears in the Frame is a graphical representation of this
object; you can control the button by invoking methods on it.

{\tt bu} takes up to 32 parameters that control the appearance
and function of the button.  These parameters are called
{\bf options}.  Instead of providing values for all 32 options,
you can use {\bf keyword arguments}, like {\tt text='Press me.'},
to specify only the options you need and use the default
values for the rest.

When you add a widget to the Frame, it gets ``shrink-wrapped;''
that is, the Frame shrinks to the size of the Button.  If you
add more widgets, the Frame grows to accomodate them.

For example, the method {\tt la} creates a Label widget:

\beforeverb
\begin{verbatim}
>>> label = g.la(text='Press the button.')
\end{verbatim}
\afterverb
%
By default, Tkinter stacks the widgets top-to-bottom and centers
them.  We'll see how to override that behavior later.

If you press the button, you will see that it doesn't do much.
That's because you haven't ``wired it up;'' that is, you haven't
told it what to do!

The option that controls the behavior of a button is {\tt command}.
The value of {\tt command} is a function that gets executed when
the button is pressed.  For example, here is a function that creates
a new Label:

\beforeverb
\begin{verbatim}
def make_label():
    g.la(text='Thank you.')
\end{verbatim}
\afterverb
%
Now we can create a button with this function as its command:

\beforeverb
\begin{verbatim}
button2 = g.bu(text='No, press me!', command=make_label)
\end{verbatim}
\afterverb
%
When you press this button, it should execute {\tt make\_label}
and a new label should appear.

The value of {\tt command} is a function object.  It is a
common error to call this function rather than passing a reference
to it, like this:

\beforeverb
\begin{verbatim}
button3 = g.bu(text='This is wrong!', command=make_label())
\end{verbatim}
\afterverb
%
If you run this code, you will see that it calls {\tt make\_label}
immediately, and {\em then} creates the button.  When you
press the button, it does nothing because the return value
from {\tt make\_label} is {\tt None}.

A function used as a Button command is called a {\bf callback} because
after you call {\tt bu} to create the button, the flow of execution
``calls back'' when the user presses the button.

This kind of flow is characteristic of {\bf event-driven programming}.
User actions, like button presses and key strokes, are called {\bf
events}.  In event-driven programming, the flow of execution is
determined by user actions rather than by the programmer.  

The challenge of event-driven programming is to construct a set of
widgets and callbacks that works correctly (or at least generate
appropriate error messages) for any sequence of user actions.

\begin{ex}
Write a program that creates a GUI with a single button.  When the
button is pressed it should create a second button.  When
{\em that} button is pressed, it should create a label that
says, ``Nice job!''.

\end{ex}


\section{Canvas widgets}

One of the most versatile widgets is the Canvas, which creates
a region for drawing lines, circles and other shapes.  The
method {\tt ca} creates a new Canvas:

\beforeverb
\begin{verbatim}
>>> canvas = g.ca(width=500, height=500)
\end{verbatim}
\afterverb
%
{\tt width} and {\tt height} are the dimensions of the canvas
in pixels.  By default, the background of the canvas is gray,
but you can override it with the {\tt bg} option.

You can change the options of a widget at any time with the
{\tt config} method:

\beforeverb
\begin{verbatim}
>>> canvas.config(bg='white')
\end{verbatim}
\afterverb
%
{\tt config} takes the same set of options as the method
that created the widget.  The value of {\tt bg} is a string
that names a color.  The set of legal color names is different
for different implementations of Python, but all implementations
provide at least the following colors:

\beforeverb
\begin{verbatim}
white   black
red     green    blue   
cyan    yellow   magenta
\end{verbatim}
\afterverb
%
Shapes on a Canvas are called {\bf items}.  For example,
the Canvas method {\tt circle} draws (you guessed it) a circle:

\beforeverb
\begin{verbatim}
>>> item = canvas.circle([0,0], 100, fill='red')
\end{verbatim}
\afterverb
%
The first argument is a coordinate pair that specifies the
center of the circle; the second is the radius.

{\tt Gui.py} provides a standard Cartesian coordinate system with
the origin at the center of the Canvas and the positive $y$ axis
pointing up.  This is different from some other graphics systems
where the the origin is in the upper left with the $y$ axis
pointing down.

The {\tt fill} option specifies that the circle should be filled
in with red.

The return value from {\tt circle} is an Item object that
provides methods for modifying the item on the canvas.  For
example, you can use {\tt config} to change any of the circle's
options:

\beforeverb
\begin{verbatim}
>>> item.config(fill='yellow', outline='orange', width=10)
\end{verbatim}
\afterverb
%
{\tt width} is the thickness of the outline in pixels;
{\tt outline} is the color.

\begin{ex}
\label{circle}
Write a program that creates a Canvas and a Button.  When the
user presses the Button, it should draw a circle on the canvas.
\end{ex}


\section{Coordinate sequences}

The {\tt rectangle} method takes a sequence of coordinates that
specify opposite corners of the rectangle.  This example
draws a green rectangle with the lower left corner at the origin
and the upper right corner at $(200, 100)$:

\beforeverb
\begin{verbatim}
>>> canvas.rectangle([[0, 0], [200, 100]], 
                     fill='blue', outline='orange', width=10)
\end{verbatim}
\afterverb
%
This way of specifying corners is called
a {\bf bounding box} because the two points
bound the rectangle.

{\tt oval} takes a bounding box and draws an oval
within the specified rectangle:

\beforeverb
\begin{verbatim}
>>> canvas.oval([[0, 0], [200, 100]], outline='orange', width=10)
\end{verbatim}
\afterverb
%
{\tt line} takes a sequence of coordinates and draws
a line that connects the points.  This example draws two legs
of a triangle:

\beforeverb
\begin{verbatim}
>>> canvas.line([[0, 100], [100, 200], [200, 100]], width=10)
\end{verbatim}
\afterverb
%
{\tt polygon} takes the same arguments, but it draws the last
leg of the polygon (if necessary) and fills it in:

\beforeverb
\begin{verbatim}
>>> canvas.polygon([[0, 100], [100, 200], [200, 100]],
                   fill='red', outline='orange', width=10)
\end{verbatim}
\afterverb
%


\section{More widgets}

Tkinter provides two widgets that let users type text: an
Entry, which is a single line, and a Text widget, which has
multiple lines.

{\tt en} creates a new Entry:

\beforeverb
\begin{verbatim}
>>> entry = g.en(text='Default text.')
\end{verbatim}
\afterverb
%
The {\tt text} option allows you to put text into the entry
when it is created.  The {\tt get} method returns the contents
of the Entry (which may have been changed by the user):

\beforeverb
\begin{verbatim}
>>> entry.get()
'Default text.'
\end{verbatim}
\afterverb
%
{\tt te} creates a text widget:

\beforeverb
\begin{verbatim}
>>> text = g.te(width=100, height=5)
\end{verbatim}
\afterverb
%
{\tt width} and {\tt height} are the dimensions of the
widget in characters and lines.

{\tt insert} puts text into the Text widget:

\beforeverb
\begin{verbatim}
>>> text.insert(END, 'A line of text.')
\end{verbatim}
\afterverb
%
{\tt END} is a special index that indicates the last character in
the Text widget.  You can also indicate a character in dot notation,
with the line number before the dot and the column number after.
This example adds the letters {\tt 'nother'} after the first character
of the first line.

\beforeverb
\begin{verbatim}
>>> text.insert(1.1, 'nother')
\end{verbatim}
\afterverb
%
This example returns all the text in the widget, including
a newline between lines:

\beforeverb
\begin{verbatim}
>>> text.get(0.0, END)
'Another line of text.\n'
\end{verbatim}
\afterverb
%

And this example deletes all but the first two characters:

\beforeverb
\begin{verbatim}
>>> text.delete(1.2, END)
>>> text.get(0.0, END)
'An\n'
\end{verbatim}
\afterverb
%

\begin{ex}
Modify your solution to Exercise~\ref{circle} by adding an
Entry widget and a second button.  When the user presses the
second button, it should read a color name from the Entry and
use it to change the fill color of the circle.
\end{ex}


\section{Packing widgets}

So far we have been stacking widgets in a single column, but in most
GUIs the layout is more complicated.  For example, here is a slightly
simplified version of TurtleWorld (see
Chapter~\ref{turtlechap}).

\beforefig
\centerline{
\includegraphics[width=1.0\textwidth]{figs/screenshot.eps}
}
\afterfig

This section presents the code that creates this GUI, broken into a
series of steps.  

At the top level, this GUI contains two widgets---a Canvas and a
Frame---arranged in a row.  So the first step is to create the row.

\beforeverb
\begin{verbatim}
class SimpleTurtleWorld(TurtleWorld):
    """This class is identical to TurtleWorld, but the code that
    lays out the GUI is simplified for explanatory purposes."""

    def setup(self):
        self.row([1,0])
        ...
\end{verbatim}
\afterverb
%
{\tt setup} is the function that creates and arranges the widgets.
Arranging widgets in a GUI is called {\bf packing}.

{\tt row} creates a row Frame and makes it the ``current Frame.''
Until this Frame is closed or another Frame is created, all
subsequent widgets are packed in a row.

The argument to {\tt row} is a list of weights, which
determine how extra space is allocated between widgets.  
The list {\tt [1,0]} means that all extra space is allocated
to the first widget, which is the Canvas.  If you run this code
and resize the window, you will see that the Canvas grows and
the column of other widgets doesn't.

Here is the code that creates the Canvas and the column:

\beforeverb
\begin{verbatim}
        self.canvas = self.ca(width=400, height=400, bg='white')
        self.col([0,0,1])
\end{verbatim}
\afterverb
%
{\tt col} creates a column Frame; the argument, again, is a list of
weights.  In this case, the third widget (which we haven't created
yet) gets the extra space.

The first widget in the column is a grid Frame, which contains
four buttons arranged two-by-two:

\beforeverb
\begin{verbatim}
        self.gr(2, [1,1], [1,1])
        self.bu(text='Print canvas', command=self.canvas.dump)
        self.bu(text='Quit', command=self.quit)
        self.bu(text='Make Turtle', command=self.make_turtle)
        self.bu(text='Clear', command=self.clear)
        self.endfr()
\end{verbatim}
\afterverb
%
{\tt gr} creates the grid; the arguments are:

\begin{itemize}

\item The number of columns in the grid.  Widgets in the grid are
layed out left-to-right, top-to-bottom.

\item The column weights, which determine how extra space is
allocated between the columns in the grid.

\item The row weights, which determine how extra space is
allocated between the rows.

\end{itemize}

In this example, extra space is allocated equally to all four
buttons.

The first button uses {\tt self.canvas.dump} as a callback; the second
uses {\tt self.quit}.  These are {\bf bound methods}, which means they
are associated with a particular object.  When they are invoked, they
are invoked on that object.  {\tt self} is the TurtleWorld that is the
subject of this method and {\tt self.canvas} is the Canvas we created
in the previous block of code.

The next widget in the column is a row Frame that contains
a Button and an Entry:

\beforeverb
\begin{verbatim}
        self.row([0,1], pady=30)
        self.bu(text='Run file', command=self.run_file)
        self.en_file = self.en(text='snowflake.py', width=5)
        self.endrow()
\end{verbatim}
\afterverb
%
In this case, extra space is allocated to the second widget in
the row, the Entry.

The option {\tt pady} ``pads'' this row in the $y$ direction,
adding 30 pixels of space above and below.

{\tt endrow} ends this row of widgets, so subsequent widgets are
packed in the column Frame.  {\tt Gui.py} keeps a stack of Frames:

\begin{itemize}

\item When you use {\tt row}, {\tt col} or {\tt gr} to create a Frame,
it goes on top of the stack and becomes the current Frame.

\item When you use {\tt endrow}, {\tt endcol} or {\tt endgr} to close
a Frame, it gets popped off the stack and the previous Frame on the
stack becomes the current Frame.

\end{itemize} 

The method {\tt run\_file} reads the contents of the Entry,
uses it as a filename to open the file, reads the contents,
and passes it to {\tt run\_code}.  {\tt self.inter} is an
Interpreter object that knows how to take a string and
execute it as Python code.

\beforeverb
\begin{verbatim}
    def run_file(self):
        filename = self.en_file.get()
        fp = open(filename)
        source = fp.read()
        self.inter.run_code(source, filename)
\end{verbatim}
\afterverb
%
The last two widgets are a Text widget and a Button:

\beforeverb
\begin{verbatim}
        self.te_code = self.te(width=25, height=10)
        self.te_code.insert(END, 'world.clear()\n')
        self.te_code.insert(END, 'bob = Turtle(world)\n')

        self.bu(text='Run code', command=self.run_text)

        self.endcol()
\end{verbatim}
\afterverb
%
{\tt run\_text} is similar to {\tt run\_file} except that it takes
the code from the Text widget instead of from a file:

\beforeverb
\begin{verbatim}
    def run_text(self):
        source = self.te_code.get(1.0, END)
        self.inter.run_code(source, '<user-provided code>')
\end{verbatim}
\afterverb
%
Unfortunately, the details of packing widgets are different in
other languages, and in different Python modules.
Tkinter alone provides three different mechanisms for arranging
widgets.  These mechanisms are called {\bf geometry managers}.
The one I demonstrated in this section is the ``grid'' geometry
manager; the others are called ``pack'' and ``place''.

Fortunately, most of the concepts in this section apply to
other GUI modules and other languages.


\section{Menus and Callables}

A Menubutton is a widget that looks like a button, but when pressed
it pops up a menu.  After the user selects an item, the menu
disappears.

Here is code that creates a color selection Menubutton:

% mb_example.py

\beforeverb
\begin{verbatim}
g = Gui()
g.la('Select a color:')
colors = ['red', 'green', 'blue']
mb = g.mb(text=colors[0])
\end{verbatim}
\afterverb
%
{\tt mb} creates the Menubutton.  Initially, the text on the button is
the name of the default color.  The following loop creates one menu
item for each color:

\beforeverb
\begin{verbatim}
for color in colors:
    g.mi(mb, text=color, command=Callable(set_color, color))
\end{verbatim}
\afterverb
%
The first argument of {\tt mi} is the Menubutton these items are
associated with.

The {\tt command} option is a Callable object, which is something new.
So far we have seen functions and bound methods used as callbacks,
which works fine as long as you don't have to pass any arguments to
the function.  To do that, you have to construct a Callable object
that contains the function, like {\tt set\_color} and the arguments,
like {\tt color}.

The Callable object stores a reference to the function and the
arguments as attributes.  Later, when the user clicks on a menu
item, the callback calls the function and passes the stored
arguments.

Here is what {\tt set\_color} might look like:

\beforeverb
\begin{verbatim}
def set_color(color):
    mb.config(text=color)
    print color
\end{verbatim}
\afterverb
%
When the user selects a menu item and {\tt set\_color} is called,
it configures the Menubutton to display the newly-selected color.
It also print the color; if you try this example, you can confirm that
{\tt set\_color} is called when you select an item (and {\em not}
called when you create the Callable object).


\section{Binding}

% canvas_example.py

A {\bf binding} is an association between a widget, an event and a
callback: when an event (like a button press) happens on a widget, the
callback is invoked.

Many widgets have default bindings.  For example, when you press
a button, the default binding changes the relief of the button
to make it look depressed.  When you release the button, the
binding restores the appearance of the button and invokes the
callback specified with the {\tt command} option.

You can use the {\tt bind} method to override these default
bindings or to add new ones.  For example, this code creates a
binding for a canvas:

\beforeverb
\begin{verbatim}
ca.bind('<ButtonPress-1>', make_circle)
\end{verbatim}
\afterverb
%
The first argument is an event string; this event is triggered
when the user presses the left mouse button.  Other mouse
events include {\tt ButtonMotion}, {\tt ButtonRelease} and
{\tt Double-Button}.

The second argument is an event-handler.  An event-handler
is a function or bound method, like a callback, but an important
difference is an event handler takes an Event object as a
parameter.  Here is an example:

\beforeverb
\begin{verbatim}
def make_circle(event):
    pos = ca.canvas_coords([event.x, event.y])
    item = ca.circle(pos, 5, fill='red')
\end{verbatim}
\afterverb
%
The Event object contains information about the type of event and
details like the coordinates of the mouse pointer.  In this example
the information we need is
the location of the mouse click.  These
values are in ``pixel coordinates,'' which are defined by the
underlying graphical system.  The method {\tt canvas\_coords}
translates them to ``Canvas coordinates,'' which are compatible with
Canvas methods like {\tt circle}.

For Entry widgets, it is common to bind the \verb+<Return>+ event,
which is triggered when the use presses the {\sf Return} or
{\sf Enter} key.  For example, the following code creates a Button
and an Entry.

\beforeverb
\begin{verbatim}
bu = g.bu('Make text item:', make_text)
en = g.en()
en.bind('<Return>', make_text)
\end{verbatim}
\afterverb
%
{\tt make\_text} is called when the Button is pressed or when
the user hits {\sf Return} while typing in the Entry.  To make
this work, we need a function that can be called as a command
(with no arguments) or as an event handler (with an Event
as an argument):

\beforeverb
\begin{verbatim}
def make_text(event=None):
    text = en.get()
    item = ca.text([0,0], text)
\end{verbatim}
\afterverb
%
{\tt make\_text} gets the contents of the Entry and displays
it as a Text item in the Canvas.

It is also possible to create bindings for Canvas items.
The following is a class definition for {\tt Draggable},
which is a child class of {\tt Item} that provides bindings
that implement drag-and-drop capability.

\beforeverb
\begin{verbatim}
class Draggable(Item):

    def __init__(self, item):
        self.canvas = item.canvas
        self.tag = item.tag
        self.bind('<Button-3>', self.select)
        self.bind('<B3-Motion>', self.drag)
        self.bind('<Release-3>', self.drop)
\end{verbatim}
\afterverb
%
The {\tt init} method takes an Item as a parameter.  It copies
the attributes of the Item and then creates bindings for
three events: a button press, button motion, and button release.

The event handler {\tt select} stores the coordinates
of the current event and the original color of the item, then
changes the color to white:

\beforeverb
\begin{verbatim}
    def select(self, event):
        self.dragx = event.x
        self.dragy = event.y

        self.fill = self.cget('fill')
        self.config(fill='white')
\end{verbatim}
\afterverb
%
{\tt cget} stands for ``get configuration;'' it takes the name of an
option as a string and returns the current value of that option.

{\tt drag} computes how far the object has moved relative to the
starting place, updates the stored coordinates, and then moves the
item.

\beforeverb
\begin{verbatim}
    def drag(self, event):
        dx = event.x - self.dragx
        dy = event.y - self.dragy

        self.dragx = event.x
        self.dragy = event.y

        self.move(dx, dy)
\end{verbatim}
\afterverb
%
This computation is done in pixel coordinates; there is no need to
convert to Canvas coordinates.

Finally, {\tt drop} restores the original color of the item:

\beforeverb
\begin{verbatim}
    def drop(self, event):
        self.config(fill=self.fill)
\end{verbatim}
\afterverb
%
You can use the {\tt Draggable} class to add drag-and-drop
capability to an existing item.  For example, here is a modified
version of {\tt make\_circle} that uses {\tt circle} to create
an Item and {\tt Draggable} to make it draggable:

\beforeverb
\begin{verbatim}
def make_circle(event):
    pos = ca.canvas_coords([event.x, event.y])
    item = ca.circle(pos, 5, fill='red')
    item = Draggable(item)
\end{verbatim}
\afterverb
%
One of the benefits of inheritance is that you can modify
the capabilities of a parent class without modifying
its definition.  This is particularly useful if you want to
change behavior defined in a module you did
not write.


%\section{Debugging}

%One of the difficulties of inheritance is that it is not easy
%to find the definition of a method, especially if there are
%many levels in the inheritance hierarchy.

%UML class diagrams.


\section{Glossary}

\begin{description}

\item[GUI:] A graphical user interface.
\index{GUI}

\item[widget:] One of the elements that makes up a GUI, including
buttons, menus, text entry fields, etc. 
\index{widget}

\item[option:] A value that controls the appearance or function of
a widget.
\index{option}

\item[keyword argument:] An argument that indicates the parameter
name as part of the function call.
\index{keyword argument}

\item[callback:] A function associated with a widget that is
called when the user performs an action.
\index{callback}

\item[bound method:] A method associated with a particular instance.
\index{bound method}

\item[event-drive programming:] A style of programming in which
the flow of execution is determined by user actions.
\index{event-drive programming}

\item[event:] A user action, like a mouse click or key press, that
causes a GUI to respond.
\index{event}

\item[item:] A graphical element on a Canvas widget.
\index{item (canvas)}

\item[bounding box:] A rectangle that encloses a set of items,
usually specified by two opposing corners.
\index{bounding box}

\item[pack:] To arrange and display the elements of a GUI.
\index{pack}

\item[geometry manager:] A system for packing widgets.
\index{geometry manager}

\item[binding:] An association between a widget, an event, and
an event handler.  The event handler is called when the the event
occurs in the widget.
\index{binding}

\end{description}


%\section{Exercises}

%\begin{ex}

%\end{ex}




\appendix

\chapter{Debugging}
\index{debugging}

Different kinds of errors can occur
in a program, and it is useful to distinguish among them
in order to track them down more quickly:

\begin{itemize}

\item Syntax errors are produced by Python when it is
translating the source code into byte code.  They usually
indicate that there is something wrong with the syntax of the program.
Example: Omitting the colon at the end of a {\tt def} statement yields
the somewhat redundant message {\tt SyntaxError: invalid syntax}.

\item Runtime errors are produced by the runtime system if something
goes wrong while the program is running.  Most runtime error messages
include information about where the error occurred and what functions
were executing.
Example: An infinite recursion eventually causes
a runtime error of ``maximum recursion depth exceeded.''

\item Semantic errors are problems with a program that compiles and
runs but doesn't do the right thing.  Example: An expression may
not be evaluated in the order you expect, yielding an unexpected
result.

\end{itemize}

\index{compile-time error}
\index{syntax error}
\index{runtime error}
\index{semantic error}
\index{error!compile-time}
\index{error!syntax}
\index{error!runtime}
\index{error!semantic}
\index{exception}

The first step in debugging is to figure out which kind of
error you are dealing with.  Although the following sections are
organized by error type, some techniques are
applicable in more than one situation.


\section{Syntax errors}

\index{error messages}
\index{compiler}

Syntax errors are usually easy to fix once you figure out what they
are.  Unfortunately, the error messages are often not helpful.
The most common messages are {\tt SyntaxError: invalid syntax} and
{\tt SyntaxError: invalid token}, neither of which is very informative.

On the other hand, the message does tell you where in the program the
problem occurred.  Actually, it tells you where Python
noticed a problem, which is not necessarily where the error
is.  Sometimes the error is prior to the location of the error
message, often on the preceding line.

\index{incremental program development}

If you are building the program incrementally, you should have
a good idea about where the error is.  It will be in the last
line you added.

If you are copying code from a book, start by comparing
your code to the book's code very carefully.  Check every character.
At the same time, remember that the book might be wrong, so
if you see something that looks like a syntax error, it might be.

Here are some ways to avoid the most common syntax errors:

\index{syntax}

\begin{enumerate}

\item Make sure you are not using a Python keyword for a variable name.

\item Check that you have a colon at the end of the header of every
compound statement, including {\tt for}, {\tt while},
{\tt if}, and {\tt def} statements.

\item Check that indentation is consistent.  You may indent with either
spaces or tabs but it's best not to mix them.  Each level should be
nested the same amount.

\item Make sure that any strings in the code have matching
quotation marks.

\item If you have multiline strings with triple quotes (single or double), make
sure you have terminated the string properly.  An unterminated string
may cause an {\tt invalid token} error at the end of your program,
or it may treat the following part of the program as a string until it
comes to the next string.  In the second case, it might not produce an error
message at all!

\item An unclosed bracket---\verb+(+, \verb+{+, or \verb+[+---makes
Python continue with the next line as part of the current statement.
Generally, an error occurs almost immediately in the next line.

\item Check for the classic {\tt =} instead of {\tt ==} inside
a conditional.

\end{enumerate}

If nothing works, move on to the next section...


\subsection{I can't get my program to run no matter
what I do.}

If the compiler says there is an error and you don't see it, that
might be because you and the compiler are not looking at the same
code.  Check your programming environment to make sure that the
program you are editing is the one Python is trying to run.  If you
are not sure, try putting an obvious and deliberate syntax error at
the beginning of the program.  Now run (or import) it again.  If the
compiler doesn't find the new error, there is probably something wrong
with the way your environment is set up.  

If this happens, one approach is to start again with a new
program like ``Hello, World!,'' and make sure you can get a known
program to run.  Then gradually add the pieces of the new program
to the working one.



\section{Runtime errors}

Once your program is syntactically correct,
Python can import it and at least start running it.  What could
possibly go wrong?


\subsection{My program does absolutely nothing.}

This problem is most common when your file consists of functions and classes
but does not actually invoke anything to start execution.  This may be
intentional if you only plan to import this module to supply classes
and functions.

If it is not intentional, make sure that you
are invoking a function to start execution, or execute one from
the interactive prompt.  Also see the ``Flow of Execution'' section
below.


\subsection{My program hangs.}
\index{infinite loop}
\index{infinite recursion}
\index{hanging}

If a program stops and seems to be doing nothing, we
say it is ``hanging.''  Often that means that it is caught in
an infinite loop or an infinite recursion.

\begin{itemize}

\item If there is a particular loop that you suspect is the
problem, add a {\tt print} statement immediately before the loop that says
``entering the loop'' and another immediately after that says
``exiting the loop.''

Run the program.  If you get the first message and not the second,
you've got an infinite loop.  Go to the ``Infinite Loop'' section
below.

\item Most of the time, an infinite recursion will cause the program
to run for a while and then produce a ``RuntimeError: Maximum
recursion depth exceeded'' error.  If that happens, go to the
``Infinite Recursion'' section below.

If you are not getting this error but you suspect there is a problem
with a recursive method or function, you can still use the techniques
in the ``Infinite Recursion'' section.

\item If neither of those steps works, start testing other
loops and other recursive functions and methods.

\item If that doesn't work, then it is possible that
you don't understand the flow of execution in your program.
Go to the ``Flow of Execution'' section below.

\end{itemize}


\subsubsection{Infinite Loop}
\index{infinite loop}
\index{loop!infinite}
\index{condition}
\index{loop!condition}

If you think you have an infinite loop and you think you know
what loop is causing the problem, add a {\tt print} statement at
the end of the loop that prints the values of the variables in
the condition and the value of the condition.

For example:

\beforeverb
\begin{verbatim}
while x > 0 and y < 0 :
    # do something to x
    # do something to y

    print  "x: ", x
    print  "y: ", y
    print  "condition: ", (x > 0 and y < 0)
\end{verbatim}
\afterverb
%
Now when you run the program, you will see three lines of output
for each time through the loop.  The last time through the
loop, the condition should be {\tt false}.  If the loop keeps
going, you will be able to see the values of {\tt x} and {\tt y},
and you might figure out why they are not being updated correctly.


\subsubsection{Infinite Recursion}
\index{infinite recursion}
\index{recursion!infinite}

Most of the time, an infinite recursion will cause the program to run
for a while and then produce a {\tt Maximum recursion depth exceeded}
error.

If you suspect that a function or method is causing an infinite
recursion, start by checking to make sure that there is a base case.
In other words, there should be some condition that will cause the
function or method to return without making a recursive invocation.
If not, then you need to rethink the algorithm and identify a base
case.

If there is a base case but the program doesn't seem to be reaching
it, add a {\tt print} statement at the beginning of the function or method
that prints the parameters.  Now when you run the program, you will see
a few lines of output every time the function or method is invoked,
and you will see the parameters.  If the parameters are not moving
toward the base case, you will get some ideas about why not.


\subsubsection{Flow of Execution}
\index{flow of execution}
\index{execution!flow}

If you are not sure how the flow of execution is moving through
your program, add {\tt print} statements to the beginning of each
function with a message like ``entering function {\tt foo},'' where
{\tt foo} is the name of the function.

Now when you run the program, it will print a trace of each
function as it is invoked.


\subsection{When I run the program I get an exception.}
\index{exception}
\index{runtime error}

If something goes wrong during runtime, Python
prints a message that includes the name of the
exception, the line of the program where the problem occurred,
and a traceback.

\index{traceback}

The traceback identifies the function that is currently running,
and then the function that invoked it, and then the function that
invoked {\em that}, and so on.  In other words, it traces the
path of function invocations that got you to where you are.  It
also includes the line number in your file where each of these
calls occurs.

The first step is to examine the place in the program where
the error occurred and see if you can figure out what happened.
These are some of the most common runtime errors:

\begin{description}

\item[NameError:]  You are trying to use a variable that doesn't
exist in the current environment.
Remember that local variables are local.  You
cannot refer to them from outside the function where they are defined.

\index{NameError}
\index{TypeError}

\item[TypeError:] There are several possible causes:

\begin{itemize}

\item  You are trying to use a value improperly.  Example: indexing
a string, list, or tuple with something other than an integer.

\index{index}

\item There is a mismatch between the items in a format string and
the items passed for conversion.  This can happen if either the number
of items does not match or an invalid conversion is called for.

\index{format operator}
\index{operator!format}

\item You are passing the wrong number of arguments to a function or method.
For methods, look at the method definition and
check that the first parameter is {\tt self}.  Then look at the
method invocation; make sure you are invoking the method on an
object with the right type and providing the other arguments
correctly.

\end{itemize}

\item[KeyError:]  You are trying to access an element of a dictionary
using a key value that the dictionary does not contain.

\index{KeyError}
\index{dictionary}

\item[AttributeError:] You are trying to access an attribute or method
that does not exist.

\index{AttributeError}

\item[IndexError:] The index you are using
to access a list, string, or tuple is greater than
its length minus one.  Immediately before the site of the error,
add a {\tt print} statement to display
the value of the index and the length of the array.
Is the array the right size?  Is the index the right value?

\index{IndexError}

\end{description}


\subsection{I added so many {\tt print} statements I get inundated with
output.}
\index{print statement}
\index{statement!print}

One of the problems with using {\tt print} statements for debugging
is that you can end up buried in output.  There are two ways
to proceed: simplify the output or simplify the program.

To simplify the output, you can remove or comment out {\tt print}
statements that aren't helping, or combine them, or format
the output so it is easier to understand.

To simplify the program, there are several things you can do.  First,
scale down the problem the program is working on.  For example, if you
are sorting an array, sort a {\em small} array.  If the program takes
input from the user, give it the simplest input that causes the
problem.

Second, clean up the program.  Remove dead code and reorganize the
program to make it as easy to read as possible.  For example, if you
suspect that the problem is in a deeply nested part of the program,
try rewriting that part with simpler structure.  If you suspect a
large function, try splitting it into smaller functions and testing them
separately.

Often the process of finding the minimal test case leads you
to the bug.  If you find that a program works
in one situation but not in another,
that gives you a clue about what is going on.

Similarly, rewriting a piece of code can help you find subtle
bugs.  If you make a change that you think doesn't affect the
program, and it does, that can tip you off.


\section{Semantic errors}
\index{semantic error}
\index{error!semantic}

In some ways, semantic errors are the hardest to debug,
because the compiler and the runtime system provide no information
about what is wrong.  Only you know what the program is supposed to
do, and only you know that it isn't doing it.

The first step is to make a connection between the program
text and the behavior you are seeing.  You need a hypothesis
about what the program is actually doing.  One of the things
that makes that hard is that computers run so fast.

You will often wish that you could slow the program down to human
speed, and with some debuggers you can.  But the time it takes to
insert a few well-placed {\tt print} statements is often short compared to
setting up the debugger, inserting and removing breakpoints, and
``walking'' the program to where the error is occurring.

\subsection{My program doesn't work.}

You should ask yourself these questions:

\begin{itemize}

\item Is there something the program was supposed to do but
which doesn't seem to be happening?  Find the section of the code
that performs that function and make sure it is executing when
you think it should.

\item Is something happening that shouldn't?  Find code in
your program that performs that function and see if it is
executing when it shouldn't.

\item Is a section of code producing an effect that is not
what you expected?  Make sure that you understand the code in
question, especially if it involves invocations to functions or methods in
other Python modules.  Read the documentation for the functions you invoke.
Try them out by writing simple test cases and checking the results.

\end{itemize}

In order to program, you need to have a mental model of how
programs work.  If you write a program that doesn't do what you expect,
very often the problem is not in the program; it's in your mental
model.

\index{model!mental}
\index{mental model}

The best way to correct your mental model is to break the program
into its components (usually the functions and methods) and test
each component independently.  Once you find the discrepancy
between your model and reality, you can solve the problem.

Of course, you should be building and testing components as you
develop the program.  If you encounter a problem,
there should be only a small amount of new code
that is not known to be correct.


\subsection{I've got a big hairy expression and it doesn't
do what I expect.}
\index{expression!big and hairy}

Writing complex expressions is fine as long as they are readable,
but they can be hard to debug.  It is often a good idea to
break a complex expression into a series of assignments to
temporary variables.

For example:

\beforeverb
\begin{verbatim}
self.hands[i].addCard(self.hands[self.findNeighbor(i)].popCard())
\end{verbatim}
\afterverb
%
This can be rewritten as:

\beforeverb
\begin{verbatim}
neighbor = self.findNeighbor(i)
pickedCard = self.hands[neighbor].popCard()
self.hands[i].addCard(pickedCard)
\end{verbatim}
\afterverb
%
The explicit version is easier to read because the variable
names provide additional documentation, and it is easier to debug
because you can check the types of the intermediate variables
and display their values.

\index{temporary variable}
\index{variable!temporary}
\index{order of evaluation}
\index{precedence}

Another problem that can occur with big expressions is
that the order of evaluation may not be what you expect.
For example, if you are translating the expression
$\frac{x}{2 \pi}$ into Python, you might write:

\beforeverb
\begin{verbatim}
y = x / 2 * math.pi
\end{verbatim}
\afterverb
%
That is not correct because multiplication and division have
the same precedence and are evaluated from left to right.
So this expression computes $x \pi / 2$.

A good way to debug expressions is to add parentheses to make
the order of evaluation explicit:

\beforeverb
\begin{verbatim}
 y = x / (2 * math.pi)
\end{verbatim}
\afterverb
%
Whenever you are not sure of the order of evaluation, use
parentheses.  Not only will the program be correct (in the sense
of doing what you intended), it will also be more readable for
other people who haven't memorized the rules of precedence.


\subsection{I've got a function or method that doesn't return what I
expect.}
\index{return statement}
\index{statement!return}

If you have a {\tt return} statement with a complex expression,
you don't have a chance to print the {\tt return} value before
returning.  Again, you can use a temporary variable.  For
example, instead of:

\beforeverb
\begin{verbatim}
return self.hands[i].removeMatches()
\end{verbatim}
\afterverb
%
you could write:

\beforeverb
\begin{verbatim}
count = self.hands[i].removeMatches()
return count
\end{verbatim}
\afterverb
%
Now you have the opportunity to display the value of
{\tt count} before returning.


\subsection{I'm really, really stuck and I need help.}

First, try getting away from the computer for a few minutes.
Computers emit waves that affect the brain, causing these
effects:

\begin{itemize}

\item Frustration and/or rage.

\item Superstitious beliefs (``the computer hates me'') and
magical thinking (``the program only works when I wear my
hat backward'').

\item Random-walk programming (the attempt to program by writing
every possible program and choosing the one that does the right
thing).

\end{itemize}

If you find yourself suffering from any of these symptoms, get
up and go for a walk.  When you are calm, think about the program.
What is it doing?  What are some possible causes of that
behavior?  When was the last time you had a working program,
and what did you do next?

Sometimes it just takes time to find a bug.  We often find bugs
when we are away from the computer and let our minds wander.  Some
of the best places to find bugs are trains, showers, and in bed,
just before you fall asleep.


\subsection{No, I really need help.}

It happens.  Even the best programmers occasionally get stuck.
Sometimes you work on a program so long that you can't see the
error.  A fresh pair of eyes is just the thing.

Before you bring someone else in, make sure you have exhausted
the techniques described here.  Your program should be as simple
as possible, and you should be working on the smallest input
that causes the error.  You should have {\tt print} statements in the
appropriate places (and the output they produce should be
comprehensible).  You should understand the problem well enough
to describe it concisely.

When you bring someone in to help, be sure to give
them the information they need:

\begin{itemize}

\item If there is an error message, what is it
and what part of the program does it indicate?

\item What was the last thing you did before this error occurred?
What were the last lines of code that you wrote, or what is
the new test case that fails?

\item What have you tried so far, and what have you learned?

\end{itemize}

When you find the bug, take a second to think about what you
could have done to find it faster.  Next time you see something
similar, you will be able to find the bug more quickly.

Remember, the goal is not just to make the program
work.  The goal is to learn how to make the program work.



\printindex

\clearemptydoublepage
%\blankpage
%\blankpage
%\blankpage


\end{document}
